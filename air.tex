\documentclass[utf8]{frontiersSCNS} % for Science, Engineering and Humanities and Social Sciences articles
% \documentclass[utf8]{frontiersFPHY} % for Physics and Applied Mathematics and Statistics articles
\usepackage{gensymb}
\usepackage{url,hyperref,lineno,microtype,subcaption}
\usepackage[onehalfspacing]{setspace}

\usepackage{tabularx}
\linenumbers
\DeclareUnicodeCharacter{0301}{}
\DeclareUnicodeCharacter{2212}{}
\usepackage{wasysym} % provides \DH, \dh, \Thorn, \thorn
% Leave a blank\usepackage{amsmath}
%\DeclareMathOperator{\sign}{sign} line between paragraphs instead of using \\

\usepackage{csvsimple} % for csv tables
\usepackage{booktabs}
\usepackage{multirow}

\def\keyFont{\fontsize{8}{11}\helveticabold }
\def\firstAuthorLast{Balasubramanian {et~al.}} %use et al only if is more than 1 author
\def\Authors{Suryanarayanan Balasubramanian\,$^{1*}$, Martin Hoelzle\,$^{1}$, Michael Lehning\,$^{2}$, Sonam
	Wangchuk\,$^{3}$, Johannes Oerlemans\,$^{4}$, Felix Keller\,$^{5,6}$ and Jordi Bolibar\,$^{4}$}
\def\Address{$^{1}$University of Fribourg, Fribourg, Switzerland\\
	$^{2}$WSL Institute for Snow and Avalanche Research, Davos, Switzerland\\
	$^{3}$Himalayan Institute of Alternatives Ladakh, Leh, India\\
	$^{4}$Institute for Marine and Atmospheric Research, Utrecht University, Utrecht, The Netherlands\\
	$^{5}$Academia Engiadina, Samedan, Switzerland\\
	$^{6}$ETH, Zürich, Switzerland}
\def\corrAuthor{Suryanarayanan Balasubramanian}

\def\corrEmail{suryanarayanan.balasubramanian@unifr.ch}



\begin{document}
\onecolumn
\firstpage{1}

\title[Artificial Ice Reservoirs]{Quantifying environmental influences on artificial ice reservoir (Icestupa)
	evolution: case studies from the Swiss Alps and Indian Himalayas}

\author[\firstAuthorLast ]{\Authors}
\address{}
\correspondance{}

\extraAuth{}

\maketitle


\begin{abstract}

	Artificial Ice Reservoirs (AIR) can freeze water during winter and generate meltwater for irrigation during spring and
	summer. Several AIRs have been built worldwide but studies of their water storage efficiency are
	scarce.  This study models processes involved in the temporal and spatial evolution of a cone-shaped AIR popularly
	called Icestupa.  These processes were quantified using meteorological data in conjunction with fountain discharge
	information (mass input of an AIR) to estimate the quantity of frozen, melted, evaporated and runoff water for two
	sites in Switzerland and one in India. At these measurement sites, AIRs were built for model calibration and validation
	purposes. Model estimates of freezing (melting) rates were more than 5 times higher (2 times lower) for the Indian
	compared to the Swiss AIR. This was a result of the much higher sublimation rates (much lower humidity) and fountain
	spray radius of the Indian AIR.  However, the storage efficiency of the Indian AIR (15 \%) was actually lower than the
	Swiss AIR (~20 \%) due to significantly more fountain water runoff.

	\tiny
	\keyFont{ \section{Keywords:} icestupa, water storage, climate change adaptation, geoengineering } %All article types: you may provide up to 8 keywords; at least 5 are mandatory.
\end{abstract}

\section{Introduction}

Seasonal snow cover, glaciers and permafrost are expected to change their water storage capacity due to climate change
with major consequences for downriver water supply \citep{Immerzeel_2020}. The challenges brought about by these changes
are especially important for dry mountain environments such as in Central Asia or the Andes, which directly rely on the
seasonal meltwater for their farming and drinking needs \citep{HoelzleBarandun_2019, Apel_2018, Buytaert_2017,
	Chen_2016, UNGERSHAYESTEH_2013}. Some villages in Ladakh, India have already been forced to relocate due to glacial
retreat and the corresponding loss of their main fresh water resources \citep{zanskar}.

\begin{figure} \begin{center} \includegraphics[width=10 cm]{Figures/Figure_1.jpg}
	\end{center} \caption{Icestupa in Ladakh, India on March 2017 was 24 $m$ tall and contained around 3700 $m^3$
		of water. Picture Credits: Lobzang Dadul} \label{fig:cone} \end{figure}

Artificial ice reservoirs (AIR) have been considered to be a feasible way to adapt to these changes \citep{IPCC_2019,
	10.1659/MRD-JOURNAL-D-18-00072.1}. An AIR is a human-made ice structure typically constructed during the cold winter
months and designed to slowly release freshwater during the warm spring and summer months. The main purpose of
AIRs is irrigation. Therefore, AIRs are designed to store water in the form of ice as long into the summer as possible.
The energy required to construct an AIR is usually derived from the gravitational head of the source water body. Some
are constructed horizontally by freezing water using a series of checkdams and others are built vertically by spraying
water through fountain systems \citep{Nusser_2018}. The latter are colloquially referred to as Icestupas and are the
subject of this study.

A typical AIR just requires a fountain nozzle mounted on a supply pipeline. The water source is usually a high altitude
lake or glacial stream. Due to the altitude difference between the pipeline input and fountain output, water ejects from
the fountain nozzle as droplets that eventually lose their latent heat to the atmosphere and accumulate as ice. The
fountain nozzle is raised through addition of metal pipes as and when significant ice accumulates.  Typically, a dome of
branches is constructed around the metal pipes so that such pipe extensions can be done from within this dome. During
the winter, the fountain is manually activated from sunset to sunrise. Threads, tree branches and fishing nets are used
to guide and accelerate the ice formation.

Since their invention in 2013 \citep{campaign}, Icestupas have gained widespread publicity in the region of Ladakh,
Northern India since they require very little infrastructure, skills and energy to be constructed in comparison to other
water storage technologies. Compared to other AIR geometries, Icestupas (Fig. \ref{fig:cone}) can be built at lower
altitudes and last much longer into the summer than other types of ice structures \citep{campaign}. However, to date, no
reliable estimates exist about the quantity of meltwater they can provide \citep{Nusser_2018}.

In this paper, we aim to develop a physically-based model of a vertical AIR (or Icestupa) that can quantify their
storage efficiency using weather and fountain discharge information. Mass and energy balance equations were used to
estimate the quantity of water frozen, melted, evaporated and runoff. Sensitivity and uncertainty analysis were
performed to identify the most critical parameters and the variance caused by them. For validation, we chose three AIR
built across the winters of 2020 and 2021 in India and Switzerland. Our model and validation experiments provide first
steps towards evaluating the potential of this new water storage technology worldwide.

\section{Study Sites}
To accurately calibrate, estimate and validate the ice volume of AIRs the model requires three kinds of datasets
containing weather, fountain and AIR measurements. Through the winters of 2019, 2020 and 2021 several scientific AIRs
were constructed by teams in Switzerland and India. Here we present the results of three scientific AIRs which have a
relatively complete dataset associated with them. As shown in Table \ref{tab:Observations}, two of them were
constructed in the same swiss location called Guttannen (referred with the prefix CH) but during different winters and
the other was constructed at Gangles, India (referred with the prefix IN).

The Guttannen site (46.66 $\degree$N, 8.29 $\degree$E) in the Bern region lies at 1047 $m$ a.s.l.. In the winter (Oct-Apr), mean
daily minimum and maximum air temperatures vary between -13 and 15 $\degree C$. Clear skies are rare, averaging around 7
days during winter \citep{eispalast}. The site was situated adjacent to a stream resulting in high humidity values
across the study period as shown in Fig. \ref{fig:CHsite}. AIR were constructed here by the Guttannen Bewegt Association
during the winters of 2019-20 and 2020-21. To initiate the ice formation process, tree branches were laid covering the
fountain pipe.  The fountain height varied between 2 to 5\,$m$ during the construction period. The water was transferred
from a spring water source and flowed via a flowmeter to the nozzle. In addition, a webcam guaranteed a continuous
survey of the site during the construction of the AIR.

\begin{figure}
	\begin{center}
		\includegraphics[width=12 cm]{Figures/2AIR.jpg}
	\end{center}
	\caption{The Swiss and Indian AIR on March 3 and January 9, 2021 respectively. Picture credits: Daniel Buerki (left)
		and Thinles Norboo (right)}
	\label{fig:CHsite}
\end{figure}

The Gangles site (34.22 $\degree$N, 77.61 $\degree$E) is located around 20 km north of Leh city in the Ladakh region,
lying at 4025 $m$ a.s.l.. AIR were constructed here by the Himalayan Institute of Alternatives, Ladakh (HIAL) every
winter since 2018. The IN21 AIR was constructed as part of the Icestupa Competition in Gangles, Ladakh, India. It was
constructed adjacent to another AIR and merged with it. It's fountain height varied between 5 to 9\,$m$.

\subsection{Meteorological data}
Air temperature, relative humidity, wind speed, pressure, longwave, shortwave direct and diffuse radiation are required
to calculate the surface energy balance of an AIR.

For the CH site, the primary weather data source was a meteoswiss AWS located 184 m away. In addition, we used ERA5
reanalysis dataset \citep{era5} for filling data gaps and adding data that were not measured directly. Zero wind speed values were recorded whenever snow accumulated on the ultrasonic wind sensor. It was
assumed this was the cause when null wind speeds were observed continuously for atleast 3 hours. All such null values
were replaced using the ERA5 dataset.

The ERA5 reanalysis dataset has a good correlation with sites in Switzerland \citep{Scherrer_2020}. The ERA5 grid point
chosen (46.64 $\degree$N, 8.25 $\degree$E) for the swiss site was around 3.6 km away from the actual site.  All the ERA5
variables were therefore fitted with the meteoswiss dataset via linear regressions.

For the IN site, three different weather data sources were used to log all the weather parameters required for the
model. A temperature and humidity logger was placed adjacent to the AIR on a mast. Wind speed and pressure data was
logged via a campbell weather station located 440 m away. Shortwave radiation data was derived from another campbell
weather station located 15 km away. Unfortunately, precipitation was not logged and was assumed to be negligible.
The diffuse fraction of the global shortwave radiation was also assumed to be negligible.

\begin{table}
	\centering
	\caption{ Summary of the weather, fountain and AIR observations}
	\label{tab:Observations}
	\begin{tabular}{@{}|llllll|@{}}
		\toprule
		\textbf{}              & \textbf{Name}               & \textbf{Symbol}           & \textbf{IN21} & \textbf{CH21} & \textbf{CH20} \\ \midrule
		\multicolumn{1}{|l|}{\multirow{7}{*}{\rotatebox[origin=c]{90}{Weather}}}
		                       & Mean discharge              & $d_F [l/min]        $     & $60 \pm 0$    & $7.5 \pm 0$   & $7.5 \pm 0$   \\
		\multicolumn{1}{|l|}{} & Air temperature             & $T_a [\degree C]    $     & $0 \pm 6$     & $1 \pm 5$     & $2 \pm 4$     \\
		\multicolumn{1}{|l|}{} & Relative humidity           & $RH  [\%]        $        & $39 \pm 17$   & $79 \pm 17$   & $75 \pm 17$   \\
		\multicolumn{1}{|l|}{} & Wind speed                  & $v_a [m/s]        $       & $2 \pm 1$     & $2 \pm 2$     & $2 \pm 2$     \\
		\multicolumn{1}{|l|}{} & Direct Shortwave            & $SW_{direct} [W\,m^{-2}]$ & $243 \pm 331$ & $77 \pm 152$  & $83 \pm 156$  \\
		\multicolumn{1}{|l|}{} & Diffuse Shortwave           & $SW_{diffuse}[W\,m^{-2}]$ & $0 \pm 0$     & $54 \pm82$    & $51 \pm 73$   \\
		\multicolumn{1}{|l|}{} & Incoming Longwave Radiation & $LW_{in}[W\,m^{-2}]$      & NA            & $54 \pm82$    & $51 \pm 73$   \\
		\multicolumn{1}{|l|}{} & Hourly Precipitation        & $ppt [mm]       $         & $0 \pm 0$     & $129 \pm 446$ & $92 \pm 400$  \\
		\multicolumn{1}{|l|}{} & Pressure                    & $p_a [hPa]        $       & $622 \pm 3$   & $793 \pm 8$   & $797 \pm7$    \\\bottomrule
		\multicolumn{1}{|l|}{\multirow{3}{*}{\rotatebox[origin=c]{90}{Fountain}}}
		                       & Mean discharge              & $d_F [l/min]     $        & $60$          & $7.5$         & $7.5$         \\
		\multicolumn{1}{|l|}{} & Fountain Duration           & $[hours]$                 & 829           & 2155          & 1553          \\
		\multicolumn{1}{|l|}{} & Spray radius                & $r_{F} [m]$               & 10.8          & 6.9           & 7.7           \\\midrule
		\multicolumn{1}{|l|}{\multirow{4}{*}{\rotatebox[origin=c]{90}{UAV}}}
		% & Altitude  && 4025 $m$ a.s.l.& 1047 $m$ a.s.l.& 1047 $m$ a.s.l.\\ 
		                       & Dome volume                 & $V_{dome}[m^{3}]$         & 78.5 $m^{3}$  & 13.2 $m^{3}$  & 23.9 $m^{3}$  \\
		\multicolumn{1}{|l|}{} & Volume measurements         & [times]                   & 6             & 9             & 3             \\
		\multicolumn{1}{|l|}{} & AIR starts                  &                           & Jan 18        & Nov 22        & Jan 3         \\
		\multicolumn{1}{|l|}{} & AIR ends                    &                           & Jun 11        & May 18        & Apr 10        \\ \bottomrule
	\end{tabular}
\end{table}

\subsection{UAV surveys}
In this study, an Anafi thermal quadcopter uncrewed aerial vehicle (UAV) was employed to conduct several surveys of the
AIR (refer to Table ).  The details of these surveys are shown in Figure . The UAV flew automatically along the flight
course predefined by Pix4Dcapture (https://www.pix4d.com/product/pix4dcapture) and took photographs at a certain time
interval. The position and altitude of the UAV at the exposure stations, which were obtained by the built-in integrated
Position and Orientation System (POS, composed of global positioning system and inertial measurement units), were
recorded in the JPEG pictures. UAV images in each survey were separately processed with Pix4Dmapper in a three-step
workflow, which is described below:

(1) Initial processing: This process generates a sparse point cloud with the structure-from motion algorithm
(\cite{Turner_2012}). First, it searches for and matches key points in the photos that have certain overlapping areas using a
feature matching algorithm (e.g. the scale-invariant feature transform (SIFT) algorithm, which can detect key points in
photos with different views and illumination conditions; \cite{Lowe_2004}). Second, the approximate locations and
orientations of the camera at each exposure station are reconstructed with the internal parameters (focal length,
coordinates of the principal point of the photograph), and external parameters (i.e. POS data). A sparse point cloud is
created.

(2) Point cloud densification: In this step, the multi-view stereo technique is applied to achieve a higher point cloud
density than in the previous step (\cite{Furukawa_2010}; \cite{Molg_2017}). Thus, the spatial resolution of the
products can be increased, and an irregular network for the next step can be created (\cite{Kung_2011}).

(3) Digital elevation model (DEM) and orthomosaic generation: DEM and orthomosaic are the two main final products. DEM
can be built from dense point cloud or irregular network, and the former one usually has higher accuracy for rugged
terrain. Every image pixel is projected on DEM and the georeferenced orthomosaic is generated (\cite{Kung_2011}).

During this process, we found photos with too much snow coverage that cannot be matched, that is, few feature points can
be detected and matched in these photos, especially in survey 2 of CH20, where almost half of the photos are invalid
(Table).

Several UAV surveys were conducted in the Swiss and Indian sites. The DEM generated through these flights were
analysed to obtain the circumference and volume of the ice structure. The first drone flight was used to set the dome
volume ($V_{dome}$) for model initialisation. Since the Indian AIR was built on top of another ice structure, it had
a much higher dome volume compared to the other AIRs.

\subsection{Fountain observations}
We define the fountain used in AIR construction through three attributes, namely its spray radius , mean discharge
quantity and discharge duration as shown in Table \ref{tab:Observations}. Continuous measurement of the discharge rate
was unsuccessful in all the sites. Instead the discharge duration was first determined and then the available discharge
measurement was used to determine the average discharge quantity $d_F$ during these periods as shown in the Appendix. The
spray radius $r_F$ was estimated from the mean AIR circumference measured in the drone flights during the fountain
duration.

For the Swiss site, the fountain was never switched off so the discharge duration was extrapolated from just one fountain
on and off event each.

For the Indian site, even though the fountain was never manually switched off, there were many pipeline freezing events that
interrupted the discharge duration. So discharge rate was extrapolated to be the mean discharge $d_F$ except during
these pipeline freezing events.

\section{Model setup}

A bulk energy and mass balance model is used to calculate the amounts of ice, meltwater, water vapour and runoff water
of the AIR every hour. This model consists of four modules which estimates the AIR, a) geometric evolution , b) energy
balance, c) surface temperature d) mass balance and e) parameter sensitivity.

\begin{figure} \begin{center} \includegraphics[width=12 cm]{Figures/model_schematic.png} \end{center} \caption{Model
		schematic showing the algorithm used in the model at every time step. } \label{fig:schema} \end{figure}

\subsection{Geometric evolution}

Radius $r_{ice}^i$ and height $h_{ice}^i$ define the dimensions of the AIR assuming its geometry to be a cone. The
surface area $A^i$ exposed to the atmosphere and volume $V^i$ are:

\begin{equation} A = \pi \cdot r_{ice} \cdot \sqrt{{r_{ice}}^2 + {h_{ice}}^ 2} \label{eqn:A} \end{equation}

\begin{equation} V = \pi/3 \cdot {r_{ice}}^2 \cdot h_{ice} \label{eqn:V} \end{equation}

Note that we do not specify the time step superscript $i$ of the shape variables $A$, $V$, $r_{ice}$ and $h_{ice}$ for
brevity.  The equations used henceforth display model time step superscript $i$ only if it is different from the
current time step.

With the mass of the AIR $M_{ice}$, its current volume can also be expressed as:

\begin{equation} V = M_{ice} /\rho_{ice} \label{eqn:V1} \end{equation}

where $\rho_{ice}$ is the density of ice (917 $kg\, m^{-3}$).


The influence of the AIR fountain is parameterised by the fountain water temperature $T_{w}$ and its spray radius $r_F$.
The initial radius of the AIR is assumed to be $r_F$. The initial height $h_0$ depends on the dome volume
$V_{dome}$ used to construct the AIR as follows:

\begin{equation}
	h_{0} =  \Delta x + \frac{3 \cdot V_{dome}}{\pi r_F^2 }
	\label{eqn:h0}
\end{equation}

where $\Delta x$ is the surface layer thickness (defined in Section \ref{section:EB})

During subsequent time steps, the dimensions of the AIR evolve assuming a uniform ice formation and decay across
its surface area with an invariant slope $s_{cone} = \frac{h_{ice}}{r_{ice}}$ .  During
these time steps, the volume is parameterised using Eqn. \ref{eqn:V} as:

\begin{equation} V = \frac{\pi \cdot {r_{ice}}^3
		\cdot s_{cone}}{3} \label{eqn:V2} \end{equation}


However, the Icestupa cannot outgrow the maximum range of the water droplets ($(r_{ice})_{max} = r_{F}$). Combining
Eqns. \ref{eqn:V},  \ref{eqn:V1}, \ref{eqn:h0} and \ref{eqn:V2}, the geometric evolution of the Icestupa at each time step $i$ can
be determined by considering the following rules:

\begin{equation} (r_{ice},\, h_{ice}) = \left\{ \begin{array}{ll} (r_F ,\, h_0)                                                                        & \textit{ if } i=0 \\
             (r_{ice}^{i-1},\, \frac{3 \cdot M_{ice}}{\pi \cdot \rho_{ice} \cdot {(r_{ice}^{i-1})}^2}) & \textit{ if }
             r_{ice}^{i-1} \geq r_{F} \textit{ and } \Delta M_{ice} > 0                                                    \\ (\frac{3 \cdot M_{ice}}{\pi \cdot \rho_{ice} \cdot s_{cone}})^{1/3} \cdot (1,\,  s_{cone}) &
             otherwise\end{array} \right.  \label{eqn:A2} \end{equation}

where $\Delta M_{ice} = M_{ice}^{i-1} - M_{ice}^{i-2}$

\begin{figure} \begin{center} \includegraphics[width=10
			cm]{Figures/Figure_5.jpg} \end{center} \caption{Shape variables and fountain constants of the AIR. $r_{ice}$ is
		the radius, $h_{ice}$ is the height and $s_{cone}$ is the slope of the ice cone. $r_F$ is the spray radius, $h_F$ is the
		height and $dia_F$ is the nozzle diameter of the fountain.} \label{fig:shape} \end{figure}

\subsection{Energy Balance} \label{section:EB}

The energy balance equation \citep{Hock_2005} for the AIR is formulated as follows:

\begin{equation} q_{SW} + q_{LW} + q_{L} + q_{S} + q_{F} + q_{G} = q_{surf} \label{eqn:EB} \end{equation}

where $q_{surf}$ is the surface energy flux in [$W\,m^{-2}$]; $q_{SW}$ is the net shortwave radiation; $q_{LW}$ is the
net longwave radiation; $q_{L}$ and $q_{S}$ are the turbulent latent and sensible heat fluxes. $q_{F}$ represents the heat
exchange of the fountain water droplets with the AIR ice surface. $q_{G}$ represents ground heat flux between the AIR
surface and its interior. Energy transferred in the direction of the ice surface is always denoted as positive and
away as negative.

Equation \ref{eqn:EB} is usually referred to as the energy budget for “the surface”, but practically it must apply to
a surface layer of ice with a finite thickness $\Delta x$. The energy flux acts upon the AIR surface layer which
has an upper and a lower boundary defined by the atmosphere and the ice body of the AIR, respectively. The
parameter selection for $\Delta x$ is based on the following two arguments: (a) the ice thickness $\Delta x$ should be
small enough to represent the surface temperature variations every model time step $\Delta t$ and (b) $\Delta x$ should
be large enough for these temperature variations to not reach the bottom of the surface layer.  Therefore, we introduced
a 20 $mm$ thick surface layer for a model time step of 1 hour, over which the energy balance is calculated. A
sensitivity analysis was later performed to understand the influence of this factor. Here, we define the surface
temperature $T_{ice}$ to be the modelled average temperature of the Icestupa surface layer and the energy flux $q_{surf}$
is assumed to act uniformly across the Icestupa area $A$.

\subsubsection{Net Shortwave Radiation \texorpdfstring{$q_{SW}$}{Lg}}
The net shortwave radiation $q_{SW}$ is computed as follows:
\begin{equation} q_{SW} = (1- \alpha)\cdot (SW_{direct} \cdot f_{cone} + SW_{diffuse}) \label{eqn:SW} \end{equation}

where $SW_{direct}$ and $SW_{diffuse}$ are the ERA5 direct and diffuse short wave radiation, $\alpha$ is the modelled
albedo and $f_{cone}$ is the area fraction of the ice structure exposed to the direct shortwave radiation.

The albedo varies depending on the water source that formed the current AIR surface layer. So during the fountain
duration, the albedo assumes a constant value corresponding to ice albedo. But after the fountain is switched off, the
albedo can reset to snow albedo during snowfall events and then decay back to ice albedo. To model this process, we use
the scheme described in \cite{OerlemansKnap_1998}. The scheme records the decay of albedo with time after fresh snow is
deposited on the surface. $\delta t$ records the number of time steps after the last snowfall event. After snowfall,
albedo changes over a time step, $\delta t$ , as

\begin{equation} \alpha=\alpha_{ice}+(\alpha_{snow}-\alpha_{ice}) \cdot e^{(-\delta t)/\tau} \label{eqn:a}
\end{equation}

where $\alpha_{ice}$ is the bare ice albedo value (0.35), $\alpha_{snow}$ is the snow ice albedo value (0.85) and $\tau$
is a decay rate, which determines how fast the albedo of the ageing snow reaches this value.  The decay rate $\tau$ is
assumed to have a intial value of 10 days similar to values obtained by \cite{Schmidt_2017} for wet surfaces. Snowfall
events are assumed if the air temperature is below $T_{ppt}=1 \degree C$ \citep{FujitaAgeta_2000}.

The area fraction $f_{cone}$ of the ice structure exposed to the direct shortwave radiation depends on the shape
considered. Using the solar elevation angle $\theta_{sun}$,  the solar beam can be considered to have a vertical
component, impinging on the horizontal surface (semicircular base of the AIR), and a horizontal component impinging on
the vertical cross section (a triangle). The solar elevation angle $\theta_{sun}$ used is modelled using the
parametrisation proposed by \cite{Woolf_1968}. Accordingly, $f_{cone}$ is determined as follows:

\begin{equation} \begin{split} f_{cone}& =\frac{(0.5 \cdot r_{ice} \cdot h_{ice}) \cdot cos \theta_{sun} +(\pi \cdot
			{r_{ice}}^2/2) \cdot sin \theta_{sun} }{\pi \cdot r_{ice} \cdot ({r_{ice}}^2+{h_{ice}}^2)^{1/2}}\\ \end{split}
	\label{eqn:f_{cone}} \end{equation}

The diffuse shortwave radiation is assumed to impact the conical AIR surface uniformly.

\subsubsection{Net Longwave Radiation \texorpdfstring{$q_{LW}$}{Lg}}

The net longwave radiation $q_{LW}$ is determined as follows:

\begin{equation} q_{LW}= LW_{in}-\sigma \cdot \epsilon_{ice} \cdot {(T_{ice}+ 273.15)}^4
	\label{eqn:LW} \end{equation}

where $T_{ice}$ is the modelled surface temperature, both temperatures are given in [$\degree C$], $\sigma=5.67\cdot
	10^{-8}\,Jm^{-2}s^{-1}K^{-4}$ is the Stefan-Boltzmann constant, $LW_{in}$ denotes the incoming longwave radiation and
$\epsilon_{ice}$ is the corresponding emissivity value for the Icestupa surface (0.95).

The incoming longwave radiation $LW_{in}$, for which there were no direct measurements available at the Indian site, is determined
as follows:

\begin{equation} LW_{in}=\sigma \cdot (\epsilon_a \cdot {(T_a+ 273.15)}^4)
	\label{eqn:LWin} \end{equation}

here  $T_a$ represents the measured air temperature and $\epsilon_a$ denotes the atmospheric emissivity. We approximate atmospheric emissivity $\epsilon_a$ using the
equation suggested by \cite{Brutsaert_1982}, considering air temperature and vapor pressure (Eqn.  \ref{eqn:atm_e}). The
vapor pressures over air and ice was obtained using Eqn. \ref{eqn:vp}.  The expression defined in \cite{Brutsaert_1975}
for clear skies (first term in equation \ref{eqn:atm_e}) is extended with the correction for cloudy skies after
\cite{Brutsaert_1982} as follows:

\begin{equation} \epsilon_a=1.24 \cdot (\frac{p_{v,a}}{(T_a+273.15)})^{1/7}\cdot(1+0.22\cdot{c}^2) \label{eqn:atm_e}
\end{equation}

with a cloudiness index $c$, ranging from 0 for clear skies to 1 for complete overcast skies. For the Indian site, we
assume cloudiness to be negligible.

\subsubsection{Turbulent fluxes }

The turbulent sensible $q_{S}$ and latent heat $q_{L}$ fluxes are computed with the following expressions proposed by
\cite{Garratt_1992}:

\begin{equation} q_{S}=\mu_{cone}\cdot c_{a} \cdot \rho_{a} \cdot p_{a}/p_{0,a} \cdot \frac{\kappa^2 \cdot v_a \cdot
		(T_a-T_{ice})}{{(\ln{\frac{h_{AWS}}{z_{0}}})}^2} \label{eqn:qs} \end{equation}

\begin{equation} q_{L}=\mu_{cone}\cdot 0.623 \cdot L_s \cdot \rho_{a}/p_{0,a} \cdot \frac{\kappa^2 \cdot
	v_a(p_{v,a}-p_{v,ice})}{{(\ln{\frac{h_{AWS}}{z_{0}}})}^2} \end{equation}

where $h_{AWS}$ is the measurement height above the ground surface of the AWS (around $2\,m$ for all sites), $v_a$ is
the wind speed in [$m\,s^{-1}$], $c_a$ is the specific heat of air at constant pressure (1010 J $kg^{-1} K^{-1}$),
$\rho_{a}$ is the air density at standard sea level (1.29 $kg m^{-3}$), $p_{0,a}$ is the air pressure at standard sea
level (1013 $hPa$), $\kappa$ is the von Karman constant (0.4) and $L_s$ is the heat of sublimation (2848 $kJ\,
	kg^{-1}$).  The vapor pressures over air ($p_{v,a}$) and ice ($p_{v,ice}$) was obtained using the following formulation
given in \cite{WMO_2018}:

\begin{equation} \begin{split} p_{v,a}&=6.107 \cdot 10^{(7.5 \cdot T_a / (T_a + 237.3))}\\ p_{v,ice}&=(1.0016 +
		3.15\cdot10^{-6}\cdot p_{a}-0.074\cdot p_{a}^{-1})\cdot(6.112 \cdot e^{(22.46 \cdot T_{ice} / (T_{ice} + 272.62))})
	\end{split} \label{eqn:vp} \end{equation}

where $p_{a}$ is the measured air pressure in [$hPa$].

The dimensionless parameter $\mu_{cone}$ is an "exposure/roughness parameter" that deals with the fact that AIR has a
rough appearance and forms an obstacle to the wind regime. As described in \cite{Oerlemans_2021}, this factor accounts
for the larger turbulent fluxes due to the roughness of the surface and is a function of the AIR slope as follows:

\begin{equation}
	\mu_{cone} = 1 + \frac{s_{cone}}{2}
\end{equation}

The surface roughness, $z_{0}$, may need more attention. Literature values for surface roughness rarely exceed 1 cm, and
for glacier ice are generally in the range $0.1-5\, mm$ \citet{BrockWillisSharp_2006}. We decided to use a commonly used
literature value for $z_{0}$ as an initial value ($1.7\,mm$), as proposed by \cite{CuffeyPaterson_2010} which is close to
the ideal value found in \cite{reid_brock_2014} and within the range presented by \cite{BrockWillisSharp_2006}.

A possible source of error is the fact that wind measurements from the horizontal plane at the AWS are used, which might
be different from those on a slope. However, without detailed datasets from the AIR surface, we retain this
assumption.

\subsubsection{Fountain discharge heat flux \texorpdfstring{$q_{F}$}{Lg} }
The fountain water temperature $T_{water}$ is assumed to cool to 0 $\degree C$. Thus, the heat flux caused by this
process is:

\begin{equation}
	q_{F} = \frac{ \Delta M_F \cdot c_{water} \cdot T_{water}}{\Delta t \cdot A}
	\label{eqn:qF}
\end{equation}
with $c_{water}$ as the specific heat of water.

\subsubsection{Bulk Icestupa heat flux \texorpdfstring{$q_{G}$}{Lg}} \label{sec:Bulkflux}
The bulk Icestupa heat flux $q_{G}$ corresponds to the ground heat flux in normal soils and is caused by the temperature
gradient between the surface layer ($T_{ice}$) and the ice body ($T_{bulk}$). It is expressed by using the heat
conduction equation as follows:

\begin{equation} q_{G} = k_{ice} \cdot (T_{bulk}-T_{ice})/l_{ice} \label{eqn:qG}    \end{equation}

where $k_{ice}$ is the thermal conductivity of ice (2.123 $W\, m^{-1}\,K^{-1}$) , $T_{bulk}$ is the mean temperature of
the ice body within the Icestupa and $l_{ice}$ is the average distance of any point in the surface to any other point in
the ice body. $T_{bulk}$ is initialised as 0 $\degree C$ and later determined from Eqn. \ref{eqn:qG} as follows:

\begin{equation} T_{bulk}^{i+1} = T_{bulk} - (q_{G} \cdot A \cdot \Delta t)/(M_{ice} \cdot c_{ice}) \end{equation}

Since AIR's typically have conical shapes with $r_{ice} >> h_{ice}$, we assume that the center of mass of the ice body
is near the base of the fountain. Thus, the distance of every point in the AIR surface layer from the ice body's center
of mass is between $h_{ice}$ and $r_{ice}$. So we calculate $q_{G}$ here assuming $l_{ice} = (r_{ice} + h_{ice})/2$.

\subsection{Surface temperature}
The available energy $q_{surf}$ can act on the surface of the AIR to a) change its temperature, b) melt ice or
c) freeze ice. So Eqn. \ref{eqn:EB} can be rewritten as: \begin{equation} q_{surf} = q_{freeze/melt} +
	q_{T} \end{equation}
where $q_{T}$, $q_{freeze}$ and $q_{melt}$ represent energy associated with process (a), (b) and (c) respectively.

To distribute the surface energy flux into these three components, we categorize the model time steps as freezing or
melting events. Freezing events can only occur if there is fountain water available and the surface energy flux is
negative. But just these two conditions are not sufficient as the latent heat energy can only contribute to temperature
fluctuations. So to prevent latent heat energy from turning a melting event into a freezing event an additional
condition namely $(q_{surf}-q_{L}) < 0$ is required. Thus, freezing and melting events are identified as follows:

\begin{equation}
	q_{freeze/melt} = \left\{ \begin{array}{ll}
		q_{freeze} & \textit{ if } \Delta M_{F} > 0 \textit{ and } q_{surf} < 0 \textit{ and }(q_{surf}-q_{L}) < 0 \\
		q_{melt}   & \textit{ otherwise}
	\end{array} \right.
\end{equation}

During a freezing event, the AIR surface is assumed to warm to $0 \degree C$. So the available energy $(q_{surf}-q_{L})$
is further augmented due to this change in surface temperature represented by the energy flux $q_{0} = \frac{\rho_{ice}
		\cdot \Delta x \cdot c_{ice} \cdot T_{ice}^{i-1}}{\Delta t}$. Also, the available energy can either be sufficient or
insufficient to freeze the fountain water available.  If insufficient, the additional energy further cools down the
surface temperature. So the surface energy flux distribution during a freezing event can be represented as:

\begin{equation}
	(q_{freeze}, q_{T}) = \left\{ \begin{array}{ll}
		(q_{surf}-q_{L}+q_{0}, q_{L}-q_{0}) & \textit{ if } \Delta M_{F} \geq -\frac{(q_{surf}-q_{L}+q_{0}) \cdot A \cdot \Delta
		t}{L_f}                                                                                                                  \\
		(\frac{\Delta M_{F} \cdot L_f
		}{A \cdot \Delta t}
		, q_{surf}+\frac{\Delta M_{F} \cdot L_f
		}{A \cdot \Delta t})                & \textit{ if } \Delta M_{F} < -\frac{(q_{surf}-q_{L}+q_0) A \cdot \Delta
		t}{L_f}
	\end{array} \right.
\end{equation}

During a melting event, the surface energy flux ($q_{surf}$) is first used to change the surface temperature to
$T_{temp}$ calculated as:

\begin{equation} T_{temp} =\frac{q_{surf} \cdot \Delta t}{\rho_{ice} \cdot c_{ice} \cdot \Delta x} + T_{ice} \end{equation}

If $T_{temp} > 0 \degree C$, then energy is reallocated from $q_{T}$ to $q_{melt}$ to maintain surface temperature at
melting point. So the surface energy flux distribution during a melting event can be represented as:

\begin{equation}
	(q_{melt}, q_{T}) = \left\{ \begin{array}{ll}
		(0, q_{surf})                                                                                                                                                 & \textit{ if } T_{temp} < 0 \\
		(\frac{T_{temp} \cdot \rho_{ice} \cdot c_{ice} \cdot \Delta x}{\Delta t}, q_{surf}-\frac{T_{temp} \cdot \rho_{ice} \cdot c_{ice} \cdot \Delta x}{\Delta t}  ) & \textit{ if } T_{temp} > 0
	\end{array} \right.
\end{equation}


\subsection{Mass Balance}
The mass balance equation for an AIR is represented as:

\begin{equation}
	\frac{\Delta M_{F} + \Delta M_{ppt} + \Delta M_{dep}}{\Delta t} = \frac{\Delta M_{ice} +\Delta M_{water} +
		\Delta M_{sub} + \Delta M_{runoff}}{\Delta t}  \\
	\label{eq:MB}
\end{equation}

where $M_{F}$ is the discharge of the fountain; $M_{ppt}$ is the cumulative precipitation;  $M_{dep}$ is the cumulative
accumulation through water vapour deposition; $M_{ice}$ is the cumulative mass of ice; $M_{water}$ is the cumulative
mass of melt water; $M_{sub}$ represents the cumulative water vapor loss by sublimation and $M_{runoff}$ represents the
fountain discharge runoff that did not interact with the AIR. The LHS of equation \ref{eq:MB} represents the rate of
mass input and the RHS represents the rate of mass output for an AIR.

Precipitation input is calculated as shown in equation \ref{eq:ppt} where $\rho_{w}$ is the density of water (1000
$kg\,m^{-3}$), $ppt$ is the measured precipitation rate in [$m\,s^{-1}$] and $T_{ppt}$ is the temperature threshold
below which precipitation falls as snow. Here, snowfall events were identified using $T_{ppt}$ as $1 \degree C$. Snow
mass input is calculated by assuming a uniform deposition over the entire circular footprint of the AIR.

The latent heat flux is used to estimate either the evaporation and condensation processes or sublimation and deposition
processes as shown in equation \ref{eq:vap}. During time steps at which surface temperature is below 0 $\degree C$ only
sublimation and deposition can occur, but if the surface temperature reaches 0 $\degree C$, evaporation and condensation
can also occur. As the differentiation between evaporation and sublimation (and condensation and deposition) when the
air temperature reaches 0 $\degree C$ is challenging, we assume that negative (positive) latent heat fluxes correspond
only to sublimation (deposition), i.e. no evaporation (condensation) is calculated.

Since we have categorized every time step as a freezing and melting event, we can determine the meltwater and  ice
generated using the associated energy fluxes as shown in equations \ref{eq:mwat} and \ref{eq:mice}. Having
calculated all the other mass components the fountain wastewater generated every time step can be calculated using
Eqn. \ref{eq:MB}.

\begin{subequations}
	\label{equations}
	\begin{align}
		\label{eq:ppt}
		\frac{\Delta M_{ppt}}{\Delta t}                                    & = \left\{ \begin{array}{ll} \pi \cdot {r_{ice}}^2 \cdot
			\rho_{w}\cdot ppt & \textit{ if } T_{a} < T_{ppt} \\ 0 & \textit{ if } T_{a} \geq T_{ppt} \\\end{array} \right.                                      \\
		\label{eq:vap}
		(\frac{\Delta M_{dep}}{\Delta t}, \frac{\Delta M_{sub}}{\Delta t}) & = \left\{ \begin{array}{ll} \frac{q_{L}
			\cdot A}{L_s}\cdot (1,0)  & \textit{ if } q_{L} \geq 0 \\ \frac{q_{L}
			\cdot A}{L_s}\cdot (0,-1) & \textit{ if } q_{L} < 0    \\\end{array} \right.                                      \\
		\label{eq:mwat}
		\frac{\Delta M_{water}}{\Delta t}                                  & = \frac{q_{melt} \cdot A }{L_f}                                                   \\
		\label{eq:mice}
		\frac{\Delta M_{ice}}{\Delta t}                                    & = \frac{q_{freeze}\cdot A }{L_f} + \frac{\Delta M_{ppt}}{\Delta t} + \frac{\Delta
			M_{dep}}{\Delta t}- \frac{\Delta M_{sub}}{\Delta t}- \frac{\Delta M_{melt}}{\Delta t}
	\end{align}
\end{subequations}

We define the freezing rate $m_{freeze}$ and the melting rate $m_{melt}$ as follows:

\begin{equation}
	m_{freeze/melt} = \frac{q_{freeze/melt} \cdot A }{L_f}
\end{equation}

To estimate the mass of any component at time step $i$, one can now sum the mass flux estimated above:
\begin{equation} M_{comp}^i = \sum_{t=0}^{t=i} (\frac{\Delta M_{comp}}{\Delta t})_{t} + M_{comp}^0 \end{equation}
where

\begin{equation} M_{comp}^0 = \left\{ \begin{array}{ll} -V_{dome} * \rho_{ice} & \textit{ if } M_{comp}=
             M_{ice}\textit{ or }
             M_{F}                                                 \\ 0 & \textit{ otherwise }\\\end{array} \right. \\
\end{equation}

Considering AIRs as water reservoirs, their storage efficiency ($SE$) can be defined as the percentage of ice and
meltwater produced as follows:

\begin{equation} \textit{SE} = \frac{M_{water}+M_{ice}}{(M_F+M_{ppt}+M_{dep})} \cdot 100 \end{equation}

In the following analysis, $SE$ will be used to compare the three different AIR.

\subsection{Sensitivity and uncertainty analysis}

\begin{table}
	\centering
	\caption{The ranges of the 10 different parameters used in the sensitivvity study.}
	\label{tab:parameters}
	\begin{tabular}{@{}llllll@{}}
		\toprule
		\textbf{No.} & \textbf{Name}                       & \textbf{Abbreviation} & \textbf{Min}        & \textbf{Max}        & \textbf{Unit} \\\midrule
		1            & Ice Emissivity                      & $\epsilon_{ice}$      & 0.95                & 0.99                &               \\
		2            & Ice Albedo                          & $\alpha_{ice}$        & 0.15                & 0.35                &               \\
		3            & Snow Albedo                         & $\alpha_{snow}$       & 0.8                 & 0.9                 &               \\
		4            & Precipitation Temperature threshold & $T_{ppt}$             & 0                   & 2                   & $\degree C$   \\
		5            & Albedo Decay Rate                   & $\tau$                & 5                   & 30                  & $days$        \\
		6            & Surface Roughness                   & $z_0$                 & $1 \times 10^{-3}$  & $5 \times 10^{-3}$  & $m$           \\
		7            & Surface layer thickness             & $\Delta x$            & $16 \times 10^{-3}$ & $24 \times 10^{-3}$ & $m$           \\
		8            & Fountain water temperature          & $T_{F}$               & 0                   & 5                   & $\degree C$   \\
		9            & Fountain mean discharge rate        & $d_{F}$               & 1                   & 60                  & $l/min$       \\
		10           & Fountain mean spray radius          & $r_{F}$               & 6                   & 11                  & $m$           \\\bottomrule
	\end{tabular}
\end{table}

The sensitivity analysis used here followed the two step method used by \cite{ZollesMaussion_2019}.  The only difference
is we use a polynomial chaos expansion approach (as in \cite{uncertainpy_2018}; \cite{Xiu_2005}).  Polynomial chaos
expansion are a much more efficient way to obtain similar results compared to the computationally demanding Monte Carlo
methods. This approach approximates the model with a polynomial (as a surrogate model), on which sensitivity and
uncertainty analysis can be performed.

First a Global sensitivity analysis (GSA) was performed with the storage efficiency as the objective on the 10 free
parameters .  The parameter sensitivity results from the GSA are used as a tool to reduce the number of free
parameters in the model by identifying those parameters which have only a marginal influence on the model output. The
model is considered insensitive to parameters with a total sensitivity index ($S_{T_{i}}$) of < 0.05, and these
parameters were fixed at the median value of the range shown in in Table \ref{tab:parameters} in subsequent model
simulations.

The ranges for the ice albedo, snow albedo and albedo decay rate were taken from \cite{ZollesMaussion_2019}; emissivity
value from \cite{steiner_2015}; temperature threshold for precipitation from \cite{Zhou_2010} and surface roughness from
\cite{BrockWillisSharp_2006}.

In a second step, we evaluated model runs varying only the significant parameters within their ranges. These model runs
reflect a global sensitivity of the model to the parameters used in the optimization, since all significant parameters
were changed at the same time.  The best 10 \% of the calibration runs (i.e. resulting in the lowest RMSE) were selected
and the frequency distribution of the parameters from these runs plotted for each AIR, to evaluate their spread.
Parameters with a large spread are those that the model is not sensitive to, as any of those values lead to a high model
performance. On the contrary, parameters with a small spread are those that the model is sensitive to. The initial
parameter choice is replaced by the optimized value for sensitive parameters whereas the initial choice is preserved for
the non sensitive parameters.

To perform the uncertainty analysis on the ice volume estimation, a polynomial chaos expansion approach (as in
\cite{uncertainpy_2018}; \cite{Xiu_2005}) was used. Polynomial chaos expansion are a much more efficient way to obtain
similar results compared to the computationally demanding Monte Carlo methods. This approach approximates the model with
a polynomial (as a surrogate model), on which sensitivity and uncertainty analysis can be performed.

\section{Results}

\subsection{Mass and Energy fluxes of the AIR}

Fig. \ref{fig:MEB} shows the daily averages of the mass and energy fluxes calculated with initial parameters for the
first and last 20 days for each AIR. The two time periods selected are characteristic of the freezing and melting period
respectively. A strong variability is evident between the freezing and melting periods, between the different years for
the Swiss AIR and between the Swiss and the Indian AIR.

\begin{figure}
	\begin{center}
		\includegraphics[width=\linewidth]{Figures/albedo.jpg}
	\end{center}
	\caption{Some derived parameters of the model, namely, albedo and $f_{cone}$ (a), Surface temperature (b). In
		(a), the purple curve shows how snow and fountain-on events reset albedo between ice albedo and snow albedo.  The
		decay of the snow albedo to ice albedo can also be observed. The orange curve shows how the solar radiation area
		fraction varied diurnally and seasonally with variations in the solar elevation angle. In (b), the surface
		temperature (blue curve) was forced to be 0 $\degree C$ during fountain duration.}
	\label{fig:albedo}
\end{figure}

The variation of the mass fluxes during the freezing period of the IN21 AIR reflect the variation in discharge quantity
more than that of the energy flux. On day 1,12,13,14 and 20 the fountain discharge was interrupted due to pipeline
freezing events. So during these days, the freezing energy was instead used up to further reduce the AIR surface
temperature as can be seen in Fig. \ref{fig:albedo}. For the Swiss AIR though, the mass flux variation is only due to
the variation in the energy flux as there were no such discharge interruptions. During fountain duration, the magnitude
of the different mass fluxes are insensitive to the discharge quantity since for all the AIR the freezing rate was more
than half that of the mean discharge quantity. However, the sensitivity of the melting energy increases with increasing
water temperature. This effect will be analysed later in the sensitivity analysis.

\begin{figure}
	\begin{center}
		\includegraphics[width=\linewidth]{Figures/mass_energy_bal.jpg} \end{center}
	\caption{Daily averages of mass and energy fluxes compared for all AIR during their freezing and melting periods.
		$q_{freeze}$ is the total freezing energy;$q_{melt}$ is the total melting energy;$q_{SW}$ is the net shortwave
		radiation; $q_{LW}$ is the net longwave radiation; $q_{L}$ and $q_{S}$ are the turbulent latent and sensible heat
		fluxes. $q_{F}$ represents energy gained by the AIR surface layer due to fountain water temperature.  $q_{G}$
		quantifies the heat conduction process between the AIR surface layer and the ice body. } \label{fig:MEB}
\end{figure}

The magnitude of the mass fluxes in Fig. \ref{fig:MEB} are explained by their energy flux counterparts. Namely ice mass
flux corresponds to freezing energy available, melt mass flux corresponds to melting energy available, snowfall is
calculated directly form the precipitation quantity and ice radius and sublimation/deposition quantities correspond to
the latent heat flux available. The rest off the energy fluxes are shown to represent the different physical processes
that contribute to this freezing and melting energy.

Longwave radiation is a major source of energy for all the AIR. During the freezing period, it is the largest
contributor to the daily ice mass for all the AIR. During the melting period though it compensates the shortwave
radiation to a large degree.

Shortwave radiation varies both in magnitude and distribution between the different locations, since the indian AIR is
much higher in altitude and has mostly clear days resulting in much higher shortwave radiation with a negligible
diffuse component.  Shortwave radiation variability is controlled by the surface albedo $\alpha$ and the area
fraction $f_{cone}$ which therefore represent key variables in the energy balance. As shown in Fig. \ref{fig:albedo},
less than half of the AIR surface area was exposed to direct shortwave radiation flux for all the AIR. Also one can
observe the diurnal and seasonal variations in this direct shortwave radiation due to diurnal and seasonal variation in
the solar elevation angle.  Albedo, on the other hand, only varied after the fountain duration and if there was any
snowfall. Since we assume there was negligible precipitation in the Indian site, albedo is constant there.

Turbulent fluxes vary significantly in magnitude and distribution between the two time periods and the two locations.
In the freezing and the melting periods, sensible heat contributes to warming or melting the AIR surface. This is not
true for the Indian AIR though where certain days(example day 7) were colder than the AIR surface temperature resulting
in a negative sensible heat flux. In general, sensible heat was never able to melt ice for the Indian AIR. For the Swiss
AIR though, there exist days (example day 20) where the AIR was melting in the day primarily due to sensible heat and
freezing in the night.

Latent heat plays a fascinating role in this whole energy balance. It contributes to both mass and energy flux
simultaneously through deposition/sublimation processes. Since the heat of vaporization is large,
sublimation(deposition) process can contribute significantly to the freezing(melting) energy without affecting the
overall mass flux much. This can be observed throughout the life cycle of the Indian AIR where sublimation either
contributed to more ice mass or compensated the sensible heat flux during the freezing and melting period respectively.
Sublimation was also significantly greater in the Indian site compared to the Swiss site as the corresponding latent heat
fluxes were an order of magnitude larger (see Table \ref{tab:Observations}). This was primarily due to the difference in
relative humidity between the sites. This difference is one of the major reasons why the corresponding freezing rate
for the Indian AIR was more than 2 times the melting rate.

\begin{table}
	\centering
	\caption{ Summary of the mass balance, energy balance and AIR charectiristics estimated by the model}
	\label{tab:Results}
	\begin{tabular}{@{}|llllll|@{}}
		\toprule
		\textbf{}              & \textbf{Name}           & \textbf{Symbol}           & \textbf{IN21} & \textbf{CH21} & \textbf{CH20} \\ \midrule
		\multicolumn{1}{|l|}{\multirow{3}{*}{\rotatebox[origin=c]{90}{Input}}}
		                       & Fountain discharge      & $M_F$                     & 2911 $tons$   & 971 $tons$    & 700 $tons$    \\
		\multicolumn{1}{|l|}{} & Snowfall                & $M_{ppt}$                 & 0 $tons$      & 56 $tons$     & 22 $tons$     \\
		\multicolumn{1}{|l|}{} & Deposition              & $M_{dep}$                 & 6 $tons$      & 4 $tons$      & 2 $tons$      \\ \midrule
		\multicolumn{1}{|l|}{\multirow{4}{*}{\rotatebox[origin=c]{90}{Output}}}
		                       & Meltwater               & $M_{water}$               & 240 $tons$    & 230 $tons$    & 149 $tons$    \\
		\multicolumn{1}{|l|}{} & Ice                     & $M_{ice}$                 & 219 $tons$    & 0 $tons$      & 0 $tons$      \\
		\multicolumn{1}{|l|}{} & Sublimation             & $M_{sub}$                 & 48 $tons$     & 5 $tons$      & 5 $tons$      \\
		\multicolumn{1}{|l|}{} & Fountain runoff         & $M_{runoff}$              & 2483 $tons$   & 796 $tons$    & 570 $tons$    \\ \midrule
		\multicolumn{1}{|l|}{\multirow{8}{*}{\rotatebox[origin=c]{90}{Energy flux}}}
		                       & Shortwave radiation     & $q_{SW} [W\,m^{-2}] $     & $ 35 \pm 63$  & $ 41 \pm 65$  & $ 44 \pm 66$  \\
		\multicolumn{1}{|l|}{} & Longwave radiation      & $q_{LW} [W\,m^{-2}] $     & $-79 \pm 29$  & $-59 \pm 32$  & $-63 \pm 31$  \\
		\multicolumn{1}{|l|}{} & Sensible heat           & $q_{S} [W\,m^{-2}]  $     & $61 \pm 81$   & $40 \pm 89$   & $37 \pm 72$   \\
		\multicolumn{1}{|l|}{} & Latent heat             & $q_{L} [W\,m^{-2}]  $     & $-36 \pm 49$  & $-3 \pm 37$   & $-9 \pm 35$   \\
		\multicolumn{1}{|l|}{} & Fountain heat           & $q_{F} [W\,m^{-2}]  $     & $1 \pm 2$     & $0 \pm 0$     & $1 \pm 0$     \\
		\multicolumn{1}{|l|}{} & Shortwave radiation     & $q_{G} [W\,m^{-2}]  $     & $0 \pm 1$     & $0 \pm 1$     & $0 \pm 1$     \\
		\multicolumn{1}{|l|}{} & Freezing energy         & $q_{freeze} [W\,m^{-2}] $ & $-161\pm 46$  & $-73 \pm 48$  & $-75\pm 39$   \\
		\multicolumn{1}{|l|}{} & Melting energy          & $q_{melt} [W\,m^{-2}] $   & $88 \pm 77$   & $114\pm 134$  & $93 \pm 92$   \\
		\multicolumn{1}{|l|}{} & Temperature             & $q_{T} [W\,m^{-2}] $      & $0 \pm 87$    & $0 \pm 37$    & $0 \pm 23$    \\\midrule
		\multicolumn{1}{|l|}{\multirow{3}{*}{\rotatebox[origin=c]{90}{Charecteristics}}}
		                       & Storage Efficiency      & SE                        & 15 \%         & 22 \%         & 20 \%         \\
		\multicolumn{1}{|l|}{} & Root mean squared error & RMSE                      & 84 $m^{3}$    & 10 $m^{3}$    & 21 $m^{3}$    \\
		\multicolumn{1}{|l|}{} & Mean freezing rate      & $(m_{freeze})_{mean}$     & 11 $l/min$    & 1.9 $l/min$   & 2.4 $l/min$   \\
		\multicolumn{1}{|l|}{} & Mean melting rate       & $(m_{melt})_{mean}$       & 5 $l/min$     & 2.2 $l/min$   & 2.5 $l/min$   \\\bottomrule
	\end{tabular}
\end{table}

The magnitude of the freezing and melting rates is controlled by the fountain attributes, particularly the spray radius.
Even though the Indian AIR's fountain duration was roughly half that of the Swiss, its spray radius was significantly
larger. This resulted in higher surface area to capitalise on the freezing energy. Moreover, since the freezing energy
was larger in magnitude than the melting energy, the surface area favored freezing over melting. However, in the Swiss
AIR both these rates are almost the same, indicating that the surface area did not favor the melting or the freezing process.

Thus the fountain attributes and energy flux combine to produce ice volumes 4 times higher for the Indian compared to the Swiss AIR, even though the fountain duration of the Indian AIR was just around half that of the Swiss AIR as can be seen in Table \ref{tab:Results}.

% \subsection{Sensitivity and uncertainty analysis}
\subsection{Sensitivity analysis}
\subsubsection{Significant parameters}



Fountain water temperature was found to be the most significant parameter causing a standard deviation of $3\%$ in the
storage efficiency in all the AIR. Since warmer fountain water provides more melting energy to the AIR surface during the fountain
duration, this parameter influences the storage efficiency negatively.

Surface roughness variance was significant for all the AIR but varied in magnitude between the Swiss and Indian
locations. It caused a standard deviation of $2\%$ and $1\%$ in the storage efficiency for the Indian and Swiss AIR
respectively.

Ice emissivity variance was significant for all the AIR but varied in magnitude between the Swiss and Indian
locations. The higher the ice emissivity the larger the maximum ice volume as the emitted longwave radiation increases
with ice emissivity.  It caused a standard deviation of $0.6\%$ and $2\%$ in the storage efficiency for the Indian and
Swiss AIR respectively.

Ice albedo variance was significant for all the AIR but varied in magnitude between the Swiss and Indian locations.
It caused a standard deviation of $0.3\%$ and $1\%$ in the storage efficiency for the Indian and Swiss AIR respectively.

The rest of the uncertain parameters were not significant since they caused a variation of less than $1\%$ in the
storage efficiency.

In total, the sensitivity analysis required 120 simulations, and the uncertainty analysis a total of 32 simulations.

\begin{figure}
	\begin{center}
		\includegraphics[width=10 cm]{Figures/sensitivities.jpg}
	\end{center}
	\caption{Sensitivities of maximum ice volume to all the uncertain parameters used in the model (Table
		\ref{tab:parameters}). } \label{fig:sensitivity} \end{figure}

\subsubsection{Optimal parameters}
In total, 1800 model runs were performed for each AIR consisting of combinations of the parameters that were varied with a range
derived from literature values and a discrete step as shown in Table \ref{tab:parameters}. Fig. \ref{fig:param_hist}
shows the frequency distribution of the calibration parameters among the best 10 \% of the model runs scored based on RMSE.

\begin{figure}
	\begin{center}
		\includegraphics[width=\linewidth]{Figures/param_hist.jpg}
	\end{center}
	\caption{Sensitivity of the model to the significant parameters used in the model optimization, shown by plotting the frequency
		distribution of the parameter values for the best 10 \% of the model runs. } \label{fig:param_hist} \end{figure}

% \section{Discussion} 
% \subsection{Storage efficiency}
% The storage efficiency of IN21, CH21 and CH20 AIR were $15\%$, $22\%$ and $20\%$ respectively. The low SE is a product
% of the high  fountain water losses ( $\frac{M_{runoff}}{M_{input}}> 75 \%$ ) in all the AIR. This indicates that just
% decreasing the mean fountain discharge could have significantly increased water storage efficiency. CH21, IN21 and CH20
% had a maximum fountain freezing rate of just 7 $l/min$, 18 $l/min$ and 6 $l/min$ respectively compared to the 7.5
% $l/min$, 60 $l/min$ and 7.5 $l/min$ of mean fountain discharge they were provided with.

% \subsection{Sublimation}
% We suggest that glaciers in dry regions lose a significant amount of mass through sublimation, while
% condensation/re-sublimation is dominant over glaciers more directly influenced by monsoons. Thus, sublimation should be
% included in hydrological modeling at least over dry regions, such as the northwest Himalaya and Karakoram, especially on
% the cold, dry Tibetan side. \cite{azam_2018}

\begin{figure}
	\begin{center}
		\includegraphics[width=\linewidth]{Figures/icev_results.jpg}
	\end{center}
	\caption{Modelled ice volume during the lifetime of the AIR (blue curve). Green points indicate the validation
		measurements. The prediction interval is based on the ice volume uncertainty caused by the significant parameters.  The
		upper and lower bounds of the y axis represents maximum ice volume and dome volume respectively.  }
	\label{fig:results} \end{figure}

\section{Conclusions}
In this paper, we have developed a bulk energy and mass balance model to simulate AIR evolution using data from field measurements in the Indian
Himalayas and the Swiss Alps. The use of this dataset, in combination with novel algorithms for calculation of the surface
area and energy fluxes, allowed an accurate representation of the complex growth dynamics typical of any AIR. The model
was calibrated with ice volume observations obtained via drone flights. We calculated the storage
efficiency for each of the three AIRs and used calculations of the energy fluxes and fountain attributes to explain the
observed variability in freezing and melting rates.

In order to properly understand the role of the different physical parameters involved in our model, we performed a sensitivity analysis based on the resulting SE. The results of this analysis suggest that AIRs grow higher and last longer in drier regions (e.g. Ladakh, Indian Himalayas). Since sublimation augments freezing rates and dampens melting
rates, regions with lower relative humidity are more conducive for AIR.

Storage efficiency of all the AIR are poor but can be optimized significantly through control of the fountain discharge
rate.  The model results indicate that more than 80\% of the fountain water is lost as runoff for all the AIRs.
Further experiments at different locations with different fountains are required to better understand the influence of
the fountain discharge rate on the results.

% \section{Appendix}

% \subsection{AIR discharge quantity and duration} \label{section:discharge} 

% \subsection{Ladakh Icestupa 2014/15} \label{section:ladakhloss} 
% A 20 $m$ tall Icestupa \citep{iceheight} was built in Phyang village, Ladakh at an altitude of 3500 $m$ a.s.l. Assuming
% a conical shape with a diameter of 20 $m$, the corresponding volume of this Icestupa becomes 2093 $m^3$ or 1,920 $m^3$
% w.e. The fountain sprayed water at a rate of $210\, l\,min^{-1}$ \citep{waterinput} from $21^{st}$ January
% \citep{waterstart} to at least until $5^{th}$ March 2015 \citep{waterend} (around 43 nights). Assuming fountain spray
% was active for 8 hours each night, we estimate water consumption to be around 4,334 $m^3$. Thus, during the
% construction/freezing period of the Icestupa, roughly 56 \% of the water provided was wasted. The actual water loss is
% bound to be much higher due to further vapour losses during the melting period. This Icestupa completely melted away on
% $6^{th}$ July 2015 \citep{iceends}. Therefore, the storage duration was 166 days or roughly 5 months. 

% \subsection{CH19 AIR}\label{section:CH19}
% The CH19 AIR in the Schwarzsee region lies at 967 $m$ a.s.l.. In the winter (Oct-Apr), mean daily maximum
% and minimum air temperatures vary between -4 and 14 $\degree C$. Clear skies are rare, averaging around 7 days, and
% precipitation amounts average 155 mm per month during winter \citep{eispalast}. The site was situated adjacent to a
% stream resulting in high humidity values across the study period. Within the EP site, an enclosure with a 1.8\,$m$
% radius was constructed for the experiment. An automatic weather station (AWS) was adjacent to the wooden
% boundary as shown in Fig. \ref{fig:site}. The fountain used for spraying water had a nozzle diameter of 5\,$mm$ and a
% height of 1.35\,$m$, and was placed in the centre of the wooden enclosure. The water was transferred from a spring
% water source at 1267 $m$ a.s.l. by pipeline and flowed via a flowmeter and an air escape valve to the nozzle, where it
% was sprinkled with a spray radius of around 1.7\,$m$. The air escape valve was installed to avoid errors in the flow
% measurements due to air bubbles. In addition, a webcam guaranteed a continuous survey of the site during the
% construction of the Icestupa. 
% 
% \begin{figure} 
%     \centering 
%     \includegraphics[width=15cm]{./Figures/Figure_2.jpg}
% \caption{(a) The ice structure during the first validation measurement as seen on the webcam image of
%   $14^{th}$ Feb. (b) The corresponding cross section of the EP ice structure with the field estimates of $r, R,
%   h, H_i, H_f$ used to determine the Icestupa volume is shown on the right.} 
%   \label{fig:CH19site} 
%   \end{figure}
% 
% \subsubsection{Weather data}
% Precipitation data was derived from the Plaffeien AWS \citep{meteoswiss} located 8.8 km away from the measurement site
% at an altitude of 1042 $m$ a.s.l.  We recognised during our data analysis that, except precipitation, all the other
% meteorological variables of the EP site correlated much better with the ERA5 dataset than the nearby Plaffeien AWS
% dataset. The $2\,m$ temperature parameter correlated with air temperature ($r^2 =0.9 $), surface pressure parameter
% correlated with air pressure ($r^2 = 1$) and 10m wind speed parameter (derived from horizontal and vertical components)
% correlated with wind speed ($r^2 =0.6 $) .
% 
% Due to a power failure, all data from the EP AWS was lost from $27^{th}$ February 15:20 2019 to $2^{nd}$ March 15:00
% 2019 (equivalent to around 7\% of the measurement period). During heavy snowfall events, the ultrasonic wind sensor was
% blocked and recorded zero values. ERA5 was used to fill such errors and data gaps .
% 
% The ERA5 grid point chosen (Latitude 46° 38' 24" N, Longitude 7° 14' 24" E) for the EP site was around 9 km away from
% the actual site. Near-surface humidity is not provided directly in ERA5 dataset, but from near-surface ($2\,m$ from the
% surface) temperature ($T_{ERA5}$) and dew point temperature ($Tw_{ERA5}$) the relative humidity ($RH$) at $2\,m$  was
% calculated as: 
% 
% \begin{equation} RH = 100 \cdot
%     \frac{e_{sat}(Tw_{ERA5})}{e_{sat}(T_{ERA5})} \end{equation} 
% 
% where the saturation vapour pressure function $e_{sat}$ is expressed with the Teten's formula \citep{Tetens}:
% \begin{equation} e_{sat}(T)= a_1 \cdot e^{(a_3 \cdot \frac{T}{(T+273.16-a_4)})} \end{equation} with T in $\degree C$ and
% the parameters set for saturation over water ($a_1$ = 611.21 Pa, $a_3$ = 17.502 and $a_4$= 32.19 K) according to
% \cite{Buck_1981}.    
% 
% All the ERA5 variables were therefore fitted with the EP dataset via linear regressions.  With the modified ERA5
% dataset, we were also able to further extend the EP dataset and allow the model to run beyond $18^{th}$ March 2019.
% Precipitation was filled as null values beyond $18^{th}$ March 2019.
% 
% \subsubsection{Fountain spray radius of CH19 AIR} \label{section:sprayCH19} 
% This fountain spray radius is determined by modelling the projectile motion of the water droplets. Using mass
% conservation, the droplet speed $v_F$ can be determined from the spray rate $d_F$ and the diameter $dia_F$ of the nozzle
% as follows:
% 
% \begin{equation} v_F = \frac{d_F}{\pi \cdot dia_F^2/4} \end{equation}
% 
% Afterwards, we assume that the water droplets move with an air friction free projectile motion from the fountain
% nozzle with a height $h_F$ to the ice/ground surface. The resulting spray radius $r_F$ was then determined from the
% projectile motion equation as follows:
% 
% \begin{equation} r_F = \frac{v_F \cdot cos\theta_F (v_F \cdot sin\theta_F + \sqrt{(v_F \cdot sin\theta_F)^{2} + 2
% \cdot g \cdot h_F})}{g} \end{equation}
% 
% where $g = 9.8\, m s^{-2}$ is the acceleration due to gravity and $\theta_F$ = 45 $\degree$ is the angle of launch.
% 
% \subsubsection{CH19 Field Measurements for validation} \label{section:validation} 
% The volume was determined by decomposing the ice structure into a cylinder (length $2R$ and height $h$) and a
% cone (radius $r$ and height $(H_i-h)$) through the following equation: 
% \begin{equation} V = \pi \cdot R^2 \cdot h + 1/3 \cdot \pi \cdot r^2 \cdot (H_i-h) \end{equation}
% 
% Manual measurements were performed at the end of the freezing period on $14^{th}$ February 16:00 2019 (only one more
% fountain run was possible after this date) to estimate $r, R, h, H_i, H_f$ (see Fig. \ref{fig:site} for the different
% geometry components):
% 
% $$ 0.55\leq r\leq 1 m\textit{ ; }1.1\leq R\leq 1.2 m\textit{ ; }0.1\leq h\leq 0.2 m\textit{ ; }0.6\leq
% H_i\leq 0.8 m\textit{ ; }1.3\leq H_f\leq 1.4 m  $$
% 
% The ranges of the variables show the variance of the Icestupa's dimensions across different compass orientations.
% Correspondingly, the volume range estimated for the first validation point was 0.857 $\pm$ 0.186 $m^{3}$ on $14^{th}$
% February 16:00 2019.
% 
% The second validation point corresponds to the end of the melting process on $10^{th}$ March 18:00 2019.  Based on the
% webcam imagery and manual measurement, a thin layer of ice with an observed thickness between 0.01 to 0.06 m could be
% quantified. This results in the volume range for the second validation to be 0.13 $\pm$ 0.09 $m^{3}$ on $11^{th}$ March
% 2019 
% 
% In reality, the EP ice structure was more cylindrical until a height of 0.2\,$m$ and conical afterwards until a
% height of 0.6\,$m$ with a radius of 1.18\,$m$. However, we assume a conical shape of this ice structure in order to
% apply the modelling strategy described below.

% \subsubsection{CH21 and CH20 Surface temperature corrections} \label{section:thermalcam} 
% We discarded some thermal camera temperature measurements due to the following reasons:
% \begin{itemize} 
%     \item Snowfall/fog: Whenever there was snow or thick fog, the thermal image was corrupted. Refer image Jan14 1900 and
% Jan3 800. We used the standard deviation of the pixel temperature to identify these events and remove them from the
% validation dataset. 
% 
% \item Strong sunlight: Usually at noon, especially in end of Feb and March, we observed that ice temperature values were
%     above zero C. We found that again the thermal cam images were corrupted as seen in Mar6 1300. So we removed all
%     temperature values above 0 C.  
% 
% \item Then there were some images that were completely blue and looked corrupted.  We cannot identify the cause here but
%     I filtered them out using the mean of all temperature pixels.
% 
% \end{itemize} 

% 
% The CH19 AIR was constructed by the Eispalast in Schwarzsee, Switzerland. It was contained inside a wooden boundary
% adjacent to a stream. Fountain operation was guided by temperature conditions.  The water spray of the fountain was
% initially adjusted so that most of the water droplets land within the wooden boundary zone. The ice formation was guided
% by adding a metal framework at the ice structure base after the first night of operation.  Several cotton threads were
% tied between the ice structure base and fountain pole for accelerating and further guiding the ice formation process. 

% \section*{Conflict of Interest Statement} The authors declare that the research was conducted in the absence of any
% commercial or financial relationships that could be construed as a potential conflict of interest.
% 
% \section*{Author Contributions} SB wrote the initial version of the manuscript. MH, ML, SW, JO, and FK commented on
% the initial manuscript and helped improve it. SB developed the methodology with inputs from MH. SB performed the
% analysis with support from MH and ML. SB and MH participated in the fieldwork.
% 
% \section*{Funding} This work was supported and funded by the University of Fribourg and by the Swiss Government
% Excellence Scholarship (SB).
% 
% \section*{Acknowledgments} We thank Mr. Adolf Kaeser and Mr. Flavio Catillaz at Eispalast Schwarzsee for their active
% participation in the fieldwork. We would also like to thank Digmesa AG for subsidising their flowmeter used in the
% experiment. We would particularly like to thank the editor Prof. Thomas Schuler and 2 anonymous reviewers who gave us
% important inputs to improve the paper. We also thank Prof. Christian Hauck, Prof. Nanna B. Karlsson and Dr. Andrew
% Tedstone for valuable suggestions that improved the manuscript.
% 
% \section*{Data Availability Statement} The data and code used to produce results and figures will be published at a
% later stage and can, until then, be obtained from the authors upon request.

\bibliographystyle{frontiersinSCNS_ENG_HUMS} \bibliography{references}

\end{document}
