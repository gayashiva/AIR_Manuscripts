\documentclass[utf8]{frontiersSCNS}
\usepackage{gensymb}
\usepackage{url,hyperref,lineno,microtype,subcaption}
\usepackage[onehalfspacing]{setspace}

\usepackage{tabularx}
\linenumbers
\DeclareUnicodeCharacter{0301}{}
\DeclareUnicodeCharacter{2212}{}
\usepackage{wasysym} % provides \DH, \dh, \Thorn, \thorn
% Leave a blank\usepackage{amsmath}
%\DeclareMathOperator{\sign}{sign} line between paragraphs instead of using \\

% \usepackage{csvsimple} % for csv tables
\usepackage{booktabs}
\usepackage{multirow}
\usepackage{siunitx} %for SI units
\usepackage{tabularx}

\def\keyFont{\fontsize{8}{11}\helveticabold }
\def\firstAuthorLast{Balasubramanian {et~al.}} %use et al only if is more than 1 author
\def\Authors{Suryanarayanan Balasubramanian\,$^{1*,3}$, Martin Hoelzle\,$^{1}$, Michael Lehning\,$^{2}$, Jordi
	Bolibar\,$^{4}$, Sonam Wangchuk\,$^{3}$, Johannes Oerlemans\,$^{4}$ and Felix Keller\,$^{5,6}$}
\def\Address{$^{1}$University of Fribourg, Fribourg, Switzerland\\ $^{2}$WSL Institute for Snow and Avalanche
	Research, Davos, Switzerland\\ $^{3}$Himalayan Institute of Alternatives Ladakh, Leh, India\\ $^{4}$Institute
	for Marine and Atmospheric Research, Utrecht University, Utrecht, The Netherlands\\ $^{5}$Academia Engiadina,
	Samedan, Switzerland\\ $^{6}$ETH, Zürich, Switzerland} \def\corrAuthor{Suryanarayanan Balasubramanian}

\def\corrEmail{suryanarayanan.balasubramanian@unifr.ch}



\begin{document}
\onecolumn
\firstpage{1}

\title[Artificial Ice Reservoirs]{Influence of meteorological conditions on artificial ice reservoir (Icestupa) evolution}

\author[\firstAuthorLast ]{\Authors}
\address{}
\correspondance{}

\extraAuth{}

\maketitle


\begin{abstract}

  Mountain communities in Ladakh, India have been using Artificial Ice Reservoirs (AIRs) for additional
  irrigation water supply. However, there is a large variability associated with this water supply due to the
  local weather influences at the location chosen. This study compared the ice volume evolution of three AIRs
  built in Ladakh, India and Guttannen, Switzerland using a surface energy balance model.  Model input consisted
  of meteorological data in conjunction with fountain discharge rate (mass input of an AIR). Model calibration
  and validation were completed using ice volume and surface area measurements taken from several drone surveys.
  The model achieved an RMSE within $11 \%$ of the maximum ice volume with the ice volume observations for all
  the three AIRs. The location in Ladakh, with a maximum ice volume around 6 times larger, was more favourable
  compared to the Guttannen site.  However, the corresponding water losses for all the AIRs were more than three
  quarters of the total fountain discharge due to high fountain water runoff. Drier and colder locations in
  relatively cloud free regions are expected to produce AIRs with a faster freezing rate and slower melting
  rate. This is a promising result for dry mountain regions, where AIR technology could provide a relatively
  economic and sustainable strategy to mitigate climate change induced water stress.

	\tiny
	\keyFont{ \section{Keywords:} icestupa, water storage, climate change adaptation, geoengineering } %All article types: you may provide up to 8 keywords; at least 5 are mandatory.
\end{abstract}

\section{Introduction}

Seasonal snow cover and glaciers are expected to change their water storage capacity due to climate
change with major consequences for downriver water supply \citep{Immerzeel_2020}. The challenges brought about
by these changes are especially important for dry mountain environments such as in Central Asia or the Andes,
which directly rely on the seasonal meltwater for their farming and drinking needs \citep{HoelzleBarandun_2019,
	Apel_2018, Buytaert_2017, Chen_2016, UNGERSHAYESTEH_2013}.

Ladakh, sandwiched between the Himalayan ranges and the Karakoram, is one of such regions experiencing climate
change induced water stress. Glaciers in the Ladakh region are vital in sustaining agricultural activities which
form the basis for regional food security and socioeconomic development \citep{Labbal_2000, Schmidt_2012}.
During a low precipitation year, glaciermelt and snowmelt are the only sources of water supply to the region
\citep{Thayyen_2010}. Some villages in Ladakh, India have already been forced to relocate due to glacial retreat
and the corresponding loss of their main fresh water resources \citep{zanskar}.

\begin{figure}
	\begin{center}
		\includegraphics[width=10 cm]{Figures/Figure_1.jpg}
	\end{center}
	\caption{Icestupa in Ladakh, India on March 2017 was 24 $m$ tall and contained around 3700 $m^3$
		of water. Picture Credits: Lobzang Dadul}
	\label{fig:old_icestupa}
\end{figure}

Around 26 villages in this region have been using artificial ice reservoirs (AIR) to adapt to these changes since
they require very little infrastructure, skills and energy to be constructed in comparison to other water
storage technologies \citep{IPCC_2019,10.1659/MRD-JOURNAL-D-18-00072.1, campaign}. An AIR is a human-made ice
structure typically constructed during the cold winter months and designed to slowly release freshwater during
the warm spring and summer months.  The main purpose of AIRs is irrigation. Therefore, AIRs are designed to
store water in the form of ice as long into the summer as possible. The energy required to construct an AIR is
usually derived from the gravitational head of the source water body. Some are constructed horizontally by
freezing water using a series of checkdams and others are built vertically by spraying water through fountain
systems \citep{Nusser_2018}. The latter are colloquially referred to as Icestupas and are the subject of this
study.

A typical AIR (see Fig. \ref{fig:old_icestupa}) just requires a fountain nozzle mounted on a supply pipeline.
The water source is usually a high altitude lake or glacial stream. Due to the altitude difference between the
pipeline input and fountain output, water ejects from the fountain nozzle as droplets that eventually lose their
energy and accumulate as ice. The fountain is manually activated during winter nights. The fountain nozzle is
raised through addition of metal pipes as and when significant ice accumulates.  Typically, a dome of branches
is constructed around the metal pipes so that such pipe extensions can be done from within this dome.  Threads,
tree branches and fishing nets are used to guide and accelerate the ice formation.

However, to date, no reliable estimates exist about the quantity of meltwater they can provide
\citep{Nusser_2018} . Moreover, preliminary estimates of AIRs in Ladakh indicate that they generate high water
losses during their lifetimes (see Appendix \ref{sec:ladakhloss}). Particularly, during their accumulation
period, AIRs can lose excessive fountain water directly as runoff and during the ablation period, sublimation
losses could be significant.  However, the relative contribution of these processes in the total water loss
remains unknown.

In this paper, we develop a physically-based model of vertical AIRs (or Icestupas) that can estimate their water
losses, freezing and melting rates. Mass and energy balance equations were used to estimate the quantity of
ice, meltwater, sublimation and runoff. Sensitivity and uncertainty analysis were performed to identify the
most sensitive parameters and the variance caused by them. For calibration, we chose two AIR built across the
winter of 2020/21 in India and Switzerland and validated the model on a Swiss AIR built during winter 2019/20. Our model
results provide a first step towards evaluating the potential of this new water storage technology worldwide.

\section{Study Sites and data}

The model requires three kinds of datasets containing weather, fountain and AIR volume measurements to
accurately calibrate, estimate and validate the ice volume of AIRs. Through the winters of 2018/19, 2019/20 and
2020/21 such datasets were acquired for four AIRs each in Switzerland and India. Here, we present the results of
three AIRs, which have a complete dataset. As shown in Table \ref{tab:Observations}, two of them were
constructed in the same Swiss location called Guttannen (referred to with the prefix CH) but during different
winters and the other was constructed at Gangles, India (referred to with the prefix IN).

The Guttannen site (46.66 $\degree$N, 8.29 $\degree$E) in the Bern region lies at 1047 $m$ a.s.l.. In the winter
(Oct-Apr), mean daily minimum and maximum air temperatures vary between -13 and 15 $\degree C$. Clear skies are
rare, averaging around 7 days during winter \citep{guttannen}. The site was situated adjacent to a stream
resulting in high humidity values across the study period as shown in Fig. \ref{fig:2AIR}. AIR were constructed
here by the Guttannen Bewegt Association during the winters of 2019-20 (CH20) and 2020-21 (CH21). Tree branches
were laid covering the fountain pipe to initiate the ice formation process. The fountain height varied between 2
to 5\,$m$ during the construction period. The water was transferred from a spring water source and flowed via a
flowmeter to the nozzle. In addition, a webcam guaranteed a continuous survey of the site during the
construction of the AIR.

\begin{figure}
	\begin{center}
		\includegraphics[width=12 cm]{Figures/Figure_2.jpg}
	\end{center}
	\caption{The Swiss and Indian AIR on January 9 and March 3, 2021 respectively. Picture credits: Daniel Buerki (left)
		and Thinles Norboo (right)}
	\label{fig:2AIR}
\end{figure}

The Gangles site (34.22 $\degree$N, 77.61 $\degree$E) is located around 20 km north of Leh city in the Ladakh
region, lying at 4025 $m$ a.s.l.. The mean annual temperature is $5.6 \, \degree C$, and the thermal range is
characterized by high seasonal variation. During January, the coldest month, the mean temperature drops to $-7.2
\, \degree C$. During August, the warmest month, the mean temperature rises to $17.5 \, \degree C$
\citep{Nusser_2012}. Because of the rain shadow effect of the Himalayan Range, mean annual precipitation in Leh
totals less than 100 mm, and there is high interannual variability. Whereas the average summer rainfall between
July and September reaches 37.5 mm, the average winter precipitation between January and March amounts to 27.3
mm and falls almost entirely as snow .  AIR were constructed here as part of the Icestupa Competition  by the
Himalayan Institute of Alternatives, Ladakh (HIAL). The fountain height of the AIR varied between 5 to 9\,$m$.

\subsection{Meteorological data}

Air temperature, relative humidity, wind speed, pressure, longwave, shortwave direct and diffuse radiation are
required to calculate the surface energy balance of an AIR (see Table \ref{tab:Observations}).

For the CH site, the primary weather data source was a Meteoswiss AWS located 184 m away . In addition, we used ERA5 reanalysis dataset \citep{era5} for filling data gaps and
adding the shortwave and longwave radiation data that were not measured directly.  The ERA5 reanalysis dataset
has a good correlation with sites in Switzerland \citep{Scherrer_2020}. The ERA5 grid point
chosen (46.64 $\degree$N, 8.25 $\degree$E) for the Swiss site was around 3.6 km away from the actual site. ERA5
variables (except incoming shortwave and longwave radiation) were fitted with the meteoswiss dataset via linear
regressions. The zero wind speed values recorded by the Meteoswiss AWS whenever snow accumulated on the
ultrasonic wind sensor were replaced using the ERA5 dataset.

For the IN site, two different weather data sources were used to log all the weather parameters required for the
model. Temperature, humidity, wind speed and pressure data was logged via a campbell weather station located 440
m away from the construction site. Shortwave radiation data was derived from another campbell weather station
located 15 km away. Unfortunately, precipitation was not logged. Since winter precipitation in Ladakh is less
than 30 mm \citep{Nusser_2012}, we can safely assume negligible precipitation and mostly clear skies. As a
consequence, the diffuse fraction of the global shortwave radiation was also assumed to be negligible .

\begin{table}
	\centering
	\caption{Summary of the weather and fountain observations. The weather measurements are shown using their
		mean ($\mu$) and standard deviation ($\sigma$) during the simulation duration as $\mu \pm \sigma$. }
	\label{tab:Observations}
	\begin{tabular}{@{}|lllllll|@{}}
		\toprule
		\textbf{}              & \textbf{Name}               & \textbf{Symbol} & \textbf{IN21} &
		\textbf{CH21}          & \textbf{CH20}               & \textbf{Units}                                                              \\ \midrule
		\multicolumn{1}{|l|}{\multirow{9}{*}{\rotatebox[origin=c]{90}{Weather}}}
		                       & Air temperature             & $T_a    $       & $0 \pm 7$     & $2 \pm 6$    & $2
		\pm 4$                 & $\degree C$                                                                                               \\
		\multicolumn{1}{|l|}{} & Relative humidity           & $RH     $       & $35 \pm 20$   & $79 \pm 18$  & $77
		\pm 17$                & \%                                                                                                        \\
		\multicolumn{1}{|l|}{} & Wind speed                  & $v_a        $   & $3 \pm 1$     & $2 \pm 2$    &
		$2 \pm 2$              & $m/s$                                                                                                     \\
		\multicolumn{1}{|l|}{} & Direct Shortwave            & $SW_{direct} $  & $246 \pm 333$ & $80 \pm 156$
		                       & $80 \pm 150$                & $W\,m^{-2}$                                                                 \\
		\multicolumn{1}{|l|}{} & Diffuse Shortwave           & $SW_{diffuse}$  & $0 \pm 0$     & $58 \pm 87$  & $51 \pm 74$  & $W\,m^{-2}$ \\
		\multicolumn{1}{|l|}{} & Incoming Longwave Radiation & $LW_{in}$       & $194 \pm 31$  & $239 \pm 35$ & $236 \pm 34$ & $W\,m^{-2}$ \\
		\multicolumn{1}{|l|}{} & Hourly Precipitation        & $ppt        $   & $0 \pm 0$     & $139 \pm
		457$                   & $95 \pm 404$                & $mm$                                                                        \\
		\multicolumn{1}{|l|}{} & Pressure                    & $p_a         $  & $623 \pm 3$   & $794 \pm 9$  &
		$798 \pm7$             & $hPa$                                                                                                     \\
		\multicolumn{1}{|l|}{} & Simulation Duration         & $h_{total} $    & 3673          & 4003
		                       & 1844                        & $hours$                                                                     \\
		\multicolumn{1}{|l|}{} & Simulation Start Date         &     & Jan 18 2021   & Nov 22 2020
		                       & Jan 3 2020                       &                                                                      \\\bottomrule
		\multicolumn{1}{|l|}{\multirow{4}{*}{\rotatebox[origin=c]{90}{Fountain}}}
		                       & Discharge rate             & $d_F     $      & $60$          & $7.5$        &
		$7.5$                  & $l/min$                                                                                                   \\
		\multicolumn{1}{|l|}{} & Runtime                     & $t_F $          & 829           & 2155
		                       & 1553                        & $hours$                                                                     \\
		\multicolumn{1}{|l|}{} & Spray radius                & $r_{F}$         & 10.8          & 6.9
		                       & 7.7                         & $m$                                                                         \\
		\multicolumn{1}{|l|}{} & Water temperature           & $T_{F}$         & 1             & 3
		                       & 3                           & $\degree C$                                                                 \\\midrule
	\end{tabular}
\end{table}


\subsection{Fountain observations}

We define the fountain used through four attributes, namely its spray radius, mean discharge quantity, discharge
runtime and water temperature as shown in Table \ref{tab:Observations}. Continuous measurement of the discharge
rate was unsuccessful in all the sites due to data logger malfunctions of the associated flowmeter. Instead the
discharge duration was first determined and then the available discharge measurement was used to determine the
average discharge quantity $d_F$ during these periods.  The spray radius $r_F$ was estimated from the mean AIR
circumference measured in the drone surveys during the fountain runtime.

The Swiss fountain discharge duration was extrapolated from just one fountain on and off event each.

Even though the Indian fountain was never manually switched off, there were many pipeline freezing events that
interrupted the discharge duration. Discharge rate was extrapolated to be the mean discharge $d_F$ except during
these pipeline freezing events.

\subsection{Drone surveys}

Several photogrammetric surveys using drones were conducted on the Swiss and Indian sites. The digital elevation
maps (DEMs) generated from the obtained imagery were analysed to document the radius, surface area and volume of
the ice structure. The number of surveys available for the IN21, CH21 and CH20 AIR were 6, 8 and 2 respectively
(see Table \ref{tab:uav}). The first drone flight was used to set the dome volume ($V_{dome}$) for model
initialisation. The remaining surveys were used for model calibration and validation. Since the Indian AIR was
built on top of another ice structure (see Fig. \ref{fig:2AIR}), it had a much higher dome volume compared to
the other AIRs.  The details of these surveys and the methodology used to produce the corresponding outputs are
explained in Appendix \ref{sec:uav} .

\begin{table}
	\centering
	\caption{ Summary of the drone surveys}
	\label{tab:uav}
	\begin{tabular}{@{}|llllll|@{}}
		\toprule
		\textbf{}              & \textbf{No.} & \textbf{Date} & \textbf{Volume} & \textbf{Radius} & \textbf{Surface Area} \\ \midrule
		\multicolumn{1}{|l|}{\multirow{6}{*}{\rotatebox[origin=c]{90}{IN21}}}
		                       & 1            & Jan 18, 2021  & 103 $m^{3}$     & 9.1 $m$
		                       & 411 $m^{2}$                                                                      \\
		\multicolumn{1}{|l|}{} & 2            & Feb 27, 2021  & 580 $m^{3}$     & 10.2 $m$
		                       & 668 $m^{2}$                                                                      \\
		\multicolumn{1}{|l|}{} & 3            & Mar 3, 2021   & 626 $m^{3}$     & 10.3 $m$
		                       & 694 $m^{2}$                                                                      \\
		\multicolumn{1}{|l|}{} & 4            & Mar 15, 2021  & 692 $m^{3}$     & 10 $m$
		                       & 681 $m^{2}$                                                                      \\
		\multicolumn{1}{|l|}{} & 5            & Mar 26, 2021  & 582 $m^{3}$     & 10.2 $m$
		                       & 671 $m^{2}$                                                                      \\
		\multicolumn{1}{|l|}{} & 6            & Apr 3, 2021   & 620 $m^{3}$     & 10.1 $m$
		                       & 658 $m^{2}$
		\\\midrule
		\multicolumn{1}{|l|}{\multirow{8}{*}{\rotatebox[origin=c]{90}{CH21}}}
		                       & 1            & Nov 22, 2020  & 13 $m^{3}$      & 5.4 $m$
		                       & 136 $m^{2}$                                                                       \\
		\multicolumn{1}{|l|}{} & 2            & Dec 2, 2020   & 26 $m^{3}$      & 5.7 $m$
		                       & 118 $m^{2}$                                                                       \\
		\multicolumn{1}{|l|}{} & 3            & Dec 30, 2020  & 43 $m^{3}$      & 7.5 $m$
		                       & 189 $m^{2}$                                                                       \\
		\multicolumn{1}{|l|}{} & 4            & Jan 9, 2021   & 82 $m^{3}$      & 6.5 $m$
		                       & 150 $m^{2}$                                                                       \\
		\multicolumn{1}{|l|}{} & 5            & Mar 6, 2021   & 108 $m^{3}$     & 7.5 $m$
		                       & 183 $m^{2}$                                                                       \\
		\multicolumn{1}{|l|}{} & 6            & Apr 2, 2021   & 83 $m^{3}$      & 6.5 $m$
		                       & 150 $m^{2}$                                                                       \\
		\multicolumn{1}{|l|}{} & 7            & Apr 16, 2021  & 64 $m^{3}$      & 6.2 $m$
		                       & 134 $m^{2}$                                                                       \\
		\multicolumn{1}{|l|}{} & 8            & Apr 24, 2021  & 37 $m^{3}$      & 4.7 $m$
		                       & 80 $m^{2}$                                                                       \\
		\midrule
		\multicolumn{1}{|l|}{\multirow{2}{*}{\rotatebox[origin=c]{90}{CH20}}}
		                       & 1            & Jan 3, 2020   & 24 $m^{3}$      & 6.7 $m$
		                       & 170 $m^{2}$                                                                      \\
		\multicolumn{1}{|l|}{} & 2            & Jan 24, 2020  & 59 $m^{3}$      & 7.7 $m$
		                       & 228 $m^{2}$                                                                      \\
		\midrule
	\end{tabular}

\end{table}

\section{Model setup}

A bulk energy and mass balance model is used to calculate the amounts of ice, meltwater, water vapour and runoff
water of the AIR. In each hourly time step, the model uses the AIR surface area, energy balance and mass
balance calculations to estimate its ice volume, surface temperature and runoff water as shown in Fig.
\ref{fig:schema} .

\begin{figure}
	\begin{center}
		\includegraphics[width=10 cm]{Figures/Figure_3.jpg}
	\end{center}
	\caption{Model schematic showing the workflow used in the model at every time step. }
	\label{fig:schema}
\end{figure}

\subsection{Surface area calculation} \label{sec:shape}

The model assumes the AIR shape to be a cone and assigns the following shape attributes:

\begin{subequations}

	% \label{equations}
	\begin{align}
		\label{eq:A}
		A_{cone}^i & = A_{corr}^i \cdot \pi \cdot r_{cone}^i \cdot \sqrt{{(r_{cone}^i)}^2 + {(h_{cone}^i})^ 2} \\
		\label{eq:V}
		V_{cone}^i & = \pi/3 \cdot {(r_{cone}^i)}^2 \cdot h_{cone}^i                                         \\
		\label{eq:thickness}
		j_{cone}^i & =\frac{\Delta M_{ice}^i}{\rho_{water}* A_{cone}^i}
	\end{align}
\end{subequations}

where $i$ denotes the model time step, $r_{cone}^i$ is the radius; $h_{cone}^i$ is the height; $A_{cone}^i$ is the surface area; $A_{corr}$ is a
correction factor with values between 1 and 2 that accounts for the deviation in AIR surface area from that of
the modelled conical surface; $V_{cone}^i$ is the volume and $j_{cone}$ is the AIR surface normal thickness
change as shown in Fig. \ref{fig:shape}. $M_{ice}^i$ is the mass of the AIR and $\Delta M_{ice}^i = M_{ice}^{i-1} -
  M_{ice}^{i-2}$. Henceforth, the equations used, display model time step superscript $i$ only if it is
  different from the current time step.

AIR volume can also be expressed as:

\begin{equation} V_{cone} =\frac{M_{ice}} {\rho_{ice}} \label{eq:V1} \end{equation}

where $\rho_{ice}$ is the density of ice (917 $kg\, m^{-3}$). 

The initial radius of the AIR is assumed to be $r_F$. The initial height $h_0$ depends on the dome volume
$V_{dome}$ used to construct the AIR as follows:

\begin{equation}
	h_{0} =  \Delta x + \frac{3 \cdot V_{dome}}{\pi \cdot (r_F)^2 }
	\label{eq:h0}
\end{equation}

where $\Delta x$ is the surface layer thickness (defined in Section \ref{sec:energy})

During subsequent time steps, the dimensions of the AIR evolve assuming a uniform thickness change ($j_{cone}$)
across its surface area with an invariant slope $s_{cone} = \frac{h_{cone}}{r_{cone}}$ .  During these time
steps, the volume is parameterised using Eqn. \ref{eq:V} as:

\begin{equation} V_{cone} = \frac{\pi \cdot {(r_{cone})}^3
		\cdot s_{cone}}{3} \label{eq:V2} \end{equation}

We define the Icestupa boundary through its spray radius, i.e. we assume ice formation is negligible when $r_{cone} >
	r_{F}$. Combining Eqns. \ref{eq:V},  \ref{eq:V1}, \ref{eq:h0} and \ref{eq:V2}, the geometric evolution of the
Icestupa at each time step $i$ can be determined by considering the following rules:

\begin{equation} (r_{cone},\, h_{cone}) = \left\{ \begin{array}{ll} (r_F ,\, h_0)                                                                          & \textit{ if } i=0 \\
		(r_{cone}^{i-1},\, \frac{3 \cdot M_{ice}}{\pi \cdot \rho_{ice} \cdot {(r_{cone}^{i-1})}^2}) & \textit{ if }
		r_{cone}^{i-1} \geq r_{F} \textit{ and } \Delta M_{ice} > 0                                                     \\ (\frac{3 \cdot M_{ice}}{\pi \cdot \rho_{ice} \cdot s_{cone}})^{1/3} \cdot (1,\,  s_{cone}) &
		otherwise\end{array} \right.  \label{eq:A2} \end{equation}



\subsection{Energy balance calculation} \label{sec:energy}

\begin{figure}
	\begin{center}
		\includegraphics[width=10 cm]{Figures/Figure_4.jpeg}
	\end{center}
	\caption{Shape variables of the AIR. $r_{cone}$ is the radius, $h_{cone}$ is the height, $j_{cone}$ is the
		thickness change and $s_{cone}$ is the slope of the ice cone. $r_F$ is the spray radius of the fountain.}
	\label{fig:shape}
\end{figure}

We approximate the energy balance at the surface of an AIR by a one-dimensional description of energy fluxes
into and out of a (thin) layer with thickness $\Delta x$:

\begin{equation}
	\rho_{ice} \cdot c_{ice} \cdot \frac{\Delta T}{\Delta t} \cdot \Delta x = q_{SW} + q_{LW} + q_{L} + q_{S} + q_{F} + q_{G}
	\label{eqn:EB}
\end{equation}

Upward fluxes are positive and downward fluxes are negative. The first term is the energy change of the surface
layer, which can be translated into a phase change energy should phase changes occur; $q_{SW}$ is the net
shortwave radiation; $q_{LW}$ is the net longwave radiation; $q_{L}$ and $q_{S}$ are the turbulent latent and
sensible heat fluxes. $q_{F}$ represents the heat exchange of the fountain water droplets with the AIR ice
surface. $q_{G}$ represents ground heat flux between the AIR surface and its interior.

The energy flux acts upon the AIR surface layer, which has an upper and a lower boundary defined by the
atmosphere and the ice body of the AIR, respectively. The parameter selection for $\Delta x$ is based on the
following two arguments: (a) the ice thickness $\Delta x$ should be small enough to represent the surface
temperature variations every model time step $\Delta t$ and (b) $\Delta x$ should be large enough for these
temperature variations to not reach the bottom of the surface layer. A sensitivity analysis was later performed
to understand the influence of this factor and decide its value. Here, we define the surface temperature
$T_{ice}$ to be the modelled average temperature of the Icestupa surface layer.

\subsubsection{Net Shortwave Radiation \texorpdfstring{$q_{SW}$}{Lg}}

The net shortwave radiation $q_{SW}$ is computed as follows:

\begin{equation} q_{SW} = (1- \alpha)\cdot (SW_{direct} \cdot f_{cone} + SW_{diffuse}) \label{eqn:SW} \end{equation}

where $SW_{direct}$ and $SW_{diffuse}$ are the direct and diffuse shortwave radiation, $\alpha$ is the
modelled albedo and $f_{cone}$ is the area fraction of the ice structure exposed to the direct shortwave
radiation.

The albedo varies depending on the water source that formed the current AIR surface layer. During the fountain
runtime, the albedo assumes a constant value corresponding to ice albedo. However, after the fountain is
switched off, the albedo can reset to snow albedo during snowfall events and then decay back to ice albedo. We
use the scheme described in \cite{OerlemansKnap_1998} to model this process. The scheme records the decay of
albedo with time after fresh snow is deposited on the surface. $\delta t$ records the number of time steps after
the last snowfall event. After snowfall, albedo changes over a time step, $\delta t$ , as

\begin{equation} \alpha=\alpha_{ice}+(\alpha_{snow}-\alpha_{ice}) \cdot e^{(-\delta t)/\tau} \label{eqn:a}
\end{equation}

where $\alpha_{ice}$ is the bare ice albedo value (0.25), $\alpha_{snow}$ is the fresh snow albedo value (0.85)
and $\tau$ is a decay rate (16 $days$), which determines how fast the albedo of the ageing snow recedes back to ice albedo.

The area fraction $f_{cone}$ of the ice structure exposed to the direct shortwave radiation depends on the shape
considered. Using the solar elevation angle $\theta_{sun}$, the solar beam can be considered to have a vertical
component, impinging on the horizontal surface (semicircular base of the AIR), and a horizontal component
impinging on the vertical cross section (a triangle). The solar elevation angle $\theta_{sun}$ used is modelled
using the parametrisation proposed by \cite{Woolf_1968}. Accordingly, $f_{cone}$ is determined as follows:

\begin{equation}
	\begin{split}
		f_{cone}& =\frac{(0.5 \cdot r_{cone} \cdot h_{cone}) \cdot cos \theta_{sun} +(\pi \cdot
			{(r_{cone})}^2/2) \cdot sin \theta_{sun} }{\pi \cdot r_{cone} \cdot ({(r_{cone})}^2+{(h_{cone})}^2)^{1/2}}\\
	\end{split}
	\label{eqn:f_{cone}}
\end{equation}

The diffuse shortwave radiation is assumed to impact the conical AIR surface uniformly.

\subsubsection{Net Longwave Radiation \texorpdfstring{$q_{LW}$}{Lg}} \label{sec:LW}

The net longwave radiation $q_{LW}$ is determined as follows:

\begin{equation}
	q_{LW}= LW_{in}-\sigma \cdot \epsilon_{ice} \cdot {(T_{ice}+ 273.15)}^4
	\label{eqn:LW}
\end{equation}

where $T_{ice}$ is the modelled surface temperature, both temperatures are given in [$\degree C$],
$\sigma=5.67\cdot10^{-8}\,Jm^{-2}s^{-1}K^{-4}$ is the Stefan-Boltzmann constant, $LW_{in}$ denotes the incoming
longwave radiation and $\epsilon_{ice}$ is the corresponding emissivity value for the Icestupa surface (0.97).

The incoming longwave radiation $LW_{in}$ for the Indian site, where no direct measurements were available, is
determined as follows:

\begin{equation}
	LW_{in}=\sigma \cdot (\epsilon_a \cdot {(T_a+ 273.15)}^4)
	\label{eqn:LWin}
\end{equation}

here $T_a$ represents the measured air temperature and $\epsilon_a$ denotes the atmospheric emissivity. We
approximate atmospheric emissivity $\epsilon_a$ using the equation suggested by \cite{Brutsaert_1982},
considering air temperature and vapor pressure (Eqn.  \ref{eqn:atm_e}). The vapor pressures over air and ice was
obtained using Eqn. \ref{eqn:vp}.  The expression defined in \cite{Brutsaert_1975} for clear skies (first term
in equation \ref{eqn:atm_e}) is extended with the correction for cloudy skies after \cite{Brutsaert_1982} as
follows:

\begin{equation}
	\epsilon_a=1.24 \cdot (\frac{p_{v,a}}{(T_a+273.15)})^{1/7}\cdot(1+0.22\cdot{c}^2) \label{eqn:atm_e}
\end{equation}

with a cloudiness index $c$, ranging from 0 for clear skies to 1 for complete overcast skies. For the Indian
site, we assume cloudiness to be negligible.

\subsubsection{Turbulent fluxes} \label{sec:Qs}

The turbulent sensible $q_{S}$ and latent heat $q_{L}$ fluxes are computed with the following expressions
proposed by \cite{Garratt_1992}:

\begin{equation}
	q_{S}=\mu_{cone}\cdot c_{a} \cdot \rho_{a} \cdot p_{a}/p_{0,a} \cdot \frac{\kappa^2 \cdot v_a \cdot
		(T_a-T_{ice})}{{(\ln{\frac{h_{AWS}}{z_{0}}})}^2}
	\label{eqn:qs}
\end{equation}

\begin{equation}
	q_{L}=\mu_{cone}\cdot 0.623 \cdot L_s \cdot \rho_{a}/p_{0,a} \cdot \frac{\kappa^2 \cdot
	v_a(p_{v,a}-p_{v,ice})}{{(\ln{\frac{h_{AWS}}{z_{0}}})}^2}
\end{equation}

where $h_{AWS}$ is the measurement height above the ground surface of the AWS (around $2\,m$ for all sites),
$v_a$ is the wind speed in [$m\,s^{-1}$], $c_a$ is the specific heat of air at constant pressure (1010 J
$kg^{-1} K^{-1}$), $\rho_{a}$ is the air density at standard sea level (1.29 $kg m^{-3}$), $p_{0,a}$ is the air
pressure at standard sea level (1013 $hPa$), $\kappa$ is the von Karman constant (0.4), $z_{0}$ is the surface
roughness (3 $mm$) and $L_s$ is the heat of sublimation (2848 $kJ\,kg^{-1}$).  The vapor pressures over air
($p_{v,a}$) and ice ($p_{v,ice}$) was obtained using the formulation given in \cite{WMO_2018} and
\cite{huang_2018} respectively  :

\begin{equation}
	\begin{split}
		p_{v,a}&=6.107 \cdot 10^{(7.5 \cdot T_a / (T_a + 237.3))} \cdot \frac{RH}{100}\\
		p_{v,ice}&=e^{(43.494 - \frac{6545.8}{T_{ice} + 278})}/(T_{ice} + 868)^2
	\end{split} \label{eqn:vp}
\end{equation}

where $p_{a}$ is the measured air pressure in [$hPa$].

The dimensionless parameter $\mu_{cone}$ is an exposure parameter that deals with the fact that AIR has a rough
appearance and forms an obstacle to the wind regime. This factor accounts for the larger turbulent fluxes due to
the roughness of the surface \citep{Oerlemans_2021}, and is a function of the AIR slope as follows:

\begin{equation}
	\mu_{cone} = 1 + \frac{s_{cone}}{2}
\end{equation}

A possible source of error is the fact that wind measurements from the horizontal plane at the AWS are used,
which might be different from those on a slope. However, without detailed datasets from the AIR surface, we
retain this assumption.

\subsubsection{Fountain discharge heat flux \texorpdfstring{$q_{F}$}{Lg} }

The fountain water temperature $T_F$ is assumed to cool to 0 $\degree C$. Thus, the heat flux caused by this
process is:

\begin{equation}
	q_{F} = \frac{ \Delta M_F \cdot c_{water} \cdot T_F}{\Delta t \cdot A_{cone}}
	\label{eqn:qF}
\end{equation}

with $c_{water}$ as the specific heat of water.

\subsubsection{Bulk Icestupa heat flux \texorpdfstring{$q_{G}$}{Lg}} \label{sec:Bulkflux}

The bulk Icestupa heat flux $q_{G}$ corresponds to the ground heat flux in normal soils and is caused by the
temperature gradient between the surface layer ($T_{ice}$) and the ice body ($T_{bulk}$). It is expressed by
using the heat conduction equation as follows:

\begin{equation} q_{G} = k_{ice} \cdot (T_{bulk}-T_{ice}^{i-1})/l_{cone} \label{eqn:qG}    \end{equation}

where $k_{ice}$ is the thermal conductivity of ice (2.123 $W\, m^{-1}\,K^{-1}$) , $T_{bulk}$ is the mean
temperature of the ice body within the Icestupa and $l_{cone}$ is the average distance of any point in the
surface to any other point in the ice body. $T_{bulk}$ is initialised as 0 $\degree C$ and later determined from
Eqn. \ref{eqn:qG} as follows:

\begin{equation} T_{bulk}^{i+1} = T_{bulk} - (q_{G} \cdot A \cdot \Delta t)/(M_{ice} \cdot c_{ice}) \end{equation}

Since AIRs typically have conical shapes with $r_{cone} > h_{cone}$, we assume that the center of mass of the cone
body is near the base of the fountain. Thus, the distance of every point in the AIR surface layer from the cone
body's center of mass is between $h_{cone}$ and $r_{cone}$. We calculate $q_{G}$ assuming $l_{cone} = (r_{cone} +
	h_{cone})/2$.

\subsubsection{Phase changes}

In this section, the numerical procedures to model phase changes at the surface layer are explained. Let
$T_{temp}$ be the calculated surface temperature. Even if the numerical heat transfer solution produces
temperatures which are $T_{temp}>0\, \degree C$, say from intense shortwave radiation, the ice temperature must
remain at $T_{temp} = 0\, \degree C$. The ‘‘excess’’ energy is used to drive the melting process. Moreover, the
energy input is used to melt the surface ice layer, and not to raise the surface temperature to some unphysical
value. Similarly, for freezing to occur, two conditions are required. Firstly, fountain water is present
($\Delta M_{F} > 0 $) and secondly the calculated temperature of the ice, $T_{temp}$, is below $0\, \degree C$.
Thus, depending on this surface temperature $T_{temp}$, the AIR can undergo further phase changes. So Eqn.
\ref{eqn:EB} can be rewritten as:

\begin{equation}
	% \rho_{ice} \cdot c_{ice} \cdot \frac{T_{temp}-T_{ice}}{\Delta t} \cdot \Delta x= q_{freeze/melt} + q_{T}
	q_{total}= q_{freeze/melt} + q_{T}
\end{equation}

where $q_{melt}$, $q_{freeze}$ and $q_{T}$ represent energy associated with melting, freezing and surface
temperature change processes respectively and the total energy available to be redistributed for these processes
is defined as $q_{total}=\rho_{ice} \cdot c_{ice} \cdot \frac{T_{temp}-T_{ice}}{\Delta t} \cdot \Delta x$.

We categorize every model time step as freezing or melting events. Freezing events can only occur, if fountain
water is available and $T_{temp}$ is below $0\,\degree C$. However, these two conditions are not sufficient as
the latent heat energy can only contribute to temperature fluctuations. Therefore, for preventing latent heat
energy from turning a melting event into a freezing event an additional condition namely, $(q_{total}-q_{L}) <
0$, is required.

\begin{equation}
	q_{freeze/melt} = \left\{ \begin{array}{ll}
		q_{freeze} & \textit{ if } \Delta M_{F} > 0 \textit{ and } T_{temp} < 0 \textit{ and }(q_{total}-q_{L}) < 0 \\
		q_{melt}   & \textit{ otherwise}
	\end{array} \right.
\end{equation}

During a freezing event, the AIR surface is assumed to warm to $0 \degree C$. The available energy
$(q_{total}-q_{L})$ is further augmented due to this change in surface temperature represented by the energy
flux:

$$q_{0} = \frac{\rho_{ice} \cdot \Delta x \cdot c_{ice} \cdot T_{ice}^{i-1}}{\Delta t}$$

The available fountain discharge may not be sufficient to utilize all the freezing energy. At such times, 
the additional freezing energy further cools down the surface temperature. Accordingly, the surface energy flux
distribution during a freezing event can be represented as:

% The available energy can either be sufficient or insufficient to freeze the fountain water available. If
% insufficient, the additional energy further cools down the surface temperature. Acoordingly, the surface energy flux
% distribution during a freezing event can be represented as:

\begin{equation}
	(q_{freeze}, q_{T}) = \left\{ \begin{array}{ll}
		(\frac{\Delta M_{F} \cdot L_f
		}{A_{cone} \cdot \Delta t}
		, q_{total}+\frac{\Delta M_{F} \cdot L_f
		}{A_{cone} \cdot \Delta t})          & \textit{ if  } \Delta M_{F} \textit{ insufficient }\\
		(q_{total}-q_{L}+q_{0}, q_{L}-q_{0}) & \textit{ otherwise }                                                                      \\
	\end{array} \right.
\end{equation}

If $T_{temp} > 0 \degree C$, then energy is reallocated from $q_{T}$ to $q_{melt}$ to maintain surface
temperature at melting point. The total energy flux distribution during a melting event can be represented as:

\begin{equation}
	(q_{melt}, q_{T}) = \left\{ \begin{array}{ll}
		(0, q_{total})                                                                                                                                                 & \textit{ if } T_{temp} < 0 \\
		(\frac{T_{temp} \cdot \rho_{ice} \cdot c_{ice} \cdot \Delta x}{\Delta t}, q_{total}-\frac{T_{temp} \cdot \rho_{ice} \cdot c_{ice} \cdot \Delta x}{\Delta t}  ) & \textit{ if } T_{temp} > 0
	\end{array} \right.
\end{equation}


\subsection{Mass balance calculation}

The mass balance equation for an AIR is represented as:

\begin{equation}
	\frac{\Delta M_{F} + \Delta M_{ppt} + \Delta M_{dep}}{\Delta t} = \frac{\Delta M_{ice} +\Delta M_{water} +
		\Delta M_{sub} + \Delta M_{runoff}}{\Delta t}  \\
	\label{eq:MB}
\end{equation}

where $M_{F}$ is the cumulative mass of the fountain discharge; $M_{ppt}$ is the cumulative precipitation;  $M_{dep}$ is the cumulative
accumulation through water vapour deposition; $M_{ice}$ is the cumulative mass of ice; $M_{water}$ is the cumulative
mass of melt water; $M_{sub}$ represents the cumulative water vapor loss by sublimation and $M_{runoff}$ represents the
fountain discharge runoff that did not interact with the AIR. The left hand side of equation \ref{eq:MB} represents the rate of
mass input and the right hand side represents the rate of mass output for an AIR.

Precipitation input is calculated as shown in equation \ref{eq:ppt} where $\rho_{w}$ is the density of water (1000
$kg\,m^{-3}$), $ppt$ is the measured precipitation rate in [$m\,s^{-1}$] and $T_{ppt}$ is the temperature threshold
below which precipitation falls as snow. Here, snowfall events were identified using $T_{ppt}$ as $1 \degree C$. Snow
mass input is calculated by assuming a uniform deposition over the entire circular footprint of the AIR.

The latent heat flux is used to estimate either the evaporation and condensation processes or sublimation and deposition
processes as shown in equation \ref{eq:vap}. During time steps at which surface temperature is below 0 $\degree C$ only
sublimation and deposition can occur, but if the surface temperature reaches 0 $\degree C$, evaporation and condensation
can also occur. As the differentiation between evaporation and sublimation (and condensation and deposition) when the
air temperature reaches 0 $\degree C$ is challenging, we assume that negative (positive) latent heat fluxes correspond
only to sublimation (deposition), i.e. no evaporation (condensation) is calculated.

Since we have categorized every time step as a freezing or melting event, we can determine the ice and meltwater
generated using the associated energy fluxes as shown in equations \ref{eq:mwat} and \ref{eq:mcone}. Having calculated
all the other mass components the fountain wastewater generated every time step can be calculated using Eqn.
\ref{eq:MB}.

\begin{subequations}
	\begin{align}
		\label{eq:ppt}
		\frac{\Delta M_{ppt}}{\Delta t}                                    & = \left\{ \begin{array}{ll} \pi \cdot
        {(r_{cone})}^2 \cdot
			\rho_{w}\cdot ppt & \textit{ if } T_{a} < T_{ppt} \\ 0 & \textit{ if } T_{a} \geq T_{ppt} \\\end{array} \right.                                             \\
		\label{eq:vap}
		(\frac{\Delta M_{dep}}{\Delta t}, \frac{\Delta M_{sub}}{\Delta t}) & = \left\{ \begin{array}{ll} \frac{q_{L}
			\cdot A_{cone}}{L_s}\cdot (1,0)  & \textit{ if } q_{L} \geq 0 \\ \frac{q_{L}
			\cdot A_{cone}}{L_s}\cdot (0,-1) & \textit{ if } q_{L} < 0    \\\end{array} \right.                                             \\
		\label{eq:mwat}
		\frac{\Delta M_{water}}{\Delta t}                                  & = \frac{q_{melt} \cdot A_{cone} }{L_f}                                                   \\
	  \label{eq:m_freeze/melt}
    \frac{\Delta M_{freeze/melt}}{\Delta t} & = \frac{q_{freeze/melt} \cdot A_{cone} }{L_f} \\
		\label{eq:mcone}
		\frac{\Delta M_{ice}}{\Delta t}                                    & = \frac{q_{freeze}\cdot A_{cone} }{L_f} + \frac{\Delta M_{ppt}}{\Delta t} + \frac{\Delta
			M_{dep}}{\Delta t}- \frac{\Delta M_{sub}}{\Delta t}- \frac{\Delta M_{water}}{\Delta t}
	\end{align}
\end{subequations}

Considering AIRs as water reservoirs, their net water loss can be defined as:

\begin{equation} \textit{Net water losses} = \frac{M_{runoff}+M_{sub}}{(M_F+M_{ppt}+M_{dep})} \cdot 100 \end{equation}

\subsection{Sensitivity and uncertainty analysis}

We used a polynomial chaos expansion approach (as in \cite{uncertainpy_2018}; \cite{Xiu_2005}) to evaluate the
model sensitivity and uncertainty. Polynomial chaos expansion is a much more efficient way to obtain similar
results compared to the computationally demanding Monte Carlo methods. This approach approximates the model with a
polynomial (as a surrogate model), on which sensitivity and uncertainty analysis can be performed.

The uncertainty in the model of estimating ice volumes are caused due to two sources, namely, the weather
and the fountain parameters. For the weather parameters, we first fix a range based on literature values
and then perform a global sensitivity analysis with the net water loss as the objective. The
distribution is always treated as uniform and the limits for every parameter are given in Table
\ref{tab:parameters}. The sensitivity analysis consists of a total ensemble size of 992 simulations per AIR. The parameter
sensitivity results from the sensitivity analysis are used as a tool to reduce the number of free parameters in the model by
removing those parameters, which have only a marginal influence on the model output.

The uncertainty associated with fountain parameters listed in Table \ref{tab:Observations} was quantified
separately. Fountain runtime has no uncertainty for the Swiss AIRs because no interruptions occured during the
study period. However, significant uncertainty exists for the IN21 AIR , where the interruptions due to
pipeline freezing events happened overnight but this was ignored in this analysis. The choice of $d_F$ for both
sites was just a best guess, based on few observations made by the flowmeter. So we associate this parameter by a
large uncertainty of $\pm \,50\, \%$. For the fountain water temperature, we set a lower and upper bound of $0\,
	\degree C$ and $3\, \degree C$.  It is very unlikely for the fountain water to have been beyond this range
considering winter conditions at all the sites.

Uncertainty also exists in the model input data, particularly for all the radiation measurements ($SW_{direct},
	SW_{diffuse}, LW_{in}$) since they were taken from ERA5 dataset or an AWS away from the construction sites.  But we are
not accounting for uncertainties related to meteorological forcing data in this analysis.

The sensitivity analysis and calibration was carried out for the CH21 and IN21 AIRs and the CH20 AIR was used for
validation of the model.


\begin{table}
	\caption{Free parameters in the model categorised as constant and uncertain parameters. The ranges of the 8 weather parameters used in the sensitivity and uncertainty analysis.}
	\label{tab:parameters}
	\begin{tabular}{@{}llllll@{}}
		\toprule
		\textbf{Constant Parameters}                       & \textbf{Symbol} & \textbf{Value} &
    \textbf{Unit} & \textbf{References} \\\midrule
    Van Karman constant & $\kappa$      & 0.4        &dimensionless & \citeauthor{CuffeyPaterson_2010}              \\
    Stefan Boltzmann constant & $\sigma$ & $\num{5.67 e-8} $& $W\, m^{-2}\, K^{-4}$ & \citeauthor{CuffeyPaterson_2010}\\
    Air pressure at sea level & $p_{0,a}$ & 1013 & $hPa$  & \citeauthor{MolgHardy_2004}\\
    Density of water & $\rho_{w}$ & 1000 & $kg\, m^{-3}$    & \citeauthor{CuffeyPaterson_2010}\\
    Density of ice & $\rho_{ice}$ & 917 & $kg\, m^{-3}$ & \citeauthor{CuffeyPaterson_2010}\\
    Density of air & $\rho_{a}$ &  1.29 & $kg\, m^{-3}$   & \citeauthor{MolgHardy_2004}\\
    Specific heat of water & $c_{w}$ & 4186 & $J\, kg^{-1}\,\degree C^{-1}$  & \citeauthor{CuffeyPaterson_2010}\\
    Specific heat of ice & $c_{ice}$ & 2097 & $J\, kg^{-1}\,\degree C^{-1}$ & \citeauthor{CuffeyPaterson_2010}\\
    Specific heat of air & $c_{a}$ & 1010 & $J\, kg^{-1}\,\degree C^{-1}$ & \citeauthor{MolgHardy_2004}\\
    Thermal conductivity of ice & $k_{ice}$ & 2.123  & $W\, m^{-1}\, K^{-1}$ & \citeauthor{Bonales_2017} \\
    Latent Heat of Sublimation & $L_{s}$ & \num{2.848e6}  & $J\, kg^{-1}$ &   \citeauthor{CuffeyPaterson_2010}\\
    Latent Heat of Fusion & $L_{f}$ & \num{3.34e5} & $J\, kg^{-1}$ & \citeauthor{CuffeyPaterson_2010}\\
    Gravitational acceleration & $g$ & 9.81 & $m\, s^{-2}$ &\citeauthor{CuffeyPaterson_2010}\\\midrule
    % Fountain spray radius & $r_{F}$             &             & $m$& measured \\
    % Model timestep & $\Delta t$ & 3600 & $s$ & assumed \\
    % Weather station height & $h_{AWS}$ & 2 & $m$ & assumed \\\midrule
		\textbf{Derived Parameters} & \textbf{Symbol} & \textbf{} & \textbf{Unit} & \textbf{Section} \\\midrule
    Atmospheric emissivity & $\epsilon_{a}$ & & dimensionless    & \ref{sec:LW}\\
    Cloudiness & $c$ &  & dimensionless  & \ref{sec:LW}\\
    Vapour pressure over air & $p_{v,a}$ &  & $hPa$  & \ref{sec:Qs}\\
    Vapour pressure over ice & $p_{v,ice}$ &  & $hPa$ & \ref{sec:Qs}\\
    Radius of AIR & $r_{cone}$ &  & $m$ & \ref{sec:shape}\\
    Height of AIR & $h_{cone}$ &  & $m$ & \ref{sec:shape}\\
    Slope of AIR  & $s_{cone}$ &  & dimensionless & \ref{sec:shape}\\
    Thickness change of AIR  & $j_{cone}$ &  & $m$  & \ref{sec:shape}\\
    Ice body and surface distance & $l_{cone}$ &  & $m$  & \ref{sec:Bulkflux}\\
\midrule
		\textbf{Calibrated Parameters} & \textbf{Symbol} & \textbf{Range} & \textbf{Unit} & \textbf{References} \\\midrule
    Surface Roughness                   & $z_0$                 & $[1,5]$            & $mm$  & \citeauthor{BrockWillisSharp_2006}       \\
    Surface Area correction factor      & $A_{corr}$            & $[1,2]$            & dimensionless       & assumed       \\
    Surface layer thickness             & $\Delta x$            & $[10,50]$           & $mm$ & assumed
    \\\midrule
		\textbf{Uncertain Weather Parameters} & \textbf{Symbol} & \textbf{Range} & \textbf{Unit} & \textbf{References} \\\midrule
    Ice Emissivity                      & $\epsilon_{ice}$      & $[0.95,0.99]$         & dimensionless & \citeauthor{HORI2006486}             \\
    Ice Albedo                          & $\alpha_{ice}$        & $[0.15,0.35]$         & dimensionless  &
    \citeauthor{steiner_2015}; \citeauthor{ZollesMaussion_2019}           \\
    Snow Albedo                         & $\alpha_{snow}$       & $[0.8,0.9]$        & dimensionless  & \citeauthor{ZollesMaussion_2019}              \\
    Precipitation Temperature threshold & $T_{ppt}$             & $[0,2]$            & $\degree C$& \citeauthor{Zhou_2010}  \\
    Albedo Decay Rate                   & $\tau$                & $[10,22]$           & $days$ &
    \citeauthor{Schmidt_2017};      \\
    & &    &  & \citeauthor{OerlemansKnap_1998}      \\\midrule
		\textbf{Uncertain Fountain Parameters} & \textbf{Symbol} & \textbf{Range} & \textbf{Unit} & \textbf{References} \\\midrule
    Discharge rate & $d_{F}$             & $\pm 50 \%$            & $l/min$& assumed  \\
    Water temperature & $T_{F}$             & $[0,3]$            & $\degree C$  & assumed  \\\bottomrule
	\end{tabular}
\end{table}

\section{Results}

\subsection{Sensitivity analysis}

The focus of the sensitivity analysis was not on the absolute sensitivity towards single parameters, but rather
to reduce the dimension of the parameter space. Therefore, the following discussion was limited to two classes:
parameters to which the model was sensitive ($S_{T_{i}} > 0.2$) and non-sensitive ($S_{T_{i}} \leq 0.2$). The
threshold of 0.2 was chosen since most of the parameters were bounded by it. For all the AIRs, just three
parameters were sensitive namely, $z_{0}$, $A_{corr}$ and $\Delta x$.

\begin{figure}
	\begin{center}
		\includegraphics[width=\linewidth]{Figures/Figure_5.jpg}
	\end{center}
	\caption{Observed ranges of the sensitive parameters used in the model optimization, shown by
		plotting the frequency distribution of the parameter values for the best 10 \% of the model runs for each
		objective. }
	\label{fig:param_hist}
\end{figure}

\subsection{Calibration}

The sensitive model parameters were calibrated based on two objectives, namely, the RMSE between the drone
surveys (see Table \ref{tab:uav}) and the model estimations of the ice volume and area. For calibration, the three sensitive
parameters namely, surface roughness, area correction factor and surface layer thickness were varied across
their ranges (defined in Table \ref{tab:parameters}) in steps of $1 \, mm$, 0.1 and $5 \, mm$ respectively. In
total, 495 model calibration runs were required for each of the objective. The frequency distribution of RMSE
among the best 10 \% calibration runs for each objective are shown in Fig.  \ref{fig:param_hist}. We choose values of
sensitive parameters populating more than 50 \% of the possibilities among the best 10 \% of the calibration
runs. For the volume objective, the RMSE of the best calibration runs ranged from 26 $m^3$ to 58 $m^3$ for the
IN21 and from 10 $m^3$ to 16 $m^3$ for the CH21. For the area objective, the RMSE of the best calibration runs
ranged from 14 $m^2$ to 53 $m^2$ for the IN21 and from 28 $m^2$ to 40 $m^2$ for the CH21. Note that this
calibration process has an inherent temporal and spatial bias due to the choice of when and how many drone
surveys were possible in each location. Among the 5 surveys of IN21 AIR used for calibration, most of them were
conducted around early March when the AIR volume was near its maximum whereas the 7 surveys of the CH21 location
were more evenly spaced out in comparison (see Table \ref{tab:uav}).

$A_{corr}$ shows different preferences whereas the other two parameters show similar preference for the two
objectives. We expect the measured surface area of the AIRs to be higher than the modelled ice area due to
the additional surface area caused by its various ice features. Particularly, we expect the IN21 surface area
correction factor to be higher than that of CH21 AIR because the IN21 surface area represented the
area of two ice cones merged into one (see Fig.  \ref{fig:2AIR}). This hypothesis is supported by the disjoint
frequency distribution of the area objective of the CH21 and IN21 AIR. Hence, we calibrated the IN21 area
correction factor to 1.5. Since no strong preference was observed for the CH21 area correction factor, we
calibrated it to the median of the values it assumed in the area objective, namely, 1.2.

The model surface layer thickness was selected based on the two conditions described in Section \ref{sec:energy}.
Since the minimum surface temperature does not go below $-30 \, \degree C$ (satisfies (b)) for the smallest
possible thickness of $20\, mm$ (satisfies (a)), we have taken this as the model surface layer thickness.

For the surface roughness, no clear preference was observed for the IN21 AIR, so we calibrated it to the median
value of $3 \, mm$. For the CH21 AIR, it was assigned its preferred value of $1 \, mm$.

\subsection{Weather and fountain uncertainty analysis}

The uncertainty in the ice volume estimates caused by the insensitive weather and fountain parameters are shown
in Fig. \ref{fig:results}. The ranges highlighted represent the 90 \% prediction interval of the ice volume
estimates. Weather uncertainty determination required 254 simulations whereas fountain uncertainty determination
required 200 simulations.

Weather uncertainty for IN21 was low compared to the other AIRs since precipitation and the associated variation in
albedo was negligible. This was expected since 4 out of the 5 insensitive parameters were part of the albedo module.
The remaining parameter, ice emissivity, caused the most variance in the ice volume estimate among all the
insensitive parameters ($S_{T_{i}} = 0.7$).

Fountain uncertainty for all the AIRs was high illustrating the importance of quantifying the fountain parameters for
a confident ice volume estimation. Among the 3 fountain parameters, ice volume variation was caused predominantly
by the uncertainty in the spray radius ($S_{T_{i}} = 0.8$).

\begin{figure}
	\begin{center}
		\includegraphics[width=\linewidth]{Figures/Figure_6.jpg}
	\end{center}
	\caption{Simulated ice volume during the lifetime of the AIRs (blue curve). The shaded regions (light blue and
		orange) represent the 90\% prediction interval of the AIR ice volume caused by the variations in weather and
    the fountain parameters, respectively. Violet points indicate the drone ice volume observations.  The grey
  dashed line represents the observed melt-out date for each AIR.  }

	\label{fig:results}
\end{figure}

\subsection{Validation}

The calibrated model was validated using two datasets, namely, CH20 AIR dataset and the melt-out observations for
each AIR. The melt-out date signifies the time when all the ice has completely disappeared and only the dome
volume remains. Model performance can be judged based on the difference between the model expectation and
observation of melt-out date.  So model performance for the CH21 and CH20 AIRs was -1 day and -16 days,
respectively. The negative sign represents underestimation of the melt-out date. For the IN21 AIR, the
determination of the melt-out date was not possible both through observation and through modelling.  In reality,
the IN21 AIR was found to have disintegrated into several ice blocks on 20th June, 2021.  Model expectation of
melt-out date was also not defined since the AIR never melted completely during the model simulation. There was
just one observation of the CH20 AIR volume (see Table \ref{tab:uav}). The RMSE of that observation
with the modelled volume was just $9\, m^3$.

\subsection{AIR ice volume estimates}

Since this model used a surface energy balance model commonly applied on glaciers, we analyse the AIR temporal
and spatial variation similar to how it is done for a glacier. Particularly, we used the AIR surface normal
thickness change ($j_{cone}$) as a measure to quantify the location influence. Note that $j_{cone}$ is similar
to the "specific mass balance" of a glacier with units $m \, w.\, e.$ but it was not conserved during the lifetime of the
AIR since the surface area varies unlike that of a glacier.  Similarly, we divided the simulation duration of the
AIR into accumulation and ablation periods. The accumulation (ablation) period ends (starts) at the last
fountain discharge event. The thickness rate during the accumulation and ablation period was referred to as
thickness growth and decay rate respectively.

The construction decisions responsible for the observed magnitude and variance of the ice volume estimates can
be categorised based on the fountain used and the location selected. According to Eqn.  \ref{eq:m_freeze/melt},
the freezing/melting rate of the AIRs can be decomposed to the corresponding freezing/melting energy and the
surface area. The construction location chosen determines the thickness growth/decay rate through the
freezing/melting energy flux and the fountain determines the surface area through its spray radius.

The influence of location can be further comprehended, if we analyse the daily surface normal thickness rate
together with the corresponding energy fluxes. Fig.  \ref{fig:MEB} shows the daily thickness and energy balance
components calculated with the calibrated parameters for the first and last 20 days for each AIR. The two time
periods selected were characteristic of the accumulation and ablation period, respectively. A strong variability
was evident between the accumulation and ablation periods and between the CH21 and the IN21 AIR.

The daily mean thickness rate of the Indian location was positive ($1\, mm \,w.e.$) with a daily mean growth
rate of $24\, mm \,w.e.$ and a mean decay rate of $12\, mm \,w.e.$. In the Swiss location, the daily mean
thickness rate was negative ($-3\, mm \,w.e.$) with a daily mean growth rate of $6\, mm \,w.e.$ and a mean decay
rate of $13\, mm \,w.e.$. The difference in magnitude between the growth and the decay rate corresponds to the
difference between the freezing and the melting energy balance components. For the Indian site, $q_{freeze}$ accounted for 63 \%,
$q_{melt}$ accounted for 32 \% and $q_{T}$ just 4 \% of overall energy turnover. The energy turnover is calculated
as the sum of energy fluxes in absolute values. For the Swiss site, $q_{melt}$ accounted for 62 \%, $q_{freeze}$
accounted for 36 \% and $q_{T}$ just 2 \% of overall energy turnover. The freezing events occurred for 19\% and
34\% of the simulation duration (see Table \ref{tab:Observations}) for the Indian and Swiss site respectively.
The accumulation period is characteristic of these freezing events and ablation period is charecteristic of the
melting events. We compare the energy turnover of different energy fluxes between these two periods to quantify
the influence of different surface processes.

\begin{table}
	\centering
	\caption{ Energy turnover of the energy balance components (EBC) during the accumulation and ablation periods
		with their daily mean ($\mu$) and standard deviation ($\sigma$) for each site. The positive/negative sign
		is indicative of the sign of the mean energy flux during the respective period.}
	\label{tab:turnover}
	\begin{tabular}{@{}|lllll|@{}}
		\toprule
		\textbf{}              & \textbf{EBC} & \textbf{Accumulation} & \textbf{Ablation} & \textbf{$\mu \pm \sigma
			$}                                                                                                             \\ \midrule
		\multicolumn{1}{|l|}{\multirow{6}{*}{\rotatebox[origin=c]{90}{IN21}}}
		                       & $q_{SW}$     & 6 \%                  & 11 \%             & $ 26 \pm 43 \, W\,m^{-2}$  \\
		\multicolumn{1}{|l|}{} & $q_{LW} $    & -48 \%                & -24 \%            & $ -80\pm 34 \, W\,m^{-2}$  \\
		\multicolumn{1}{|l|}{} & $q_{S}  $    & 22 \%                 & 45 \%             & $ 96 \pm140 \, W\,m^{-2}$  \\
		\multicolumn{1}{|l|}{} & $q_{L}  $    & -21 \%                & -20 \%            & $ -52 \pm 74 \, W\,m^{-2}$ \\
		\multicolumn{1}{|l|}{} & $q_{F}  $    & 2 \%                  & 0 \%              & $ 2 \pm 3 \, W\,m^{-2}$    \\
		\multicolumn{1}{|l|}{} & $q_{G}   $   & 1\%                   & 0 \%              & $ 0 \pm 2 \, W\,m^{-2}$    \\\midrule
		\multicolumn{1}{|l|}{\multirow{6}{*}{\rotatebox[origin=c]{90}{CH21}}}
		                       & $q_{SW} $    & 22 \%                 & 26 \%             & $ 37 \pm 56 \, W\,m^{-2}$  \\
		\multicolumn{1}{|l|}{} & $q_{LW} $    & -44 \%                & -31 \%            & $ -57 \pm 33 \, W\,m^{-2}$ \\
		\multicolumn{1}{|l|}{} & $q_{S}  $    & 19 \%                 & 34 \%             & $ 36 \pm 74 \, W\,m^{-2}$  \\
		\multicolumn{1}{|l|}{} & $q_{L}  $    & -8 \%                 & 9 \%              & $ -2 \pm 31 \, W\,m^{-2}$  \\
		\multicolumn{1}{|l|}{} & $q_{F}  $    & 6 \%                  & 0 \%              & $ 5 \pm 4 \, W\,m^{-2}$    \\
		\multicolumn{1}{|l|}{} & $q_{G}   $   & 0 \%                  & 1 \%              & $ 0 \pm 1 \, W\,m^{-2}$    \\\bottomrule
	\end{tabular}
\end{table}

To understand the overall impact of the radiation fluxes (longwave and shortwave) and the turbulent fluxes
(sensible and latent) on the freezing and melting energies, we sum their respective energy turnover taking into
account the sign of their mean energy during the accumulation/ablation period (see Table \ref{tab:turnover}). A
negative sign indicates that the corresponding energy flux increased/decreased the freezing/melting energy
respectively. Note that all the energy fluxes maintain the same sign for both the accumulation and ablation
periods for the Indian location but the latent heat changes sign for the Swiss location. The radiation fluxes
contributed -42 \% and -13 \% to the freezing and melting energies for the Indian location and -22 \% and -6 \%
to the Swiss location respectively.  Similarly, the turbulent fluxes at the Indian location contribute -1 \% and
25 \% and at the Swiss location contribute -11 \% and 43\%  respectively. So the AIR thickness growth rate was
driven by the net radiation fluxes and the AIR thickness decay rate was driven by the net turbulent fluxes.

Now we examine which surface process contributed the most to the net radiation fluxes and the net turbulent
fluxes respectively. To do so, we analyse the magnitudes of the thickness components and their corresponding
energy balance component (EBC) together using Fig.  \ref{fig:MEB}.  Particularly, ice thickness corresponds to
freezing EBC, melt thickness corresponds to melting EBC available and the net thickness represents the total EBC
available.  Snow deposition is calculated directly from the precipitation quantity and the
sublimation/deposition quantities corresponding to the latent heat EBC available.  The rest of the EBC are shown
to represent the different physical processes that contribute to this freezing and melting EBC .

\begin{figure}
	\begin{center}
		\includegraphics[width=\linewidth]{Figures/Figure_7.jpg} \end{center}
	\caption{Daily averages of thickness and energy balance components for the Indian and Swiss AIRs during the
		first 20 days of the accumulation and the last 20 days of the ablation period respectively.  } \label{fig:MEB}
\end{figure}

The longwave radiation flux had the highest energy turnover during the accumulation period for both the
locations. It increased and decreased the freezing and melting energy balance components during the accumulation and ablation period
respectively. However, its magnitude was much lower in the ablation period compared to the accumulation period
since the rising air temperature increased the incoming longwave radiation in the ablation period. The magnitude
of longwave radiation flux was much higher for the Indian site as its incoming longwave radiation was strongly
reduced due to its low cloudiness (see Table \ref{tab:Observations}).

Direct shortwave radiation was more than three times higher for the IN21 location due to its higher altitude and
lower latitude. However, the total energy turnover of shortwave radiation was two times lower for the Indian
site compared to the Swiss. Moreover, there was no significant difference in magnitude of the net shortwave
radiation absorbed by both the AIRs (see Table \ref{tab:turnover}). This was because the higher diffuse shortwave
radiation at the Swiss location was compensating for its lower direct shortwave radiation . Since the IN21 site
has mostly clear days, its diffuse shortwave radiation was very low (see Table \ref{tab:Observations}).
Moreover, less than half of the AIR surface area was exposed to direct shortwave radiation flux for both the
locations due to the area fraction $f_{cone}$. Temporal variation in the $f_{cone}$ factor due to increasing
solar elevation angle and decreasing AIR slope leads to higher shortwave radiation in the ablation period
compared to the accumulation period. Albedo, on the other hand, only varied temporally for the Swiss location
because there was no precipitation for the IN21 site.

\begin{table}
	\centering
	\caption{ Summary of the mass balance and AIR characteristics estimated by the model}
	\label{tab:Results}
	\begin{tabular}{@{}|llllll|@{}}
		\toprule
		\textbf{}              & \textbf{Name}                   & \textbf{Symbol} & \textbf{IN21} & \textbf{CH21} &
		\textbf{Units}                                                                                                       \\ \midrule
		\multicolumn{1}{|l|}{\multirow{3}{*}{\rotatebox[origin=c]{90}{Input}}}
		                       & Fountain discharge              & $M_F$           & \num{2.9e6}   & \num{1.0e6}     & $kg$  \\
		\multicolumn{1}{|l|}{} & Snowfall                        & $M_{ppt}$       & 0             & \num{5.5e4}   & $kg$  \\
		\multicolumn{1}{|l|}{} & Deposition                      & $M_{dep}$       & \num{1.2e4}   & \num{4e3}     & $kg$  \\ \midrule
		\multicolumn{1}{|l|}{\multirow{4}{*}{\rotatebox[origin=c]{90}{Output}}}
		                       & Meltwater                       & $M_{water}$     & \num{4.4e5} & \num{2.3e5}   & $kg$  \\
		\multicolumn{1}{|l|}{} & Ice                             & $M_{ice}$       & \num{1.5e5} & 0             & $kg$  \\
		\multicolumn{1}{|l|}{} & Sublimation                     & $M_{sub}$       & \num{1.3e5} & \num{8.0e3}     & $kg$  \\
		\multicolumn{1}{|l|}{} & Fountain runoff                 & $M_{runoff}$    & \num{2.3e6} & \num{8.0e5}     & $kg$  \\ \midrule
		\multicolumn{1}{|l|}{\multirow{7}{*}{\rotatebox[origin=c]{90}{AIR}}}

		                       & Freezing rate                   & ${\Delta M_{freeze}}/{\Delta t}$    & $14 \pm 7$    & $1 \pm 2$     & $l/min$ \\
		\multicolumn{1}{|l|}{} & Melting rate                    & ${\Delta M_{melt}}/{\Delta t}$      & $2 \pm 6$     & $1 \pm 2$     & $l/min$ \\
		\multicolumn{1}{|l|}{} & Thickness change                & $j_{cone}$      & $1 \pm 25$    & $-3 \pm 21$   &
		$mm \, w.\,e.$                                                                                                       \\
		\multicolumn{1}{|l|}{} & Accumulation, Ablation period   &                 & $52, 102$     & $91,79$       & $days$  \\
		\multicolumn{1}{|l|}{} & Net Water Loss                  &                 & 78            & 77
		                       & \%                                                                                          \\
		\multicolumn{1}{|l|}{} & Maximum ice Volume              &                 & 793           & 138           & $m^{3}$ \\
		\multicolumn{1}{|l|}{} & Surface Area                    & $A_{cone}$      & $488 \pm 94$  & $147 \pm 43$  & $m^{2}$ \\\midrule
		\multicolumn{1}{|l|}{\multirow{3}{*}{\rotatebox[origin=c]{90}{Model}}}
		                       & Meltout date error              &                 & N.A.          & 1             & $days$  \\
		\multicolumn{1}{|l|}{} & RMSE with ice volume        &                 & 86            & 14            & $m^{3}$ \\
		\multicolumn{1}{|l|}{} & Correlation with ice volume &                 & 0.99          & 0.93          &
		N.A.                                                                                                                 \\\bottomrule
	\end{tabular}
\end{table}


Turbulent fluxes play a very important role in the energy balance. Sensible heat flux had the highest energy
turnover during the ablation period for both the locations. It decreased and increased the freezing and melting
energy balance components. The Indian location had much higher sensible heat due to higher wind speeds and higher temperature gradient
between the AIR surface and the atmosphere. The sensible heat contributes much more to the energy turnover
during ablation period than the latent heat flux due to rising air temperature. Latent heat flux does not vary
much in energy turnover between the accumulation and ablation periods. For the Indian site, latent heat flux
increased and decreased the freezing and melting energy since sublimation process was favored throughout the
simulation duration. But for the Swiss location, latent heat increased both the freezing and the melting energy
since sublimation and deposition process was favored during accumulation and ablation period respectively.

The mass contribution of the sublimation/deposition process shown in Table \ref{tab:Results} was insignificant
compared to the energy flux contribution of this process since the heat of vaporization is around nine times
higher than the heat of fusion. The magnitude of the sublimation/deposition process was significantly different
for both the AIRs.  IN21 AIR lost 4 \% of its mass input to the sublimation process compared to the 1 \% mass
loss for the CH21 AIR (see Table \ref{tab:Results}). For the IN21 AIR, mass gain due to the deposition process
was negligible compared to the mass loss due to the sublimation process. For the CH21 AIR, half of the mass lost
to sublimation was regained by deposition. These process differences were primarily caused by the two fold
difference in relative humidity between the sites. This was expected since glaciers near the IN21 location have
been hypothesized to lose a significant amount of mass through sublimation as suggested by \cite{azam_2018}.

Hence, for both the locations, incoming longwave radiation and diffuse shortwave radiation were responsible for
how well the net radiation fluxes were able to augment the thickness growth rate. Similarly, the sublimation
process was responsible for how well the net turbulent fluxes were able to dampen the thickness decay rate.
Particularly, the Indian location had a much higher thickness growth rate because its incoming longwave
radiation and diffuse shortwave radiation during the accumulation period was much lower than that of the Swiss
location due to lower mean winter temperature and less cloudiness. The Indian location had a similar mean
thickness decay rate because its latent fluxes were acting against the sensible heat fluxes using the
sublimation process during the ablation period unlike in the Swiss location, where deposition process was
favoured during the ablation period due to higher relative humidity.

The fountain had some influence on the energy fluxes through its water temperature, temperature
forcing and albedo forcing . However, this influence was insignificant compared to its influence on the
surface area. As seen from the uncertainty analysis, ice volume estimates were most sensitive to the fountain
spray radius parameter. The variance of this surface area was quite low in the accumulation period since the ice
radius was initialised and bounded by the spray radius. So the thickness growth rate was uniformly scaled to
produce the corresponding ice volume. The higher spray radius of the Indian fountain resulted in a higher
maximum ice volume but this was at the expense of a faster melt-out date. This was because the fountain
determined spray radius increased both the freezing and the melting rate.  Moreover, the fountain parameters were
not independent, since fountain height (ignored in this analysis), discharge rate and spray radius were related
through the trajectories of the water droplets. So a proper optimization of the fountain is much
more complex and requires a closer look at the correlation of the fountain parameters amongst themselves and
with the freezing/melting energy flux. This will be investigated in a follow up study, with this study focusing
on the weather aspects of the model.

\section{Discussion}

\subsection{Water losses of AIRs}

The net water losses of IN21 and CH21 AIR were $78\,\%$ and $77\,\%$ of the total mass input, respectively. The
high water losses was caused by the fountain water runoff in both the AIRs. Since vapour losses were
negligible compared to fountain runoff losses, the AIR wastes water mostly during the accumulation period. The
maximum freezing rate of the IN21 AIR was less than half the mean water supply rate, meaning that the growth was
not limited by the supply rate but by the freezin efficiency. The CH21 AIR was able to attain the mean fountain
discharge provided but this was only for 6 hours from the 2155 hours of fountain runtime available. Thus, water
losses could have been significantly reduced by just decreasing the mean fountain discharge.

\subsection{Freezing and melting rates}

The freezing rates were much higher than the melting rates for the Indian location but these rates have a
negligible difference for the Swiss location. The smaller mean thickness decay rate compensated for the higher
surface area of the Indian location compared to the Swiss. The freezing rates of the IN21 AIR was much higher
than the CH21 AIR because of the much faster thickness growth rate and the larger surface area of the Indian
location compared to the Swiss.

\subsection{Favourable AIR locations}

The Indian location was favourable because its mean thickness rate was positive. Particularly, the mean
thickness growth rate was much higher than the decay rate. This enabled the fountain determined surface area to
favor AIR growth over decay. However, the Swiss location was not favourable since its mean thickness rate was
negative, indicating that the surface area favored AIR decay over growth (see Table \ref{tab:Results}).

Some weather conditions play a significant role in making the Indian AIR larger and survive longer than the
Swiss AIR, namely, cloudiness, temperature and relative humidity. The lower cloudiness and mean winter
temperature of the Indian location significantly reduce the net radiation flux during the accumulation period.
This enables a faster AIR thickness growth rate.  The lower humidity favours the sublimation over the deposition
process. So lower humidity decreases the magnitude of net turbulent fluxes during the ablation period. This
results in a slower thickness decay rate.  Hence, for AIRs with similar fountain parameters, we expect locations
with lower cloudiness, lower mean winter temperature to augment freezing rates and locations with lower humidity
to dampen melting rates.


\section{Conclusions}

In this paper, we have developed a bulk energy and mass balance model to simulate AIR evolution using data from
field measurements in Gangles, India and Guttannen, Switzerland. The use of these datasets, in combination with
the novel model allowed for an accurate representation of the complex evolution typical of an AIR. The model was
calibrated and validated with ice volume and surface area observations obtained via drone surveys. We calculated
the water losses, freezing and melting rates for each of the three AIRs and explained their corresponding
magnitudes in terms of the influence of the location chosen and the fountain used. Our main conclusions are
summarized below:

\begin{itemize}
	\item The model was successful in reproducing the observed ice volume evolution with a correlation greater
	      than $0.93$ and an RMSE less than $11 \, \%$ of the maximum ice volume for all the AIRs.

	\item The ice volume achieved after the accumulation period was much higher for the Indian AIR compared to the
	      Swiss AIRs. The lower radiation fluxes of the Indian location favored a faster thickness growth rate and the
	      spray radius of the Indian fountain produced a higher surface area compared to the Swiss counterparts. Thus,
	      the more than three times higher mean surface area and four times higher mean thickness growth rate during
	      the two times shorter accumulation period of the Indian location result in a six times higher maximum ice
	      volume of the Indian AIR compared to the Swiss.

	\item The ablation period of the Indian AIR was longer than the Swiss AIRs. The lower turbulent fluxes resulted in
	      a slower thickness decay rate on a larger surface area. This made the difference between the IN21 and CH21
	      melting rates negligible. Since the accumulation period produced much higher ice volumes, the Indian AIR was
	      able to last much longer than the Swiss AIRs.

	\item Water losses were high ($>75\,\%$) mostly due to fountain water runoff for all the AIRs. Vapour losses were
	      insignificant ($<4\,\%$) in comparison. However, significant improvement in water storage efficiency is possible
	      through optimization of fountain discharge rate.

	\item The Indian construction site produced long lasting AIRs with higher maximum ice volumes since it was
	      colder, drier and less cloudy compared to the Swiss construction site. Thus, the AIR technology is ideally
	      suited to serve as a water management strategy especially in dry mountain regions impacted by climate change
	      induced water stress.

\end{itemize}

\section{Appendix}

\subsection{Ladakh icestupa 2014/15} \label{sec:ladakhloss}

A 20 $m$ tall icestupa \citep{iceheight} was built in Phyang village, Ladakh at an altitude of 3500 $m$ a.s.l.
Assuming a conical shape with a diameter of 20 $m$, the corresponding volume of this icestupa becomes 2093 $m^3$ or
1,920 $m^3$ w.e. The fountain sprayed water at a rate of $210\, l\,min^{-1}$ \citep{waterinput} from $21^{st}$
January \citep{waterstart} to at least until $5^{th}$ March 2015 \citep{waterend} (around 43 nights). Assuming
fountain spray was active for 8 hours each night, we estimate water consumption to be around 4,334 $m^3$. Thus,
during the accumulation period of the icestupa, roughly 56 \% of the water provided was wasted.  This icestupa
completely melted away on $6^{th}$ July 2015 \citep{iceends}.

\subsection{Drone data processing} \label{sec:uav}

\begin{figure}
	\begin{center}
		\includegraphics[width=10 cm]{Figures/gangles_DEM.jpg}
	\end{center}
	\caption{Digital elevation map of Indian AIR constructed from the drone survey on March 3, 2021. The green
		area represents the area bounded by the marked perimeter.
	}
	\label{fig:DEM}
\end{figure}

The drone flew automatically along a predefined flight course and took photographs at a certain time interval. The
position and altitude of the drone at the exposure stations, which were obtained by the built-in integrated
Position and Orientation System (POS, composed of global positioning system and inertial measurement units),
were recorded in the JPEG pictures. Drone images in each survey were separately processed with Pix4Dmapper in a
three-step workflow, which is described below:

(1) Initial processing: This process generates a sparse point cloud with the structure-from motion algorithm
(\cite{Turner_2012}). First, it searches for and matches key points in the photos that have certain overlapping
areas using a feature matching algorithm (e.g. the scale-invariant feature transform (SIFT) algorithm, which can
detect key points in photos with different views and illumination conditions; \cite{Lowe_2004}). Second, the
approximate locations and orientations of the camera at each exposure station are reconstructed with the internal
parameters (focal length, coordinates of the principal point of the photograph), and external parameters (i.e. POS
data). A sparse point cloud is created.

(2) Point cloud densification: In this step, the multi-view stereo technique is applied to achieve a higher point
cloud density than in the previous step (\cite{Furukawa_2010}; \cite{Molg_2017}). Thus, the spatial resolution of
the products can be increased, and an irregular network for the next step can be created (\cite{Kung_2011}).

(3) AIR delineation: Ice radius, area and volume are the three main final products. Perimeter was manually marked
on the point cloud by identifying the AIR boundary (see Fig. \ref{fig:DEM}). For the Indian location, we identified identical rock features
near the ice boundary to mark as vertices of this perimeter. For the Swiss AIR, no such feature was available due
to snowfall, so instead the perimeter was marked by identifying the ice and snow boundary.

There is temporal and spatial uncertainty associated with this process. Weather conditions influence the quality
of each drone survey variably. Moreover, since ice/snow surfaces do not have many identifiable features, few
feature points can be detected and matched in the vicinity of the AIR. Thus, we attach a high uncertainty of
$\pm 10 \%$ for all the AIR observations to accommodate for this.


\section*{Conflict of Interest Statement} The authors declare that the research was conducted in the absence of
any commercial or financial relationships that could be construed as a potential conflict of interest.

\section*{Author Contributions} SB, MH, SW and FK designed the study.  SB developed the methodology with inputs
from MH.  MH, ML and JO reviewed the model algorithm and helped improve it. SB processed the drone data. JB helped
with model validation and uncertainty assessment. SB, MH, FK and SW participated in the fieldwork.  SB led the
writing of the paper and all co-authors contributed to it.

\section*{Funding} This work was supported and funded by the University of Fribourg and by the Swiss Government
Excellence Scholarship (SB). The associated field work in India was supported by Himalayan Institute of
Alternatives and funded by the Swiss Polar Institute.

\section*{Acknowledgments} This work would not have been possible without the untiring efforts of the Swiss and
Indian icestupa construction teams through the winters of 2019, 2020 and 2021. We thank Mr. Adolf Kaeser and Mr.
Flavio Catillaz from Eispalast Schwarzsee (CH19); Daniel Beurki from the Guttannen Bewegt Association (CH20 and
CH21); Norboo Thinles, Nishant Tiku, Sourabh Maheshwari and the icestupa project team from HIAL (IN21).  We
would also like to thank Hanseuli Gubler for designing the Swiss AWS; Dr. Tom Matthews for designing the Indian
AWS; Michelle Stirnimann for conducting the CH20 drone surveys and Digmesa AG for subsidising their flowmeter
used in the experiment.  We would particularly like to thank Prof. Thomas Schuler and 2 anonymous reviewers who
gave us important inputs to improve the paper. We also thank Prof. Christian Hauck, Prof.  Nanna B. Karlsson and
Dr.  Andrew Tedstone for valuable suggestions that improved the manuscript.

\section*{Data Availability Statement} AIR timelapses (CH20, CH21) and results can be viewed interactively in
the web app (\url{https://share.streamlit.io/gayashiva/air_model/src/visualization/webApp.py}).  The latest version of
the model code is available in GitHub (\url{https://github.com/Gayashiva/air_model}, last access: 10 August 2020). The
drone data can be obtained from the authors upon request.

\bibliographystyle{frontiersinSCNS_ENG_HUMS} \bibliography{references}

\end{document}
