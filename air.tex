\documentclass[utf8]{frontiersSCNS} % for Science, Engineering and Humanities and Social Sciences articles
% \documentclass[utf8]{frontiersFPHY} % for Physics and Applied Mathematics and Statistics articles
\usepackage{gensymb}
\usepackage{url,hyperref,lineno,microtype,subcaption}
\usepackage[onehalfspacing]{setspace}

\usepackage{tabularx}
\linenumbers
\DeclareUnicodeCharacter{0301}{}
\DeclareUnicodeCharacter{2212}{}
\usepackage{wasysym} % provides \DH, \dh, \Thorn, \thorn
% Leave a blank\usepackage{amsmath}
%\DeclareMathOperator{\sign}{sign} line between paragraphs instead of using \\

% \usepackage{csvsimple} % for csv tables
\usepackage{booktabs}
\usepackage{multirow}

\def\keyFont{\fontsize{8}{11}\helveticabold }
\def\firstAuthorLast{Balasubramanian {et~al.}} %use et al only if is more than 1 author
\def\Authors{Suryanarayanan Balasubramanian\,$^{1*,3}$, Martin Hoelzle\,$^{1}$, Michael Lehning\,$^{2}$, Sonam
	Wangchuk\,$^{3}$, Johannes Oerlemans\,$^{4}$, Felix Keller\,$^{5,6}$ and Jordi Bolibar\,$^{4}$}
\def\Address{$^{1}$University of Fribourg, Fribourg, Switzerland\\
	$^{2}$WSL Institute for Snow and Avalanche Research, Davos, Switzerland\\
	$^{3}$Himalayan Institute of Alternatives Ladakh, Leh, India\\
	$^{4}$Institute for Marine and Atmospheric Research, Utrecht University, Utrecht, The Netherlands\\
	$^{5}$Academia Engiadina, Samedan, Switzerland\\
	$^{6}$ETH, Zürich, Switzerland}
\def\corrAuthor{Suryanarayanan Balasubramanian}

\def\corrEmail{suryanarayanan.balasubramanian@unifr.ch}



\begin{document}
\onecolumn
\firstpage{1}

\title[Artificial Ice Reservoirs]{Influence of construction location on artificial ice reservoir (Icestupa)
	evolution: case studies from the Swiss Alps and Indian Himalayas}

\author[\firstAuthorLast ]{\Authors}
\address{}
\correspondance{}

\extraAuth{}

\maketitle


\begin{abstract}

	Artificial Ice Reservoirs (AIR) are used to provide irrigation water for communities in Ladakh, India facing
	climate change induced water stress. This study attempts to quantify the influence of this construction location
	and thereby the potential of this water storage technology in other climatic zones. A cone-shaped AIR popularly
	called Icestupa is simulated based on a surface energy balance model. AIR surface processes were quantified using
	meteorological data in conjunction with fountain discharge information (mass input of an AIR) to estimate the
	quantity of frozen, melted, evaporated and runoff water for two sites in Switzerland and one in India. At these
	measurement sites, AIR ice volume and area were measured using Uncrewed Aerial Vehicles (UAV) for model
	calibration and validation purposes.  The Indian location, with a maximum ice volume 5 times larger, was more
	favourable compared to the Swiss primarily because of much higher sublimation rates.  However, the storage efficiency of
	the Indian AIR (19 \%) was actually lower than the Swiss AIR (23 \%) due to significantly more fountain water
	runoff. Although the uncertainties due to weather and fountain parameters were significant, the model achieved a
	high correlation ($r^2 > 0.9$) with the UAV ice volume observations and was able to estimate their survival
	duration with an accuracy of $\pm\, 16\, days $ for two construction locations with radically different climates.

	\tiny
	\keyFont{ \section{Keywords:} icestupa, water storage, climate change adaptation, geoengineering } %All article types: you may provide up to 8 keywords; at least 5 are mandatory.
\end{abstract}

\section{Introduction}

Seasonal snow cover, glaciers and permafrost are expected to change their water storage capacity due to climate change
with major consequences for downriver water supply \citep{Immerzeel_2020}. The challenges brought about by these changes
are especially important for dry mountain environments such as in Central Asia or the Andes, which directly rely on the
seasonal meltwater for their farming and drinking needs \citep{HoelzleBarandun_2019, Apel_2018, Buytaert_2017,
	Chen_2016, UNGERSHAYESTEH_2013}. Some villages in Ladakh, India have already been forced to relocate due to glacial
retreat and the corresponding loss of their main fresh water resources \citep{zanskar}.

\begin{figure} \begin{center} \includegraphics[width=10 cm]{Figures/Figure_1.jpg}
	\end{center} \caption{Icestupa in Ladakh, India on March 2017 was 24 $m$ tall and contained around 3700 $m^3$
		of water. Picture Credits: Lobzang Dadul} \label{fig:old_icestupa} \end{figure}

Artificial ice reservoirs (AIR) have been considered to be a feasible way to adapt to these changes
\citep{IPCC_2019,10.1659/MRD-JOURNAL-D-18-00072.1}. An AIR is a human-made ice structure typically constructed during
the cold winter months and designed to slowly release freshwater during the warm spring and summer months. The main
purpose of AIRs is irrigation. Therefore, AIRs are designed to store water in the form of ice as long into the summer as
possible.  The energy required to construct an AIR is usually derived from the gravitational head of the source water
body. Some are constructed horizontally by freezing water using a series of checkdams and others are built vertically by
spraying water through fountain systems \citep{Nusser_2018}. The latter are colloquially referred to as Icestupas and
are the subject of this study.

A typical AIR just requires a fountain nozzle mounted on a supply pipeline. The water source is usually a high altitude
lake or glacial stream. Due to the altitude difference between the pipeline input and fountain output, water ejects from
the fountain nozzle as droplets that eventually lose their latent heat to the atmosphere and accumulate as ice. The
fountain nozzle is raised through addition of metal pipes as and when significant ice accumulates.  Typically, a dome of
branches is constructed around the metal pipes so that such pipe extensions can be done from within this dome. During
the winter, the fountain is manually activated from sunset to sunrise. Threads, tree branches and fishing nets are used
to guide and accelerate the ice formation.

Since their invention in 2013 \citep{campaign}, Icestupas have gained widespread publicity in the region of Ladakh,
Northern India since they require very little infrastructure, skills and energy to be constructed in comparison to other
water storage technologies. Compared to other AIR geometries, Icestupas (Fig. \ref{fig:old_icestupa}) can be built at lower
altitudes and last much longer into the summer than other types of ice structures \citep{campaign}. However, to date, no
reliable estimates exist about the quantity of meltwater they can provide \citep{Nusser_2018}.

In this paper, we develop a physically-based model of a vertical AIRs (or Icestupas) that can quantify the
influence of the location and the fountain used on their freezing and melting rates. Mass and energy balance
equations were used to estimate the quantity of water frozen, melted, evaporated and runoff. Sensitivity and
uncertainty analysis were performed to identify the most critical parameters and the variance caused by them. For
calibration, we chose two AIR built across the winters of 2020 and 2021 in India and Switzerland and validated the
calibrated model on a Swiss AIR built on winter 2019. Our model results provide first steps towards evaluating the
potential of this new water storage technology worldwide.

\section{Study Sites}
The model requires three kinds of datasets containing weather, fountain and UAV measurements to accurately calibrate,
estimate and validate the ice volume of AIRs. Through the winters of 2019, 2020 and 2021 several scientific AIRs were
constructed by teams in Switzerland and India. Here, we present the results of three scientific AIRs, which have
relatively complete dataset. As shown in Table \ref{tab:Observations}, two of them were constructed in the same Swiss
location called Guttannen (referred with the prefix CH) but during different winters and the other was constructed at
Gangles, India (referred with the prefix IN).

The Guttannen site (46.66 $\degree$N, 8.29 $\degree$E) in the Bern region lies at 1047 $m$ a.s.l.. In the winter
(Oct-Apr), mean daily minimum and maximum air temperatures vary between -13 and 15 $\degree C$. Clear skies are rare,
averaging around 7 days during winter \citep{guttannen}. The site was situated adjacent to a stream resulting in high
humidity values across the study period as shown in Fig. \ref{fig:2AIR}. AIR were constructed here by the Guttannen
Bewegt Association during the winters of 2019-20 (CH20) and 2020-21 (CH21). Tree branches were laid covering the
fountain pipe to initiate the ice formation process. The fountain height varied between 2 to 5\,$m$ during the
construction period. The water was transferred from a spring water source and flowed via a flowmeter to the nozzle. In
addition, a webcam guaranteed a continuous survey of the site during the construction of the AIR.

\begin{figure}
	\begin{center}
		\includegraphics[width=12 cm]{Figures/2AIR.jpg}
	\end{center}
	\caption{The Swiss and Indian AIR on March 3 and January 9, 2021 respectively. Picture credits: Daniel Buerki (left)
		and Thinles Norboo (right)}
	\label{fig:2AIR}
\end{figure}

The Gangles site (34.22 $\degree$N, 77.61 $\degree$E) is located around 20 km north of Leh city in the Ladakh region,
lying at 4025 $m$ a.s.l.. AIR were constructed by the Himalayan Institute of Alternatives, Ladakh (HIAL) every winter
since 2018. The AIR was constructed as part of the Icestupa Competition in Gangles, Ladakh, India (IN21). Fountain
height varied between 5 to 9\,$m$.

\subsection{Meteorological data}
Air temperature, relative humidity, wind speed, pressure, longwave, shortwave direct and diffuse radiation are required
to calculate the surface energy balance of an AIR.

For the CH site, the primary weather data source was a Meteoswiss AWS located 184 m away. In addition, we used ERA5
reanalysis dataset \citep{era5} for filling data gaps and adding data that were not measured directly.  The ERA5
reanalysis dataset has a good correlation with sites in Switzerland \citep{Scherrer_2020}. The ERA5 grid point
chosen (46.64 $\degree$N, 8.25 $\degree$E) for the Swiss site was around 3.6 km away from the actual site.  All the
ERA5 variables were therefore fitted with the meteoswiss dataset via linear regressions. The zero wind speed values
recorded by the Meteoswiss AWS whenever snow accumulated on the ultrasonic wind sensor were replaced using the ERA5
dataset.

For the IN site, three different weather data sources were used to log all the weather parameters required for the
model. A temperature and humidity logger was placed adjacent to the AIR on a mast. Wind speed and pressure data was
logged via a campbell weather station located 440 m away. Shortwave radiation data was derived from another campbell
weather station located 15 km away. Unfortunately, precipitation was not logged and was assumed to be negligible. The
diffuse fraction of the global shortwave radiation was also assumed to be negligible.

\begin{table}
	\centering
	\caption{ Summary of the weather and fountain observations}
	\label{tab:Observations}
	\begin{tabular}{@{}|lllllll|@{}}
		\toprule
		\textbf{}              & \textbf{Name}               & \textbf{Symbol}     & \textbf{IN21} &
		\textbf{CH21}          & \textbf{CH20}               & \textbf{Units}                                                                 \\ \midrule
		\multicolumn{1}{|l|}{\multirow{9}{*}{\rotatebox[origin=c]{90}{Weather}}}
		                       & Air temperature             & $T_a    $           & $0 \pm 6$     & $1 \pm 5$    & $2
		\pm 4$                 & $\degree C$                                                                                                  \\
		\multicolumn{1}{|l|}{} & Relative humidity           & $RH     $           & $39 \pm 17$   & $79 \pm 17$  & $75
		\pm 17$                & \%                                                                                                           \\
		\multicolumn{1}{|l|}{} & Wind speed                  & $v_a        $       & $2 \pm 1$     & $2 \pm 2$    &
		$2 \pm 2$              & $m/s$                                                                                                        \\
		\multicolumn{1}{|l|}{} & Direct Shortwave            & $SW_{direct} $      & $243 \pm 331$ & $77 \pm 152$
		                       & $83 \pm 156$                & $W\,m^{-2}$                                                                    \\
		\multicolumn{1}{|l|}{} & Diffuse Shortwave           & $SW_{diffuse}$      & $0 \pm 0$     & $54 \pm 82$  & $51 \pm 73$ & $W\,m^{-2}$ \\
		\multicolumn{1}{|l|}{} & Incoming Longwave Radiation & $LW_{in}$           & $200 \pm 34$  & $239 \pm 35$ & $51 \pm 73$ & $W\,m^{-2}$ \\
		\multicolumn{1}{|l|}{} & Hourly Precipitation        & $ppt        $       & $0 \pm 0$     & $129 \pm
		446$                   & $92 \pm 400$                & $mm$                                                                           \\
		\multicolumn{1}{|l|}{} & Pressure                    & $p_a         $      & $622 \pm 3$   & $793 \pm 8$  &
		$797 \pm7$             & $hPa$                                                                                                        \\
		\multicolumn{1}{|l|}{} & Observation Duration        & $h_{total} $        & 3456          & 4042
		                       & 2252                        & $hours$                                                                        \\\bottomrule
		\multicolumn{1}{|l|}{\multirow{4}{*}{\rotatebox[origin=c]{90}{Fountain}}}
		                       & Mean discharge              & $d_F     $          & $60$          & $7.5$        &
		$7.5$                  & $l/min$                                                                                                      \\
		\multicolumn{1}{|l|}{} & Runtime                     & $h_F $              & 829           & 2155
		                       & 1553                        & $hours$                                                                        \\
		\multicolumn{1}{|l|}{} & Spray radius                & $r_{F} [m]$         & 10.8          & 6.9
		                       & 7.7                         & $m$                                                                            \\
		\multicolumn{1}{|l|}{} & Water temperature           & $T_{F} [\degree C]$ & 1             & 3
		                       & 3                           & $\degree C$                                                                    \\\midrule
	\end{tabular}
\end{table}

\subsection{Uncrewed Aerial Vehicle surveys}

Several uncrewed aerial vehicle (UAV) surveys were conducted in the Swiss and Indian sites.The DEM generated
through these flights were analysed to obtain the radius, area and volume of the ice structure.  The first drone
flight was used to set the dome volume ($V_{dome}$) for model initialisation. Since the Indian AIR was built on top
of another ice structure, it had a much higher dome volume compared to the other AIRs.The details of these surveys
and the methodology used to produce the corresponding outputs are explained in Appendix \ref{sec:uav} .

\subsection{Fountain observations}
We define the fountain used in AIR construction through four attributes, namely its spray radius, mean discharge
quantity, discharge runtime and water temperature as shown in Table \ref{tab:Observations}. Continuous measurement of
the discharge rate was unsuccessful in all the sites. Instead the discharge duration was first determined and then the
available discharge measurement was used to determine the average discharge quantity $d_F$ during these periods as shown
in the Appendix. The spray radius $r_F$ was estimated from the mean AIR circumference measured in the drone flights
during the fountain runtime.

For the Swiss site, the fountain was never switched off so the discharge duration was extrapolated from just one
fountain on and off event each.

For the Indian site, even though the fountain was never manually switched off, there were many pipeline freezing events
that interrupted the discharge duration. Discharge rate was extrapolated to be the mean discharge $d_F$ except during
these pipeline freezing events.

\section{Model setup}

A bulk energy and mass balance model is used to calculate the amounts of ice, meltwater, water vapour and runoff water
of the AIR every hour. This model consists of four modules which estimates the AIR, a) geometric evolution, b) energy
balance, c) surface temperature, d) mass balance and e) parameter sensitivity.

\begin{figure} \begin{center} \includegraphics[width=12 cm]{Figures/model_schematic.png} \end{center} \caption{Model
		schematic showing the algorithm used in the model at every time step. } \label{fig:schema} \end{figure}

\subsection{Geometric evolution}

Radius $r_{ice}^i$ and height $h_{ice}^i$ define the dimensions of the AIR assuming its geometry to be a cone. The
surface area $A^i$ and volume $V^i$ are:

\begin{equation} A = A_{corr} \cdot \pi \cdot r_{ice} \cdot \sqrt{{r_{ice}}^2 + {h_{ice}}^ 2} \label{eqn:A} \end{equation}

\begin{equation} V = \pi/3 \cdot {r_{ice}}^2 \cdot h_{ice} \label{eqn:V} \end{equation}

where $A_{corr}$ is a correction factor with values between 1 and 2 that accounts for the increase in surface area
due to the irregular surface of the AIR. We do not specify the time step superscript $i$ of the shape variables
$A$, $V$, $r_{ice}$ and $h_{ice}$. The equations used, display model time step superscript $i$ only if it is
different from the current time step.

With the mass of the AIR $M_{ice}$, its current volume can also be expressed as:

\begin{equation} V = M_{ice} /\rho_{ice} \label{eqn:V1} \end{equation}

where $\rho_{ice}$ is the density of ice (917 $kg\, m^{-3}$).


The influence of the AIR fountain is parameterised by the fountain water temperature $T_{F}$ and its spray radius $r_F$.
The initial radius of the AIR is assumed to be $r_F$. The initial height $h_0$ depends on the dome volume $V_{dome}$
used to construct the AIR as follows:

\begin{equation}
	h_{0} =  \Delta x + \frac{3 \cdot V_{dome}}{\pi r_F^2 }
	\label{eqn:h0}
\end{equation}

where $\Delta x$ is the surface layer thickness (defined in Section \ref{section:EB})

During subsequent time steps, the dimensions of the AIR evolve assuming a uniform ice formation and decay across its
surface area with an invariant slope $s_{cone} = \frac{h_{ice}}{r_{ice}}$ .  During these time steps, the volume is
parameterised using Eqn. \ref{eqn:V} as:

\begin{equation} V = \frac{\pi \cdot {r_{ice}}^3
		\cdot s_{cone}}{3} \label{eqn:V2} \end{equation}


However, the Icestupa cannot outgrow the maximum range of the water droplets ($(r_{ice})_{max} = r_{F}$). Combining
Eqns. \ref{eqn:V},  \ref{eqn:V1}, \ref{eqn:h0} and \ref{eqn:V2}, the geometric evolution of the Icestupa at each time
step $i$ can be determined by considering the following rules:

\begin{equation} (r_{ice},\, h_{ice}) = \left\{ \begin{array}{ll} (r_F ,\, h_0)                                                                        & \textit{ if } i=0 \\
             (r_{ice}^{i-1},\, \frac{3 \cdot M_{ice}}{\pi \cdot \rho_{ice} \cdot {(r_{ice}^{i-1})}^2}) & \textit{ if }
             r_{ice}^{i-1} \geq r_{F} \textit{ and } \Delta M_{ice} > 0                                                    \\ (\frac{3 \cdot M_{ice}}{\pi \cdot \rho_{ice} \cdot s_{cone}})^{1/3} \cdot (1,\,  s_{cone}) &
             otherwise\end{array} \right.  \label{eqn:A2} \end{equation}

where $\Delta M_{ice} = M_{ice}^{i-1} - M_{ice}^{i-2}$

\begin{figure} \begin{center} \includegraphics[width=10
			cm]{Figures/shape_parameters.jpeg} \end{center} \caption{Shape variables and fountain constants of the AIR. $r_{ice}$ is
		the radius, $h_{ice}$ is the height and $s_{cone}$ is the slope of the ice cone. $r_F$ is the spray radius, $h_F$ is the
		height and $T_F$ is the water temperature of the fountain.} \label{fig:shape} \end{figure}

\subsection{Energy Balance} \label{section:EB}

The energy balance equation (e.g. \cite{Hock_2005}) for the AIR is formulated as follows:

\begin{equation} q_{surf} = q_{SW} + q_{LW} + q_{L} + q_{S} + q_{F} + q_{G}\label{eqn:EB} \end{equation}

where $q_{surf}$ is the surface energy flux in [$W\,m^{-2}$]; $q_{SW}$ is the net shortwave radiation; $q_{LW}$ is the
net longwave radiation; $q_{L}$ and $q_{S}$ are the turbulent latent and sensible heat fluxes. $q_{F}$ represents the
heat exchange of the fountain water droplets with the AIR ice surface. $q_{G}$ represents ground heat flux between the
AIR surface and its interior. Energy transferred in the direction of the ice surface is always denoted as positive and
away as negative.

Equation \ref{eqn:EB} is usually referred to as the energy budget for “the surface”, but practically it must apply to a
surface layer of ice with a finite thickness $\Delta x$. The energy flux acts upon the AIR surface layer, which has an
upper and a lower boundary defined by the atmosphere and the ice body of the AIR, respectively. The parameter selection
for $\Delta x$ is based on the following two arguments: (a) the ice thickness $\Delta x$ should be small enough to
represent the surface temperature variations every model time step $\Delta t$ and (b) $\Delta x$ should be large enough
for these temperature variations to not reach the bottom of the surface layer. A sensitivity analysis was later
performed to understand the influence of this factor and decide its value. Here, we define the surface temperature
$T_{ice}$ to be the modelled average temperature of the Icestupa surface layer and the energy flux $q_{surf}$ is assumed
to act uniformly across the Icestupa area $A$.

\subsubsection{Net Shortwave Radiation \texorpdfstring{$q_{SW}$}{Lg}}

The net shortwave radiation $q_{SW}$ is computed as follows:
\begin{equation} q_{SW} = (1- \alpha)\cdot (SW_{direct} \cdot f_{cone} + SW_{diffuse}) \label{eqn:SW} \end{equation}

where $SW_{direct}$ and $SW_{diffuse}$ are the ERA5 direct and diffuse shortwave radiation, $\alpha$ is the modelled
albedo and $f_{cone}$ is the area fraction of the ice structure exposed to the direct shortwave radiation.

The albedo varies depending on the water source that formed the current AIR surface layer. During the fountain
occurrence, the albedo assumes a constant value corresponding to ice albedo. However, after the fountain is
switched off, the albedo can reset to snow albedo during snowfall events and then decay back to ice albedo. We use
the scheme described in \cite{OerlemansKnap_1998} to model this process. The scheme records the decay of albedo
with time after fresh snow is deposited on the surface. $\delta t$ records the number of time steps after the last
snowfall event. After snowfall, albedo changes over a time step, $\delta t$ , as

\begin{equation} \alpha=\alpha_{ice}+(\alpha_{snow}-\alpha_{ice}) \cdot e^{(-\delta t)/\tau} \label{eqn:a}
\end{equation}

where $\alpha_{ice}$ is the bare ice albedo value (0.25), $\alpha_{snow}$ is the snow ice albedo value (0.85) and $\tau$
is a decay rate (16 $days$), which determines how fast the albedo of the ageing snow reaches this value.

The area fraction $f_{cone}$ of the ice structure exposed to the direct shortwave radiation depends on the shape
considered. Using the solar elevation angle $\theta_{sun}$, the solar beam can be considered to have a vertical
component, impinging on the horizontal surface (semicircular base of the AIR), and a horizontal component impinging on
the vertical cross section (a triangle). The solar elevation angle $\theta_{sun}$ used is modelled using the
parametrisation proposed by \cite{Woolf_1968}. Accordingly, $f_{cone}$ is determined as follows:

\begin{equation} \begin{split} f_{cone}& =\frac{(0.5 \cdot r_{ice} \cdot h_{ice}) \cdot cos \theta_{sun} +(\pi \cdot
			{r_{ice}}^2/2) \cdot sin \theta_{sun} }{\pi \cdot r_{ice} \cdot ({r_{ice}}^2+{h_{ice}}^2)^{1/2}}\\ \end{split}
	\label{eqn:f_{cone}} \end{equation}

The diffuse shortwave radiation is assumed to impact the conical AIR surface uniformly.

\subsubsection{Net Longwave Radiation \texorpdfstring{$q_{LW}$}{Lg}}

The net longwave radiation $q_{LW}$ is determined as follows:

\begin{equation} q_{LW}= LW_{in}-\sigma \cdot \epsilon_{ice} \cdot {(T_{ice}+ 273.15)}^4
	\label{eqn:LW} \end{equation}

where $T_{ice}$ is the modelled surface temperature, both temperatures are given in [$\degree C$],
$\sigma=5.67\cdot10^{-8}\,Jm^{-2}s^{-1}K^{-4}$ is the Stefan-Boltzmann constant, $LW_{in}$ denotes the incoming longwave
radiation and $\epsilon_{ice}$ is the corresponding emissivity value for the Icestupa surface (0.97).

The incoming longwave radiation $LW_{in}$for the Indian site, where no direct measurements were available, is determined
as follows:

\begin{equation} LW_{in}=\sigma \cdot (\epsilon_a \cdot {(T_a+ 273.15)}^4)
	\label{eqn:LWin} \end{equation}

here $T_a$ represents the measured air temperature and $\epsilon_a$ denotes the atmospheric emissivity. We approximate
atmospheric emissivity $\epsilon_a$ using the equation suggested by \cite{Brutsaert_1982}, considering air temperature
and vapor pressure (Eqn.  \ref{eqn:atm_e}). The vapor pressures over air and ice was obtained using Eqn. \ref{eqn:vp}.
The expression defined in \cite{Brutsaert_1975} for clear skies (first term in equation \ref{eqn:atm_e}) is extended
with the correction for cloudy skies after \cite{Brutsaert_1982} as follows:

\begin{equation} \epsilon_a=1.24 \cdot (\frac{p_{v,a}}{(T_a+273.15)})^{1/7}\cdot(1+0.22\cdot{c}^2) \label{eqn:atm_e}
\end{equation}

with a cloudiness index $c$, ranging from 0 for clear skies to 1 for complete overcast skies. For the Indian site, we
assume cloudiness to be negligible.

\subsubsection{Turbulent fluxes}

The turbulent sensible $q_{S}$ and latent heat $q_{L}$ fluxes are computed with the following expressions proposed by
\cite{Garratt_1992}:

\begin{equation} q_{S}=\mu_{cone}\cdot c_{a} \cdot \rho_{a} \cdot p_{a}/p_{0,a} \cdot \frac{\kappa^2 \cdot v_a \cdot
		(T_a-T_{ice})}{{(\ln{\frac{h_{AWS}}{z_{0}}})}^2} \label{eqn:qs} \end{equation}

\begin{equation} q_{L}=\mu_{cone}\cdot 0.623 \cdot L_s \cdot \rho_{a}/p_{0,a} \cdot \frac{\kappa^2 \cdot
	v_a(p_{v,a}-p_{v,ice})}{{(\ln{\frac{h_{AWS}}{z_{0}}})}^2} \end{equation}

where $h_{AWS}$ is the measurement height above the ground surface of the AWS (around $2\,m$ for all sites), $v_a$ is
the wind speed in [$m\,s^{-1}$], $c_a$ is the specific heat of air at constant pressure (1010 J $kg^{-1} K^{-1}$),
$\rho_{a}$ is the air density at standard sea level (1.29 $kg m^{-3}$), $p_{0,a}$ is the air pressure at standard sea
level (1013 $hPa$), $\kappa$ is the von Karman constant (0.4), $z_{0}$ is the surface roughness (5 $mm$) and $L_s$ is the heat of sublimation (2848 $kJ\,kg^{-1}$).
The vapor pressures over air ($p_{v,a}$) and ice ($p_{v,ice}$) was obtained using the following formulation given in
\cite{WMO_2018}:

\begin{equation} \begin{split} p_{v,a}&=6.107 \cdot 10^{(7.5 \cdot T_a / (T_a + 237.3))}\\ p_{v,ice}&=(1.0016 +
		3.15\cdot10^{-6}\cdot p_{a}-0.074\cdot p_{a}^{-1})\cdot(6.112 \cdot e^{(22.46 \cdot T_{ice} / (T_{ice} + 272.62))})
	\end{split} \label{eqn:vp} \end{equation}

where $p_{a}$ is the measured air pressure in [$hPa$].

The dimensionless parameter $\mu_{cone}$ is an "exposure/roughness parameter" that deals with the fact that AIR has a
rough appearance and forms an obstacle to the wind regime. This factor accounts for the larger turbulent fluxes due to
the roughness of the surface \cite{Oerlemans_2021}, and is a function of the AIR slope as follows:

\begin{equation}
	\mu_{cone} = 1 + \frac{s_{cone}}{2}
\end{equation}

A possible source of error is the fact that wind measurements from the horizontal plane at the AWS are used, which might
be different from those on a slope. However, without detailed datasets from the AIR surface, we retain this assumption.

\subsubsection{Fountain discharge heat flux \texorpdfstring{$q_{F}$}{Lg} }

The fountain water temperature $T_F$ is assumed to cool to 0 $\degree C$. Thus, the heat flux caused by this process is:

\begin{equation}
	q_{F} = \frac{ \Delta M_F \cdot c_{water} \cdot T_F}{\Delta t \cdot A}
	\label{eqn:qF}
\end{equation}
with $c_{water}$ as the specific heat of water.

\subsubsection{Bulk Icestupa heat flux \texorpdfstring{$q_{G}$}{Lg}} \label{sec:Bulkflux}

The bulk Icestupa heat flux $q_{G}$ corresponds to the ground heat flux in normal soils and is caused by the temperature
gradient between the surface layer ($T_{ice}$) and the ice body ($T_{bulk}$). It is expressed by using the heat
conduction equation as follows:

\begin{equation} q_{G} = k_{ice} \cdot (T_{bulk}-T_{ice}^{i-1})/l_{ice} \label{eqn:qG}    \end{equation}

where $k_{ice}$ is the thermal conductivity of ice (2.123 $W\, m^{-1}\,K^{-1}$) , $T_{bulk}$ is the mean temperature of
the ice body within the Icestupa and $l_{ice}$ is the average distance of any point in the surface to any other point in
the ice body. $T_{bulk}$ is initialised as 0 $\degree C$ and later determined from Eqn. \ref{eqn:qG} as follows:

\begin{equation} T_{bulk}^{i+1} = T_{bulk} - (q_{G} \cdot A \cdot \Delta t)/(M_{ice} \cdot c_{ice}) \end{equation}

Since AIRs typically have conical shapes with $r_{ice} > h_{ice}$, we assume that the center of mass of the ice body is
near the base of the fountain. Thus, the distance of every point in the AIR surface layer from the ice body's center of
mass is between $h_{ice}$ and $r_{ice}$. We calculate $q_{G}$ assuming $l_{ice} = (r_{ice} + h_{ice})/2$.


\subsection{Surface temperature}

The available energy $q_{surf}$ can act on the surface of the AIR to a) change its temperature, b) melt ice or c) freeze
ice.

So Eqn. \ref{eqn:EB} can be rewritten as: \begin{equation} q_{surf} = q_{freeze/melt} + q_{T} \end{equation} where
$q_{T}$, $q_{freeze}$ and $q_{melt}$ represent energy associated with process (a), (b) and (c) respectively.

We categorize the model time steps as freezing or melting events to distribute the surface energy flux into these three
components. Freezing/Melting events can only occur, if fountain water is available and the surface energy flux is
negative/positive. However, these two conditions are not sufficient as the latent heat energy can only contribute to
temperature fluctuations. Therefore, preventing latent heat energy from turning a melting event into a freezing event an
additional condition namely $(q_{surf}-q_{L}) < 0$ is required.

\begin{equation}
	q_{freeze/melt} = \left\{ \begin{array}{ll}
		q_{freeze} & \textit{ if } \Delta M_{F} > 0 \textit{ and } q_{surf} < 0 \textit{ and }(q_{surf}-q_{L}) < 0 \\
		q_{melt}   & \textit{ otherwise}
	\end{array} \right.
\end{equation}

During a freezing event, the AIR surface is assumed to warm to $0 \degree C$. The available energy $(q_{surf}-q_{L})$ is
further augmented due to this change in surface temperature represented by the energy flux $q_{0} = \frac{\rho_{ice}
		\cdot \Delta x \cdot c_{ice} \cdot T_{ice}^{i-1}}{\Delta t}$. The available energy can either be sufficient or
insufficient to freeze the fountain water available. If insufficient, the additional energy further cools down the
surface temperature. The surface energy flux distribution during a freezing event can be represented as:

\begin{equation}
	(q_{freeze}, q_{T}) = \left\{ \begin{array}{ll}
		(q_{surf}-q_{L}+q_{0}, q_{L}-q_{0}) & \textit{ if } \Delta M_{F} \geq -\frac{(q_{surf}-q_{L}+q_{0}) \cdot A \cdot \Delta
		t}{L_f}                                                                                                                  \\
		(\frac{\Delta M_{F} \cdot L_f
		}{A \cdot \Delta t}
		, q_{surf}+\frac{\Delta M_{F} \cdot L_f
		}{A \cdot \Delta t})                & \textit{ if } \Delta M_{F} < -\frac{(q_{surf}-q_{L}+q_0) A \cdot \Delta
		t}{L_f}
	\end{array} \right.
\end{equation}

During a melting event, the surface energy flux ($q_{surf}$) is first used to change the surface temperature to
$T_{temp}$ calculated as:

\begin{equation} T_{temp} =\frac{q_{surf} \cdot \Delta t}{\rho_{ice} \cdot c_{ice} \cdot \Delta x} + T_{ice} \end{equation}

If $T_{temp} > 0 \degree C$, then energy is reallocated from $q_{T}$ to $q_{melt}$ to maintain surface temperature at
melting point. The surface energy flux distribution during a melting event can be represented as:

\begin{equation}
	(q_{melt}, q_{T}) = \left\{ \begin{array}{ll}
		(0, q_{surf})                                                                                                                                                 & \textit{ if } T_{temp} < 0 \\
		(\frac{T_{temp} \cdot \rho_{ice} \cdot c_{ice} \cdot \Delta x}{\Delta t}, q_{surf}-\frac{T_{temp} \cdot \rho_{ice} \cdot c_{ice} \cdot \Delta x}{\Delta t}  ) & \textit{ if } T_{temp} > 0
	\end{array} \right.
\end{equation}


\subsection{Mass Balance}

The mass balance equation for an AIR is represented as:

\begin{equation}
	\frac{\Delta M_{F} + \Delta M_{ppt} + \Delta M_{dep}}{\Delta t} = \frac{\Delta M_{ice} +\Delta M_{water} +
		\Delta M_{sub} + \Delta M_{runoff}}{\Delta t}  \\
	\label{eq:MB}
\end{equation}

where $M_{F}$ is the discharge of the fountain; $M_{ppt}$ is the cumulative precipitation;  $M_{dep}$ is the cumulative
accumulation through water vapour deposition; $M_{ice}$ is the cumulative mass of ice; $M_{water}$ is the cumulative
mass of melt water; $M_{sub}$ represents the cumulative water vapor loss by sublimation and $M_{runoff}$ represents the
fountain discharge runoff that did not interact with the AIR. The LHS of equation \ref{eq:MB} represents the rate of
mass input and the RHS represents the rate of mass output for an AIR.

Precipitation input is calculated as shown in equation \ref{eq:ppt} where $\rho_{w}$ is the density of water (1000
$kg\,m^{-3}$), $ppt$ is the measured precipitation rate in [$m\,s^{-1}$] and $T_{ppt}$ is the temperature threshold
below which precipitation falls as snow. Here, snowfall events were identified using $T_{ppt}$ as $1 \degree C$. Snow
mass input is calculated by assuming a uniform deposition over the entire circular footprint of the AIR.

The latent heat flux is used to estimate either the evaporation and condensation processes or sublimation and deposition
processes as shown in equation \ref{eq:vap}. During time steps at which surface temperature is below 0 $\degree C$ only
sublimation and deposition can occur, but if the surface temperature reaches 0 $\degree C$, evaporation and condensation
can also occur. As the differentiation between evaporation and sublimation (and condensation and deposition) when the
air temperature reaches 0 $\degree C$ is challenging, we assume that negative (positive) latent heat fluxes correspond
only to sublimation (deposition), i.e. no evaporation (condensation) is calculated.

Since we have categorized every time step as a freezing and melting event, we can determine the meltwater and  ice
generated using the associated energy fluxes as shown in equations \ref{eq:mwat} and \ref{eq:mice}. Having calculated
all the other mass components the fountain wastewater generated every time step can be calculated using Eqn.
\ref{eq:MB}.

\begin{subequations}
	\label{equations}
	\begin{align}
		\label{eq:ppt}
		\frac{\Delta M_{ppt}}{\Delta t}                                    & = \left\{ \begin{array}{ll} \pi \cdot {r_{ice}}^2 \cdot
			\rho_{w}\cdot ppt & \textit{ if } T_{a} < T_{ppt} \\ 0 & \textit{ if } T_{a} \geq T_{ppt} \\\end{array} \right.                                      \\
		\label{eq:vap}
		(\frac{\Delta M_{dep}}{\Delta t}, \frac{\Delta M_{sub}}{\Delta t}) & = \left\{ \begin{array}{ll} \frac{q_{L}
			\cdot A}{L_s}\cdot (1,0)  & \textit{ if } q_{L} \geq 0 \\ \frac{q_{L}
			\cdot A}{L_s}\cdot (0,-1) & \textit{ if } q_{L} < 0    \\\end{array} \right.                                      \\
		\label{eq:mwat}
		\frac{\Delta M_{water}}{\Delta t}                                  & = \frac{q_{melt} \cdot A }{L_f}                                                   \\
		\label{eq:mice}
		\frac{\Delta M_{ice}}{\Delta t}                                    & = \frac{q_{freeze}\cdot A }{L_f} + \frac{\Delta M_{ppt}}{\Delta t} + \frac{\Delta
			M_{dep}}{\Delta t}- \frac{\Delta M_{sub}}{\Delta t}- \frac{\Delta M_{melt}}{\Delta t}
	\end{align}
\end{subequations}

We define the freezing rate $m_{freeze}$ and the melting rate $m_{melt}$ as follows:

\begin{equation}
	m_{freeze/melt} = \frac{q_{freeze/melt} \cdot A }{L_f}
	\label{eq:m_freeze/melt}
\end{equation}

To estimate the mass of any component at time step $i$, one can now sum the mass flux estimated above: \begin{equation}
	M_{comp}^i = \sum_{t=0}^{t=i} (\frac{\Delta M_{comp}}{\Delta t})_{t} + M_{comp}^0 \end{equation} where

\begin{equation} M_{comp}^0 = \left\{ \begin{array}{ll} -V_{dome} * \rho_{ice} & \textit{ if } M_{comp}=
             M_{ice}\textit{ or }
             M_{F}                                                 \\ 0 & \textit{ otherwise }\\\end{array} \right. \\
\end{equation}

Considering AIRs as water reservoirs, their storage efficiency ($SE$) can be defined as the percentage of ice and
meltwater produced as follows:

\begin{equation} \textit{SE} = \frac{M_{water}+M_{ice}}{(M_F+M_{ppt}+M_{dep})} \cdot 100 \end{equation}

\subsection{Sensitivity and uncertainty analysis}

We used a polynomial chaos expansion approach (as in \cite{uncertainpy_2018}; \cite{Xiu_2005}) to evaluate the
model sensitivity and uncertainty. Polynomial chaos expansion are a much more efficient way to obtain similar
results compared to the computationally demanding Monte Carlo methods. This approach approximates the model with a
polynomial (as a surrogate model), on which sensitivity and uncertainty analysis can be performed.  The surrogate
model produced was a polynomial of order 4. The rosenblatt transformation was also used since the parameters were
correlated.

The uncertainty in the model ice volume estimates are caused due to uncertainty in two sources, namely model
parameters and input. For the model parameters, we first fix a range based on literature values and then perform a
global sensitivity analysis (GSA) with the storage efficiency as the objective. The distribution is always treated
as uniform and the limits for every parameter are given in Table \ref{tab:parameters}. The GSA consists of a total
ensemble size of 2000 simulations per AIR. The parameter sensitivity results from the GSA are used as a tool to
reduce the number of free parameters in the model by identifying those parameters which have only a marginal
influence on the model output. The model is considered insensitive to parameters with a total sensitivity index
($S_{T_{i}}$) of $\leq 0.2$, and these parameters were fixed at the median value of the range shown in Table
\ref{tab:parameters} in subsequent model simulations.

The ranges for the snow albedo were taken from \cite{ZollesMaussion_2019}; ice albedo minimum was taken from
\cite{steiner_2015} and maximum from \cite{ZollesMaussion_2019}; albedo decay rate is assumed to have a minimum
value of 10 days similar to values obtained by \cite{Schmidt_2017} for wet surfaces and a maximum of 22 days from
\cite{OerlemansKnap_1998}; emissivity range from \cite{steiner_2015} and temperature threshold for precipitation
from \cite{Zhou_2010}. The surface area correction factor, $A_{corr}$, quantifies the deviation of the AIR shape
from the conical shape assumed in the model. We assume the range of this deviation to be between 1 to 2. Literature
values for surface roughness, $z_{0}$, of glacier ice are generally in the range $0.1-5\, mm$
\citet{BrockWillisSharp_2006}. Since the surface layer thickness, $\Delta x$, for an AIR does not bear resemblance
to any parameter in the glaciological literature, we attribute a wide range of values for it and try to constrain
it further during the calibration process.

For the model input data, uncertainty associated with two kinds of observations namely, weather and fountain were
evaluated. All the radiation measurements ($SW_{direct}, SW_{diffuse}, LW_{in}$) in the weather data were uncertain
since they were taken from ERA5 dataset or an AWS far away from the AIR site. All the measured fountain parameters are
uncertain since they vary significantly temporally and are not constant. The UAV measurements have also uncertainty
associated with them. So all these measurements were assumed to have an uncertainty of $\pm 10 \%$ to evaluate the
corresponding uncertainty produced in model ice volume estimates.

The sensitivity analysis was only carried out for the CH21 and IN21 AIR since CH20 AIR had too few ice volume
observations for calibration.

\begin{table}
	\centering
	\caption{The ranges of the 8 different parameters used in the sensitivity study.}
	\label{tab:parameters}
	\begin{tabular}{@{}llllll@{}}
		\toprule
		\textbf{No.} & \textbf{Name}                       & \textbf{Abbreviation} & \textbf{Min} & \textbf{Max} & \textbf{Unit} \\\midrule
		1            & Ice Emissivity                      & $\epsilon_{ice}$      & 0.95         & 0.99         &               \\
		2            & Ice Albedo                          & $\alpha_{ice}$        & 0.15         & 0.35         &               \\
		3            & Snow Albedo                         & $\alpha_{snow}$       & 0.8          & 0.9          &               \\
		4            & Precipitation Temperature threshold & $T_{ppt}$             & 0            & 2            & $\degree C$   \\
		5            & Albedo Decay Rate                   & $\tau$                & 10           & 22           & $days$        \\
		6            & Surface Roughness                   & $z_0$                 & 1            & 5            & $mm$          \\
		7            & Surface Area correction factor      & $A_{corr}$            & 1            & 2            &               \\
		8            & Surface layer thickness             & $\Delta x$            & 10           & 50           & $mm$          \\\bottomrule
	\end{tabular}
\end{table}

\section{Results}

\subsection{Sensitivity analysis}

The focus of the GSA is not on the absolute sensitivity towards single parameters, but rather to reduce the dimension
of the parameter space. Therefore, the following discussion is limited to two classes: parameters to which the model is
sensitive ($S_{T_{i}} > 0.2$) and non-sensitive ($S_{T_{i}} \leq 0.2$). For all the AIR, just three parameters were sensitive namely, $z_{0}$, $A_{corr}$ and $\Delta x$. Model simulations
were carried out varying all the sensitive parameters within the ranges defined in Table \ref{tab:parameters}.

Two objectives were chosen namely, ice volume and area, to determine the model performance with respect to these
sensitive parameters. The frequency distribution  of RMSE determined between the UAV observations (see Table
\ref{tab:uav}) and the model outputs among the best 10 \% runs are shown in Fig.\ref{fig:param_hist} for each
objective on each AIR.

\begin{figure}
	\begin{center}
		\includegraphics[width=\linewidth]{Figures/param_hist.jpg}
	\end{center}
	\caption{Observed ranges of the sensitive parameters used in the model optimization, shown by
		plotting the frequency distribution of the parameter values for the best 10 \% of the model runs. }
	\label{fig:param_hist} \end{figure}

\subsection{Calibration}

For model calibration, we choose values of sensitive parameters which populate more the 50 \% of the possibilities
among the best 10 \% of the model runs. $SA_{corr}$ shows different preferences whereas the other two parameters show similar preference for the two
objectives.

We expect the UAV measured surface area of AIR to be higher than the modelled cone area due to the many
ice features that tend to form. Particularly, we also expect the IN21 area to deviate significantly from CH21 AIR
since it was the product of two ice cones merged into one (see Fig. \ref{fig:2AIR}). This is clearly represented in
the disjoint frequency distribution of the area objective of the CH21 and IN21 AIR. Hence, we calibrate the IN21
area correction factor to 1.5 and CH21 area correction factor to 1.3.

We choose the model surface layer thickness based on two principles namely, (a) the ice thickness $\Delta x$ should
be small enough to represent the surface temperature variations every model time step $\Delta t$ and (b) $\Delta x$
should be large enough for these temperature variations to not reach the bottom of the surface layer. Since the
minimum surface temperature does not go beyond $-50 \, \degree C$ for the smallest possible thickness of $20\, mm$, we
choose this as the model surface layer thickness.

For the surface roughness, no clear preference is observed for the IN21 AIR so it is calibrated to the median
value of $3 \, mm$. For the CH21 AIR, we calibrate it to its preferred value of $1 \, mm$.

\subsection{Validation and Uncertainty analysis}

The validation of the model is carried out using the survival duration observation. The survival duration signifies
the time when all the ice had melted. Model performance can be judged based on the difference between the model
expectation and observation of survival duration. For the IN21, CH21 and CH20 AIR, this was 8 days, -1 day and -16 days
respectively where the negative (positive) sign represents underestimation (overestimation).

The uncertainty in the ice volume estimates caused by the insensitive model parameters are shown in Fig.
\ref{fig:results}. Model results for IN21 exhibit more confidence since precipitation and the associated variation
in albedo was negligible.

\begin{figure}
	\begin{center}
		\includegraphics[width=\linewidth]{Figures/icev_results.jpg}
	\end{center}
	\caption{Modelled ice volume during the lifetime of the AIR (blue curve). The prediction interval is based on the
		ice volume uncertainty caused by the insensitive parameters. The upper and lower bounds of the y axis represents
		maximum ice volume and dome volume respectively. Green points indicate the UAV ice volume observations.  The
		orange line represents the observed survival duration for each AIR.  }

	\label{fig:results}
\end{figure}


\subsection{AIR ice volume estimates}

The construction decisions responsible for the observed magnitude and variance of the ice volume estimates can be
categorised based on the the fountain used and location chosen. According to Eqn. \ref{eq:m_freeze/melt}, the
freezing/melting rate of the AIRs can be decomposed to the corresponding freezing/melting energy and the surface
area. The construction location chosen determines the freezing/melting energy flux through its weather and the
fountain determines the surface area through its spray radius.

\subsubsection{Location influence}

\begin{figure}
	\begin{center}
		\includegraphics[width=\linewidth]{Figures/albedo.jpg}
	\end{center}
	\caption{Some derived parameters of the model, namely, albedo and $f_{cone}$ (a), Surface temperature (b). In
		(a), the purple curve shows how snow and fountain spray reset albedo between ice albedo and snow albedo.  The
		decay of the snow albedo to ice albedo can also be observed. The orange curve shows how the solar radiation area
		fraction varied diurnally and seasonally with variations in the solar elevation angle. In (b), the surface
		temperature (blue curve) was forced to be 0 $\degree C$ during fountain runtime.}
	\label{fig:albedo}
\end{figure}

To further understand the influence of location, we analyse the daily surface normal thickness change and the
energy flux respectively. Fig.  \ref{fig:MEB} shows these daily fluxes calculated with calibrated parameters for
the first and last 20 days for each AIR. The two time periods selected are characteristic of the freezing and
melting period respectively. A strong variability is evident between the freezing and melting periods and between
the Swiss and the Indian AIR.

The magnitude of the thickness change in Fig. \ref{fig:MEB} are explained by their energy flux counterparts. Namely
ice mass flux corresponds to freezing energy available, melt mass flux corresponds to melting energy available,
snowfall is calculated directly from the precipitation quantity and sublimation/deposition quantities correspond to
the latent heat flux available. The rest of the energy fluxes are shown to represent the different physical
processes that contribute to this freezing and melting energy.

\begin{figure}
	\begin{center}
		\includegraphics[width=\linewidth]{Figures/mass_energy_bal.jpg} \end{center}
	\caption{Daily averages of mass and energy fluxes compared for all AIRs during their freezing and melting periods.
	} \label{fig:MEB}
\end{figure}

The largest contributor to the melting energy flux was the sensible heat flux for both the locations. The Indian
AIR melted gradually but the Swiss AIR melted rapidly on certain days (e.g.  day 140) due to the well known foehn
events (Reference). These foehn events produced meltwater in the Swiss location even during the freezing period
(e.g.  day 17).

Latent heat plays a fascinating role in this whole energy balance. It contributes to both mass and energy flux
simultaneously through deposition/sublimation processes. Sublimation was significantly greater in the Indian site
compared to the Swiss site as the corresponding latent heat fluxes were an order of magnitude larger (see Table
\ref{tab:Observations}). This was primarily due to the difference in relative humidity between the sites. This was
expected since glaciers near the Indian location have been hypothesized to lose a significant amount of mass
through sublimation by \cite{azam_2018}. Throughout the life cycle of the Indian AIR, mass was lost through
sublimation but the corresponding energy flux was used to freeze much more ice since the heat of vaporization is
much larger than the heat of fusion.

Longwave radiation compensates the shortwave radiation to a large degree during the melting period for both the
locations.

Direct shortwave radiation is more than 3 times higher for the Indian location (due to its higher altitude) but
there is no significant difference in the net shortwave radiation absorbed by both the AIR. This is because of the
higher diffuse shortwave radiation in the Swiss location compensates for its lower direct shortwave radiation (see
Table \ref{tab:Observations}).  Since the Indian location has mostly clear days, its diffuse shortwave radiation
was negligible. Moreover, the effect of the direct shortwave radiation of both the locations was dampened by the
area fraction $f_{cone}$. As shown in Fig. \ref{fig:albedo}, less than half of the AIR surface area was exposed to
direct shortwave radiation flux.  Albedo, on the other hand, only varied for the Swiss location since there was
negligible precipitation for the Indian site.

\begin{table}
	\centering
	\caption{ Summary of the mass balance, energy balance and AIR characteristics estimated by the model}
	\label{tab:Results}
	\begin{tabular}{@{}|llllll|@{}}
		\toprule
		\textbf{}              & \textbf{Name}           & \textbf{Symbol} & \textbf{IN21} & \textbf{CH21} &
		\textbf{Units}                                                                                                   \\ \midrule
		\multicolumn{1}{|l|}{\multirow{3}{*}{\rotatebox[origin=c]{90}{Input}}}
		                       & Fountain discharge      & $M_F$           & 2911          & 971           & $tons$      \\
		\multicolumn{1}{|l|}{} & Snowfall                & $M_{ppt}$       & 0             & 52            & $tons$      \\
		\multicolumn{1}{|l|}{} & Deposition              & $M_{dep}$       & 10            & 4             & $tons$      \\ \midrule
		\multicolumn{1}{|l|}{\multirow{4}{*}{\rotatebox[origin=c]{90}{Output}}}
		                       & Meltwater               & $M_{water}$     & 414           & 229           & $tons$      \\
		\multicolumn{1}{|l|}{} & Ice                     & $M_{ice}$       & 139           & 0             & $tons$      \\
		\multicolumn{1}{|l|}{} & Sublimation             & $M_{sub}$       & 121           & 10            & $tons$      \\
		\multicolumn{1}{|l|}{} & Fountain runoff         & $M_{runoff}$    & 2321          & 788           & $tons$      \\ \midrule
		\multicolumn{1}{|l|}{\multirow{8}{*}{\rotatebox[origin=c]{90}{Energy flux}}}
		                       & Shortwave radiation     & $q_{SW} $       & $ 31 \pm 51$  & $ 30 \pm 51$
		                       & $W\,m^{-2}$                                                                             \\
		\multicolumn{1}{|l|}{} & Longwave radiation      & $q_{LW} $       & $-77 \pm 32$  & $-53 \pm 36$  & $W\,m^{-2}$ \\
		\multicolumn{1}{|l|}{} & Sensible heat           & $q_{S}  $       & $87 \pm111$   & $38 \pm 99$   & $W\,m^{-2}$ \\
		\multicolumn{1}{|l|}{} & Latent heat             & $q_{L}  $       & $-47 \pm 63$  & $-6 \pm 39$   & $W\,m^{-2}$ \\
		\multicolumn{1}{|l|}{} & Fountain heat           & $q_{F}  $       & $2 \pm 3$     & $5 \pm 5$     & $W\,m^{-2}$ \\
		\multicolumn{1}{|l|}{} & Ground heat             & $q_{G}   $      & $0 \pm 1$     & $0 \pm 2$     & $W\,m^{-2}$ \\
		\multicolumn{1}{|l|}{} & Freezing energy         & $q_{freeze} $   & $-165\pm 46$  & $-71 \pm 48$  & $W\,m^{-2}$ \\
		\multicolumn{1}{|l|}{} & Melting energy          & $q_{melt}  $    & $90 \pm 84$   & $106\pm 140$  & $W\,m^{-2}$ \\
		\multicolumn{1}{|l|}{} & Temperature             & $q_{T}  $       & $0 \pm 110$   & $0 \pm 40$    & $W\,m^{-2}$ \\
		\multicolumn{1}{|l|}{} & Surface Area            & $A$             & $458 \pm 84$  & $134 \pm 65$  & $m^{2}$     \\\midrule
		\multicolumn{1}{|l|}{\multirow{4}{*}{\rotatebox[origin=c]{90}{Charecteristics}}}

		                       & Freezing rate           & $m_{freeze}$    & $14 \pm 7$    & $2 \pm 2$     & $l/min$     \\
		\multicolumn{1}{|l|}{} & Melting rate            & $m_{melt}$      & $2 \pm 6$     & $1 \pm 2$     & $l/min$     \\
		\multicolumn{1}{|l|}{} & Storage Efficiency      & SE              & 19            & 23            & \%          \\
		\multicolumn{1}{|l|}{} & Root mean squared error & RMSE            & 84            & 12 $m^{3}$    & $m^{3}$     \\\bottomrule
	\end{tabular}
\end{table}

\subsubsection{Fountain influence}

The fountain has some influence on the energy fluxes through its water temperature ($q_{F}$) and the albedo forcing
($q_{SW}$). However, this influence is minimal compared to the influence of surface area on the ice volume
($r^2=0.6$). The fountain determines the surface area through its spray radius during the freezing period. The
variance of this surface area is quite low in the freezing period since the ice radius is initialised and bounded
by the spray radius. So the thickness rate is uniformly scaled to produce the corresponding ice volume during the
freezing period.

To understand the fountain influence on the energy fluxes in the freezing period, we select days when the
fountain was switched off and compare it to the rest. Such days (e.g. day 13) are present in IN21 AIR since
fountain discharge was interrupted due to pipeline freezing events. Here, one can observe how latent heat is
favoured over sensible heat by the fountain. During fountain runtime, the surface temperature is maintained at 0
$\degree C$ which reduces (increases) the temperature difference (vapour flux) between the AIR surface and the
atmosphere, thereby reducing (increasing) the sensible heat (latent heat) flux.

\section{Discussion}

\subsection{Freezing and melting rates}

The Indian location was favourable because the mean freezing energy was around 2 times larger in magnitude than the
melting energy (see Table \ref{tab:Results}) . This enabled the fountain determined surface area to favor freezing over melting. However, the
Swiss location was not favourable since the mean freezing energy was lower than the mean melting energy, indicating that the
surface area favored melting over the freezing process.

Thus, the more than 3 times higher surface area of the fountain and 2 times higher freezing energy of the Indian
location result in more than 5 times higher Indian AIR ice volume compared to the Swiss AIR even though the Swiss
fountain runtime was roughly twice that of the Indian one as shown in Table \ref{tab:Observations}.

\subsection{Storage efficiency}

The storage efficiency of IN21 and CH21 AIR were $19\%$ and $23\%$ respectively. The low SE is a product of the
high  fountain water losses ( $\frac{M_{runoff}}{M_{input}}> 75 \%$ ) in all the AIR. This indicates that just
decreasing the mean fountain discharge could have significantly increased water storage efficiency.  The maximum
freezing rate of the IN21 AIR was less than half the mean fountain discharge. The Swiss AIR was able to attain the
mean fountain discharge provided but this was only for 6 hours from the 2155 hours of fountain runtime available.

\section{Conclusions}

In this paper, we have developed a bulk energy and mass balance model to simulate AIR evolution using data from
field measurements in the Indian Himalayas and the Swiss Alps. The use of this dataset, in combination with novel
algorithms for calculation of the surface area and freezing energy fluxes, allowed an accurate representation of
the complex growth dynamics typical of any AIR. The model was calibrated with ice volume and surface area
observations obtained via UAV flights. We calculated the storage efficiency, freezing and melting rates for each of
the three AIRs and explained their corresponding values through the influence of the location chosen and the
fountain used.

The results suggest that drier locations (e.g. Ladakh, Indian Himalayas) are significantly more favourable since
their freezing and melting rates are augmented and dampened by sublimation respectively.

Storage efficiency of all the AIR are poor but can be optimized significantly through control of the fountain
discharge rate.  The model results indicate that more than 75\% of the fountain water is lost as runoff for all the
AIRs.  Further experiments with different fountains are required to better understand the influence of the fountain
discharge rate on the results.

\section{Appendix}

\subsection{UAV data processing} \label{sec:uav}
The UAV flew automatically along the flight course predefined by Pix4Dcapture
(https://www.pix4d.com/product/pix4dcapture) and took photographs at a certain time interval. The position and
altitude of the UAV at the exposure stations, which were obtained by the built-in integrated Position and
Orientation System (POS, composed of global positioning system and inertial measurement units), were recorded in
the JPEG pictures. UAV images in each survey were separately processed with Pix4Dmapper in a three-step workflow,
which is described below:

(1) Initial processing: This process generates a sparse point cloud with the structure-from motion algorithm
(\cite{Turner_2012}). First, it searches for and matches key points in the photos that have certain overlapping
areas using a feature matching algorithm (e.g. the scale-invariant feature transform (SIFT) algorithm, which can
detect key points in photos with different views and illumination conditions; \cite{Lowe_2004}). Second, the
approximate locations and orientations of the camera at each exposure station are reconstructed with the internal
parameters (focal length, coordinates of the principal point of the photograph), and external parameters (i.e. POS
data). A sparse point cloud is created.

(2) Point cloud densification: In this step, the multi-view stereo technique is applied to achieve a higher point
cloud density than in the previous step (\cite{Furukawa_2010}; \cite{Molg_2017}). Thus, the spatial resolution of
the products can be increased, and an irregular network for the next step can be created (\cite{Kung_2011}).

(3) AIR delineation: Ice radius, area and volume are the three main final products. Perimeter was manually marked
on the point cloud by identifying the AIR boundary. For the Indian location, we identified identical rock features
near the ice boundary to mark as vertices of this perimenter. For the Swiss AIR, no such feature was available due
to snowfall, so instead the perimeter was identified by identifying the ice and snow boundary through guidance of
the constant fountain location. During this process, we found photos with too much snow coverage that cannot be matched, that is, few feature
points can be detected and matched in these photos, especially in the surveys of the Swiss AIR, where precipitation
was high. We attach a high uncertainty of $\pm 10 \%$ for all the AIR observations to accomodate for this.

\begin{table}
	\centering
	\caption{ Summary of the UAV observations}
	\label{tab:uav}
	\begin{tabular}{@{}|llllll|@{}}
		\toprule
		\textbf{}              & \textbf{No.} & \textbf{Date} & \textbf{Volume} & \textbf{Radius} & \textbf{Area} \\ \midrule
		\multicolumn{1}{|l|}{\multirow{6}{*}{\rotatebox[origin=c]{90}{IN21}}}
		                       & 1            & Jan 18, 2021  & 103 $m^{3}$     & 9.1 $m$
		                       & 411 $m^{2}$                                                                      \\
		\multicolumn{1}{|l|}{} & 2            & Feb 27, 2021  & 580 $m^{3}$     & 10.2 $m$
		                       & 668 $m^{2}$                                                                      \\
		\multicolumn{1}{|l|}{} & 3            & Mar 3, 2021   & 626 $m^{3}$     & 10.3 $m$
		                       & 694 $m^{2}$                                                                      \\
		\multicolumn{1}{|l|}{} & 4            & Mar 15, 2021  & 692 $m^{3}$     & 10 $m$
		                       & 681 $m^{2}$                                                                      \\
		\multicolumn{1}{|l|}{} & 5            & Mar 26, 2021  & 582 $m^{3}$     & 10.2 $m$
		                       & 671 $m^{2}$                                                                      \\
		\multicolumn{1}{|l|}{} & 6            & Apr 3, 2021   & 620 $m^{3}$     & 10.1 $m$
		                       & 658 $m^{2}$
		\\\midrule
		\multicolumn{1}{|l|}{\multirow{8}{*}{\rotatebox[origin=c]{90}{CH21}}}
		                       & 1            & Nov 22, 2020  & 13 $m^{3}$      & 5.4 $m$
		                       & 136$m^{2}$                                                                       \\
		\multicolumn{1}{|l|}{} & 2            & Dec 2, 2020   & 26 $m^{3}$      & 5.7 $m$
		                       & 118$m^{2}$                                                                       \\
		\multicolumn{1}{|l|}{} & 3            & Dec 30, 2020  & 43 $m^{3}$      & 7.5 $m$
		                       & 189$m^{2}$                                                                       \\
		\multicolumn{1}{|l|}{} & 4            & Jan 9, 2021   & 82 $m^{3}$      & 6.5 $m$
		                       & 150$m^{2}$                                                                       \\
		\multicolumn{1}{|l|}{} & 5            & Mar 6, 2021   & 108 $m^{3}$     & 7.5 $m$
		                       & 183$m^{2}$                                                                       \\
		\multicolumn{1}{|l|}{} & 6            & Apr 2, 2021   & 83 $m^{3}$      & 6.5 $m$
		                       & 150$m^{2}$                                                                       \\
		\multicolumn{1}{|l|}{} & 7            & Apr 16, 2021  & 64 $m^{3}$      & 6.2 $m$
		                       & 134$m^{2}$                                                                       \\
		\multicolumn{1}{|l|}{} & 8            & Apr 24, 2021  & 37 $m^{3}$      & 4.7 $m$
		                       & 80 $m^{2}$                                                                       \\
		\midrule
		\multicolumn{1}{|l|}{\multirow{2}{*}{\rotatebox[origin=c]{90}{CH20}}}
		                       & 1            & Jan 3, 2020   & 24 $m^{3}$      & 6.7 $m$
		                       & 170 $m^{2}$                                                                      \\
		\multicolumn{1}{|l|}{} & 2            & Jan 24, 2020  & 59 $m^{3}$      & 7.7 $m$
		                       & 228 $m^{2}$                                                                      \\
		\midrule
	\end{tabular}

\end{table}

\subsection{Mean discharge estimation} \label{sec:discharge}

\section*{Conflict of Interest Statement} The authors declare that the research was conducted in the absence of any
commercial or financial relationships that could be construed as a potential conflict of interest.

\section*{Author Contributions} SB wrote the initial version of the manuscript. SB developed the methodology with
inputs from MH, ML and JO. MH and ML reviewed the model algorithm and helped improve it. JB reviewed the model
code.  SB, MH and SW participated in the fieldwork. ML, JO, FK and JB reviewed the initial manuscript.

\section*{Funding} This work was supported and funded by the University of Fribourg and by the Swiss Government
Excellence Scholarship (SB). The associated field work in India was supported by Himalayan Institute of
Alternatives and funded by the Swiss Polar Institute.

\section*{Acknowledgments} This work would not have been possible without the untiring effort of the Swiss and
Indian icestupa construction teams through the winters of 2019, 2020 and 2021. We thank Mr. Adolf Kaeser and Mr.
Flavio Catillaz from Eispalast Schwarzsee (CH19); Daniel Beurki from the Guttannen Bewegt Association (CH20 and
CH21); Norboo Thinles, Nishant Tiku, Sourabh Maheshwari and the whole icestupa competition team from HIAL (IN21).
We would also like to thank Hanseuli Gubler for designing the Swiss AWS and Digmesa AG for subsidising their
flowmeter used in the experiment.  We would particularly like to thank the editor Prof. Thomas Schuler and 2
anonymous reviewers who gave us important inputs to improve the paper. We also thank Prof. Christian Hauck, Prof.
Nanna B. Karlsson and Dr.  Andrew Tedstone for valuable suggestions that improved the manuscript.

\section*{Data Availability Statement} Some AIR timelapses and results can be viewed interactively in this
\href{https://share.streamlit.io/gayashiva/air_model/src/visualization/webApp.py}{app}.  The model code used is
available in \href{https://github.com/Gayashiva/air_model}{GitHub}. The UAV data can be obtained from the authors upon
request.

\bibliographystyle{frontiersinSCNS_ENG_HUMS} \bibliography{references}

\end{document}
